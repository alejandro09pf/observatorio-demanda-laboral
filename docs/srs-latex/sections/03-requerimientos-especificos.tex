\chapter{REQUERIMIENTOS ESPECÍFICOS}

Este capítulo detalla de manera exhaustiva los requerimientos funcionales y no funcionales del sistema, organizados según las categorías definidas en las secciones anteriores.

\section{Requerimientos de Interfaces Externas}

\subsection{Interfaces con el Usuario}

\begin{description}
    \item[REI-01] El sistema debe permitir la ejecución de scripts desde CLI para scraping y visualización.
    \begin{itemize}
        \item \textbf{Prioridad:} Alta
        \item \textbf{Módulo:} Interfaz Usuario
        \item \textbf{Criterio:} Ejecución funcional por línea de comandos
    \end{itemize}

    \item[REI-02] El sistema debe incluir notebooks Jupyter con celdas documentadas.
    \begin{itemize}
        \item \textbf{Prioridad:} Media
        \item \textbf{Módulo:} Interfaz Usuario
        \item \textbf{Criterio:} Visualización y reproducción de notebooks
    \end{itemize}

    \item[REI-03] El sistema debe generar reportes en formato PDF, PNG y HTML.
    \begin{itemize}
        \item \textbf{Prioridad:} Alta
        \item \textbf{Módulo:} Interfaz Usuario/Visualización
        \item \textbf{Criterio:} Visualización correcta de reportes generados
    \end{itemize}
\end{description}

\subsection{Interfaces con el Hardware}

\begin{description}
    \item[REI-04] El sistema debe operar en equipos sin GPU dedicada y mínimo 8 GB de RAM.
    \begin{itemize}
        \item \textbf{Prioridad:} Alta
        \item \textbf{Módulo:} Interfaz Hardware
        \item \textbf{Criterio:} Ejecución sin errores en equipos personales
    \end{itemize}
\end{description}

\subsection{Interfaces con el Software}

\begin{description}
    \item[REI-05] El sistema debe conectarse a una base de datos PostgreSQL local.
    \begin{itemize}
        \item \textbf{Prioridad:} Alta
        \item \textbf{Módulo:} Interfaz Software
        \item \textbf{Criterio:} Inserción y lectura desde PostgreSQL
    \end{itemize}

    \item[REI-06] El sistema debe acceder a portales web por HTTP/HTTPS para extraer datos.
    \begin{itemize}
        \item \textbf{Prioridad:} Alta
        \item \textbf{Módulo:} Interfaz Comunicación
        \item \textbf{Criterio:} Scraping exitoso desde URLs definidas
    \end{itemize}
\end{description}

\section{Requerimientos Funcionales}

\subsection{Funcionalidad 1: Extracción de Vacantes (Scraping)}

\begin{description}
    \item[RF-01] El sistema debe extraer vacantes desde portales como Computrabajo, Bumeran y elempleo.com.
    \begin{itemize}
        \item \textbf{Prioridad:} Alta
        \item \textbf{Módulo:} Scraping
        \item \textbf{Criterio:} Extracción visible y consistente de datos
    \end{itemize}
\end{description}

\subsection{Funcionalidad 2: Procesamiento de Texto}

\begin{description}
    \item[RF-02] El sistema debe almacenar las vacantes en PostgreSQL con deduplicación SHA256.
    \begin{itemize}
        \item \textbf{Prioridad:} Alta
        \item \textbf{Módulo:} Almacenamiento
        \item \textbf{Criterio:} Consultas e inserciones validadas, duplicados detectados
        \item \textbf{Estado:} IMPLEMENTADO (23,352 jobs, dedup rate 0.5\%)
    \end{itemize}

    \item[RF-03] El sistema debe preprocesar el texto: limpieza HTML, tokenización, normalización.
    \begin{itemize}
        \item \textbf{Prioridad:} Alta
        \item \textbf{Módulo:} Procesamiento NLP
        \item \textbf{Criterio:} Verificación de campos procesados, detección de jobs basura
        \item \textbf{Estado:} IMPLEMENTADO (23,188 jobs limpios, 99.5\% usable)
    \end{itemize}

    \item[RF-04] El sistema debe extraer habilidades explícitas mediante NER y Regex (Pipeline A).
    \begin{itemize}
        \item \textbf{Prioridad:} Alta
        \item \textbf{Módulo:} Extracción
        \item \textbf{Criterio:} Skills extraídas con método y confianza asociados
        \item \textbf{Métodos:}
        \begin{itemize}
            \item Regex: 200+ patrones tecnológicos (78-89\% precision)
            \item NER: spaCy + custom entity ruler (13\% precision con filtros)
        \end{itemize}
        \item \textbf{Estado:} IMPLEMENTADO (Test 100 jobs: 2,756 skills extraídas)
    \end{itemize}
\end{description}

\subsection{Funcionalidad 2.1: Mapeo contra Taxonomía ESCO}

\begin{description}
    \item[RF-04.1] El sistema debe cargar y mantener una taxonomía unificada de 14,174 skills.
    \begin{itemize}
        \item \textbf{Prioridad:} Alta
        \item \textbf{Módulo:} Taxonomía (Fase 0)
        \item \textbf{Criterio:} ESCO v1.1.0 (13,939) + O*NET Hot Tech (152) + Manual Curated (83)
        \item \textbf{Estado:} IMPLEMENTADO
    \end{itemize}

    \item[RF-04.2] El sistema debe mapear skills extraídas contra ESCO usando estrategia de tres capas.
    \begin{itemize}
        \item \textbf{Prioridad:} Alta
        \item \textbf{Módulo:} Matching
        \item \textbf{Criterio:} Layer 1 (Exact, SQL ILIKE, confidence 1.0) → Layer 2 (Fuzzy, threshold 0.85) → Layer 3 (Semantic, threshold 0.87, DESHABILITADO)
        \item \textbf{Estado:} PARCIALMENTE IMPLEMENTADO (Layer 1 + 2 activos, Layer 3 disabled)
        \item \textbf{Match rate actual:} 12.6\% (esperado para taxonomías 2016-2017)
    \end{itemize}

    \item[RF-04.3] El sistema debe identificar y rastrear skills emergentes no mapeadas.
    \begin{itemize}
        \item \textbf{Prioridad:} Alta
        \item \textbf{Módulo:} Emergent Skills Tracking
        \item \textbf{Criterio:} Skills sin match ESCO almacenadas con frecuencia y contexto
        \item \textbf{Estado:} IMPLEMENTADO (87.4\% emergent rate en test de 100 jobs)
    \end{itemize}

    \item[RF-04.4] El sistema debe generar embeddings semánticos para todas las skills.
    \begin{itemize}
        \item \textbf{Prioridad:} Alta
        \item \textbf{Módulo:} Embeddings (Fase 0)
        \item \textbf{Criterio:} Modelo intfloat/multilingual-e5-base (768D), L2-normalized
        \item \textbf{Performance:} 721 skills/segundo (GPU), 30 skills/segundo (CPU)
        \item \textbf{Estado:} IMPLEMENTADO (14,133 embeddings, 94.6\% test pass)
    \end{itemize}

    \item[RF-04.5] El sistema debe construir y mantener índice FAISS para búsqueda semántica.
    \begin{itemize}
        \item \textbf{Prioridad:} Alta
        \item \textbf{Módulo:} FAISS Index (Fase 0)
        \item \textbf{Criterio:} IndexFlatIP (exact search), 30,147 queries/segundo
        \item \textbf{Archivos:} data/embeddings/esco.faiss (41.41 MB), esco\_mapping.pkl (545 KB)
        \item \textbf{Estado:} IMPLEMENTADO (25x más rápido que PostgreSQL pgvector)
    \end{itemize}
\end{description}

\subsection{Funcionalidad 3: Representación Semántica}

\begin{description}
    \item[RF-05] El sistema debe generar representaciones semánticas (embeddings) y realizar clustering automático.
    \begin{itemize}
        \item \textbf{Prioridad:} Alta
        \item \textbf{Módulo:} Agrupamiento
        \item \textbf{Criterio:} Clústeres coherentes generados
    \end{itemize}
\end{description}

\subsection{Funcionalidad 4: Visualización}

\begin{description}
    \item[RF-06] El sistema debe generar visualizaciones estáticas con gráficas de frecuencia de habilidades y comparativas.
    \begin{itemize}
        \item \textbf{Prioridad:} Alta
        \item \textbf{Módulo:} Visualización
        \item \textbf{Criterio:} Visualizaciones exportadas exitosamente
    \end{itemize}
\end{description}

\section{Requerimientos de Desempeño}

\begin{description}
    \item[RD-01] El sistema debe extraer y almacenar vacantes desde múltiples portales de empleo.
    \begin{itemize}
        \item \textbf{Prioridad:} Alta
        \item \textbf{Módulo:} Scraping
        \item \textbf{Criterio:} Mínimo 300 vacantes por país desde dos portales distintos
        \item \textbf{Performance actual:} 23,352 jobs scraped (hiring.cafe: 23,313, elempleo: 38, zonajobs: 1)
        \item \textbf{Estado:} CUMPLIDO (7,784\% sobre mínimo requerido)
    \end{itemize}

    \item[RD-02] El preprocesamiento textual debe ejecutarse sin errores críticos.
    \begin{itemize}
        \item \textbf{Prioridad:} Alta
        \item \textbf{Módulo:} Procesamiento
        \item \textbf{Criterio:} Pipeline completado con success rate > 95\%
        \item \textbf{Performance actual:} 99.5\% usable rate (23,188/23,352 jobs)
        \item \textbf{Estado:} CUMPLIDO
    \end{itemize}

    \item[RD-03] La generación de embeddings debe completarse en tiempo razonable.
    \begin{itemize}
        \item \textbf{Prioridad:} Alta
        \item \textbf{Módulo:} Embeddings (Fase 0)
        \item \textbf{Criterio:} Completar 14K skills en < 5 minutos con GPU
        \item \textbf{Performance actual:} 19.65 segundos para 14,133 skills (721 skills/seg)
        \item \textbf{Estado:} CUMPLIDO (15x mejor que requerimiento)
    \end{itemize}

    \item[RD-04] La búsqueda semántica con FAISS debe ser eficiente.
    \begin{itemize}
        \item \textbf{Prioridad:} Alta
        \item \textbf{Módulo:} FAISS Index
        \item \textbf{Criterio:} Mínimo 100 queries por segundo
        \item \textbf{Performance actual:} 30,147 queries/segundo (301x sobre requerimiento)
        \item \textbf{Estado:} CUMPLIDO (25x más rápido que PostgreSQL pgvector)
    \end{itemize}

    \item[RD-05] El sistema de extracción de skills debe procesar jobs sin fallos.
    \begin{itemize}
        \item \textbf{Prioridad:} Alta
        \item \textbf{Módulo:} Extracción Pipeline A
        \item \textbf{Criterio:} Success rate > 95\% en procesamiento
        \item \textbf{Performance actual:} 100\% success rate en test de 100 jobs (2,756 skills extraídas)
        \item \textbf{Tiempo promedio:} 1.82 segundos por job
        \item \textbf{Estado:} CUMPLIDO
    \end{itemize}

    \item[RD-06] El matching contra ESCO debe completarse para todas las skills extraídas.
    \begin{itemize}
        \item \textbf{Prioridad:} Alta
        \item \textbf{Módulo:} Matching (3-layer strategy)
        \item \textbf{Criterio:} Todas las skills procesadas por las 3 layers
        \item \textbf{Performance actual:} 12.6\% match rate (Layer 1: 5.4\%, Layer 2: 7.1\%, Layer 3: disabled)
        \item \textbf{Estado:} PARCIALMENTE CUMPLIDO (Layer 3 deshabilitado temporalmente)
        \item \textbf{Nota:} 87.4\% emergent skills es esperado para taxonomías 2016-2017
    \end{itemize}
\end{description}

\section{Restricciones de Diseño}

\begin{description}
    \item[RDZ-01] El sistema debe ejecutarse completamente en local, sin servicios web de pago.
    \begin{itemize}
        \item \textbf{Prioridad:} Alta
        \item \textbf{Módulo:} Arquitectura General
        \item \textbf{Criterio:} Funciona sin acceso a servicios externos
    \end{itemize}

    \item[RDZ-02] La ejecución de modelos LLM se limitará a versiones descargables.
    \begin{itemize}
        \item \textbf{Prioridad:} Alta
        \item \textbf{Módulo:} Enriquecimiento
        \item \textbf{Criterio:} Modelos configurados desde Hugging Face
    \end{itemize}

    \item[RDZ-03] El desarrollo deberá realizarse en Python 3.10 o superior.
    \begin{itemize}
        \item \textbf{Prioridad:} Alta
        \item \textbf{Módulo:} Infraestructura
        \item \textbf{Criterio:} Repositorio sin dependencias privativas
    \end{itemize}

    \item[RDZ-04] No se implementará una interfaz gráfica interactiva.
    \begin{itemize}
        \item \textbf{Prioridad:} Alta
        \item \textbf{Módulo:} Visualización
        \item \textbf{Criterio:} El sistema exporta reportes, no tiene frontend
    \end{itemize}

    \item[RDZ-05] El sistema debe ser modular.
    \begin{itemize}
        \item \textbf{Prioridad:} Alta
        \item \textbf{Módulo:} Arquitectura General
        \item \textbf{Criterio:} Cada módulo puede lanzarse individualmente
    \end{itemize}
\end{description}

\section{Atributos del Sistema de Software (No funcionales)}

\subsection{Confiabilidad}

\begin{description}
    \item[NFA-01] El sistema debe producir resultados consistentes ante entradas iguales.
    \begin{itemize}
        \item \textbf{Prioridad:} Alta
        \item \textbf{Módulo:} Todos
        \item \textbf{Criterio:} Repetición del proceso genera mismos resultados
    \end{itemize}

    \item[NFA-02] Se debe registrar el comportamiento del sistema mediante logs detallados.
    \begin{itemize}
        \item \textbf{Prioridad:} Alta
        \item \textbf{Módulo:} Todos
        \item \textbf{Criterio:} Archivos de log por módulo
    \end{itemize}

    \item[NFA-03] En caso de interrupciones, los módulos deben permitir ser reiniciados.
    \begin{itemize}
        \item \textbf{Prioridad:} Alta
        \item \textbf{Módulo:} Arquitectura General
        \item \textbf{Criterio:} Pipeline puede reiniciarse parcialmente
    \end{itemize}
\end{description}

\subsection{Disponibilidad}

\begin{description}
    \item[NFA-04] El sistema estará disponible para ejecución local en cualquier momento.
    \begin{itemize}
        \item \textbf{Prioridad:} Alta
        \item \textbf{Módulo:} Infraestructura
        \item \textbf{Criterio:} Pipeline completo corre offline
    \end{itemize}

    \item[NFA-05] Scripts y notebooks deben estar organizados en GitHub.
    \begin{itemize}
        \item \textbf{Prioridad:} Alta
        \item \textbf{Módulo:} Infraestructura
        \item \textbf{Criterio:} Repositorio contiene notebooks funcionales
    \end{itemize}
\end{description}

\subsection{Seguridad}

\begin{description}
    \item[NFA-06] Se evitará recolectar información personal.
    \begin{itemize}
        \item \textbf{Prioridad:} Alta
        \item \textbf{Módulo:} Scraping
        \item \textbf{Criterio:} Ningún campo sensible almacenado
    \end{itemize}

    \item[NFA-07] Los spiders implementarán throttling.
    \begin{itemize}
        \item \textbf{Prioridad:} Alta
        \item \textbf{Módulo:} Scraping
        \item \textbf{Criterio:} Tiempo entre requests configurable
    \end{itemize}

    \item[NFA-08] El código incluirá controles básicos de errores.
    \begin{itemize}
        \item \textbf{Prioridad:} Alta
        \item \textbf{Módulo:} Todos
        \item \textbf{Criterio:} Pipeline continúa sin detenerse
    \end{itemize}
\end{description}

\subsection{Mantenibilidad}

\begin{description}
    \item[NFA-09] El sistema debe estar documentado a nivel de código.
    \begin{itemize}
        \item \textbf{Prioridad:} Alta
        \item \textbf{Módulo:} Todos
        \item \textbf{Criterio:} Documentación presente en repositorio
    \end{itemize}

    \item[NFA-10] Cada módulo será independiente y versionado.
    \begin{itemize}
        \item \textbf{Prioridad:} Alta
        \item \textbf{Módulo:} Arquitectura Modular
        \item \textbf{Criterio:} Se puede actualizar un módulo sin afectar demás
    \end{itemize}
\end{description}

\subsection{Portabilidad}

\begin{description}
    \item[NFA-11] El sistema debe poder ejecutarse en Linux, macOS o Windows con WSL.
    \begin{itemize}
        \item \textbf{Prioridad:} Alta
        \item \textbf{Módulo:} Infraestructura
        \item \textbf{Criterio:} Se ejecuta en tres sistemas operativos
    \end{itemize}

    \item[NFA-12] Se debe proporcionar un archivo de entorno reproducible.
    \begin{itemize}
        \item \textbf{Prioridad:} Alta
        \item \textbf{Módulo:} Infraestructura
        \item \textbf{Criterio:} Archivo requirements.txt ejecutado con éxito
    \end{itemize}
\end{description}

\section{Requerimientos de la Base de Datos}

\begin{description}
    \item[BD-01] El sistema debe utilizar PostgreSQL como sistema de gestión de base de datos.
    \begin{itemize}
        \item \textbf{Prioridad:} Alta
        \item \textbf{Módulo:} Almacenamiento
        \item \textbf{Criterio:} PostgreSQL 13+ con soporte para arrays REAL[]
        \item \textbf{Estado:} IMPLEMENTADO
    \end{itemize}

    \item[BD-02] La base de datos debe almacenar vacantes con deduplicación SHA256.
    \begin{itemize}
        \item \textbf{Prioridad:} Alta
        \item \textbf{Módulo:} Almacenamiento
        \item \textbf{Criterio:} Hash único por job\_description para prevenir duplicados
        \item \textbf{Estado:} IMPLEMENTADO (0.5\% dedup rate)
    \end{itemize}

    \item[BD-03] La base de datos debe almacenar skills extraídas con método y confianza.
    \begin{itemize}
        \item \textbf{Prioridad:} Alta
        \item \textbf{Módulo:} Extracción
        \item \textbf{Criterio:} Tabla job\_extracted\_skills con campos: job\_id, skill, method, confidence
        \item \textbf{Estado:} IMPLEMENTADO
    \end{itemize}

    \item[BD-04] La base de datos debe almacenar embeddings de 768 dimensiones.
    \begin{itemize}
        \item \textbf{Prioridad:} Alta
        \item \textbf{Módulo:} Embeddings
        \item \textbf{Criterio:} Tabla skill\_embeddings con columna embedding REAL[768]
        \item \textbf{Estado:} IMPLEMENTADO (14,133 embeddings almacenados)
    \end{itemize}

    \item[BD-05] La base de datos debe mantener trazabilidad de matching ESCO.
    \begin{itemize}
        \item \textbf{Prioridad:} Alta
        \item \textbf{Módulo:} Matching
        \item \textbf{Criterio:} Tabla job\_esco\_matches con layer, confidence score, skill\_id
        \item \textbf{Estado:} IMPLEMENTADO
    \end{itemize}
\end{description}

\section{Trazabilidad de Requerimientos}

La trazabilidad entre los objetivos del proyecto y los requerimientos específicos se presenta en la siguiente tabla:

\begin{table}[H]
\centering
\begin{tabular}{|p{3cm}|p{10cm}|}
\hline
\textbf{Objetivo} & \textbf{Requerimientos Relacionados} \\
\hline
Extracción automatizada de vacantes & RF-01, RD-01, NFA-06, NFA-07, REI-06 \\
\hline
Procesamiento y limpieza textual & RF-02, RF-03, RD-02, BD-01, BD-02 \\
\hline
Extracción de habilidades & RF-04, RF-04.1, RF-04.2, RF-04.3, RD-05, RD-06, BD-03, BD-05 \\
\hline
Generación de embeddings & RF-04.4, RF-04.5, RD-03, RD-04, BD-04 \\
\hline
Clustering de perfiles & RF-05, RD-03 \\
\hline
Visualización de resultados & RF-06, REI-03, RDZ-04 \\
\hline
Arquitectura modular & RDZ-01, RDZ-03, RDZ-05, NFA-10 \\
\hline
Mantenibilidad y reproducibilidad & NFA-09, NFA-12, REI-01, REI-02 \\
\hline
\end{tabular}
\caption{Trazabilidad entre objetivos y requerimientos}
\end{table}

\section{Validación y Verificación}

Para cada requerimiento especificado se define un método de validación:

\begin{itemize}
    \item \textbf{Validación por Inspección (VI):} Revisión manual de código, logs y resultados
    \item \textbf{Validación por Prueba (VP):} Ejecución de scripts de test automatizados
    \item \textbf{Validación por Demostración (VD):} Ejecución completa del pipeline con datos reales
    \item \textbf{Validación por Análisis (VA):} Evaluación de métricas, tiempos y performance
\end{itemize}

\begin{table}[H]
\centering
\begin{tabular}{|p{2.5cm}|p{2cm}|p{8cm}|}
\hline
\textbf{Requerimiento} & \textbf{Método} & \textbf{Criterio de Aceptación} \\
\hline
RF-01 & VD & Scraping exitoso desde 3+ portales con > 300 jobs por país \\
\hline
RF-02 & VP & Deduplicación funcional, 0 duplicados en test de 1000 jobs \\
\hline
RF-03 & VI & Texto limpio sin HTML, tokens válidos, 552 palabras promedio \\
\hline
RF-04 & VP & Extracción exitosa con 2+ métodos (Regex + NER), success rate > 95\% \\
\hline
RF-04.1 & VP & Taxonomía cargada: 14,174 skills (ESCO + O*NET + Manual) \\
\hline
RF-04.2 & VA & Match rate > 10\%, Layer 1 + Layer 2 activos, confianza > 0.85 \\
\hline
RF-04.3 & VI & Skills emergentes rastreadas con frecuencia y contexto \\
\hline
RF-04.4 & VA & Embeddings generados en < 5 min (GPU), L2-normalized, 768D \\
\hline
RF-04.5 & VA & FAISS construido en < 1 min, queries > 100/seg \\
\hline
RF-05 & VD & Clustering ejecutado sin errores, clusters coherentes \\
\hline
RF-06 & VI & Visualizaciones exportadas en PDF/PNG/HTML \\
\hline
RD-01 a RD-06 & VA & Métricas de performance validadas contra umbrales \\
\hline
NFA-01 a NFA-12 & VI/VD & Tests de reproducibilidad, logs, portabilidad \\
\hline
BD-01 a BD-05 & VP & Queries SQL exitosas, integridad referencial \\
\hline
\end{tabular}
\caption{Métodos de validación por requerimiento}
\end{table}

\section{Estado de Implementación - Resumen}

\begin{table}[H]
\centering
\begin{tabular}{|p{4cm}|p{3cm}|p{6cm}|}
\hline
\textbf{Módulo/Fase} & \textbf{Estado} & \textbf{Observaciones} \\
\hline
Fase 0 - Setup Inicial & COMPLETADO & Taxonomía, embeddings, FAISS index \\
\hline
Fase 1 - Scraping & COMPLETADO & 23,352 jobs, 3 portales \\
\hline
Fase 2 - Pipeline A & IMPLEMENTADO & Regex + NER + Mapping activo \\
\hline
Fase 2 - Pipeline B & PENDIENTE & Extracción con LLMs \\
\hline
Módulo 6 - Clustering & PENDIENTE & HDBSCAN + UMAP \\
\hline
Módulo 7 - Visualización & PENDIENTE & Plotly + reportes analíticos \\
\hline
\end{tabular}
\caption{Estado de implementación por módulo}
\end{table}
