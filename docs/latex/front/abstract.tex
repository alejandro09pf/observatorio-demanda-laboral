\thispagestyle{fancy}

\begin{center}
    {\large\bfseries RESUMEN}
\end{center}

\vspace{1cm}

El mercado laboral tecnológico latinoamericano carece de sistemas automatizados para caracterizar la demanda de habilidades técnicas IT de manera sistemática y actualizada, enfrentando el desafío de capturar tecnologías emergentes que evolucionan más rápido que taxonomías oficiales. Este proyecto evaluó la viabilidad de construir un observatorio automatizado que recolectó 30,660 ofertas laborales de Colombia, México y Argentina mediante web scraping de seis portales, focalizándose en extracción de hard skills (lenguajes de programación, frameworks, herramientas cloud/DevOps, metodologías ágiles). Se implementó Pipeline A (NER + Regex) procesando el corpus completo con latencia de 0.97s por oferta, y se desarrolló Pipeline B experimental (LLM Gemma 3 4B) evaluado sobre gold standard de 300 ofertas anotadas manualmente con 6,174 hard skills. Los resultados demostraron superioridad de modelos de lenguaje para aproximar mapeo humano de competencias técnicas (F1=84.26\% vs 72.53\%), capturando habilidades implícitas inferibles del contexto y tecnologías emergentes ausentes en vocabularios controlados que métodos basados en patrones omiten. El sistema normalizó extracciones contra taxonomía ESCO v1.1.0 extendida con 152 habilidades técnicas de O*NET mediante matching de dos capas (exacto + difuso threshold 0.92), evolucionando desde versión sin bias hacia matcher con mapeos manuales curados que permitió cuantificar proporción significativa de skills modernas no representadas en estándares internacionales. El clustering no supervisado (UMAP reducción 768D$\rightarrow$2D + HDBSCAN density-based) descubrió 156 familias tecnológicas sin categorías predefinidas. Los hallazgos validan tres conclusiones: (1) viabilidad técnica de observatorios basados en scraping multi-portal; (2) superioridad de LLMs versus métodos deterministas para extracción semántica y detección de emergentes, con trade-off de costo computacional 43$\times$ mayor; (3) obsolescencia crítica de taxonomías oficiales para vocabulario IT actual, evidenciando necesidad de actualización continua.
