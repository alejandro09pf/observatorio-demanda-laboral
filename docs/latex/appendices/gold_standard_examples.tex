\section*{APÉNDICE F: EJEMPLOS DE ANOTACIONES DEL GOLD STANDARD}
\addcontentsline{toc}{section}{Apéndice F: Ejemplos de Anotaciones del Gold Standard}

Este apéndice complementa el Capítulo 7 (Resultados), específicamente la Sección 7.1 que presenta la metodología de evaluación cuantitativa de Pipeline A y Pipeline B mediante el gold standard de 300 ofertas laborales anotadas manualmente. Mientras el capítulo principal describe el proceso de construcción del gold standard con selección aleatoria estratificada por país (33\% Colombia, 33\% México, 33\% Argentina) garantizando representatividad geográfica, el protocolo de anotación estricto que requirió formato atómico sin construcciones compuestas ni paréntesis explicativos para facilitar comparación automatizada, y las métricas agregadas de evaluación mostradas en la Tabla 7.2 (Precision, Recall, F1-Score desglosados Pre-ESCO y Post-ESCO para ambos pipelines), este apéndice proporciona evidencia concreta del ground truth mediante dos ejemplos completos y representativos que ilustran la diversidad del dataset y la aplicación práctica del protocolo.

Los dos ejemplos documentados capturan características distintivas del gold standard que fundamentan su validez metodológica: el Ejemplo 1 corresponde a una oferta argentina en inglés para Developer Advocate con perfil senior que combina trabajo técnico hands-on y evangelización comunitaria, extrayendo 16 skills totales (8 hard como Python, Scrapy, REST API, Git, y 8 soft como Comunicación, Oratoria, Creación de contenido); el Ejemplo 2 corresponde a una oferta colombiana en español para IBM ACE Developer con perfil mid-level enfocada en integración de sistemas empresariales bajo metodologías ágiles, extrayendo 28 skills totales (25 hard incluyendo tecnologías especializadas como IBM App Connect Enterprise, ESQL, Java Compute Nodes, ODBC/JDBC, y 3 soft como Trabajo en equipo y Orientación a resultados). Esta diversidad geográfica (Argentina vs Colombia), idiomática (inglés vs español), de seniority (senior vs mid), de dominio técnico (web scraping vs enterprise integration), y de distribución hard/soft (50\%-50\% vs 89\%-11\%) demuestra la representatividad del gold standard para evaluar pipelines que deben operar sobre el mercado laboral tecnológico heterogéneo de América Latina. Adicionalmente, el apéndice documenta seis observaciones metodológicas que justifican decisiones del protocolo de anotación: variabilidad natural en número de skills por oferta (16 a 28 en estos ejemplos), predominancia estadística de hard skills (78.7\% promedio en 300 ofertas), términos atómicos estrictos sin construcciones compuestas, granularidad técnica específica en lugar de categorías genéricas, captura de soft skills implícitas desde descripciones de responsabilidades, y formato que garantiza consistencia para evaluación automatizada mediante comparación directa de conjuntos de términos.

\subsection*{Protocolo de Anotación}

Cada oferta laboral fue anotada siguiendo un protocolo estricto que requirió:

\begin{enumerate}
    \item \textbf{Lectura completa}: Análisis exhaustivo de título, descripción y sección de requisitos
    \item \textbf{Verificación de validez}: Confirmación de que la oferta corresponde a un rol de desarrollo de software
    \item \textbf{Formato atómico estricto}: Listado de términos individuales sin paréntesis, abreviaciones ni narrativas
    \item \textbf{Clasificación binaria}: Separación explícita entre hard skills (tecnologías, herramientas, lenguajes, metodologías) y soft skills (comunicación, liderazgo, colaboración, resolución de problemas)
    \item \textbf{Comentarios contextuales}: Nota breve identificando empresa, sector y tipo de rol para validación posterior
\end{enumerate}

El formato atómico prohibió construcciones como ``Python (pandas, numpy)'' o ``Conocimientos de AWS y Azure'', requiriendo en su lugar términos separados: ``Python'', ``Pandas'', ``NumPy'', ``AWS'', ``Azure''. Esta decisión metodológica simplificó la comparación automatizada con skills extraídas por Pipeline A y Pipeline B.

\subsection*{Ejemplo 1: Developer Advocate (Argentina, Inglés, Backend, Senior)}

\textbf{Job ID}: \texttt{44fc6c70-4887-4317-9349-80d96cb1160b}

\textbf{Título}: Developer Advocate - Remote role

\textbf{Metadata}: AR / en, Backend, Senior, 460 palabras

\textbf{Descripción resumida}: Zyte (empresa de web scraping) busca Developer Advocate para actuar como puente entre tecnología y comunidad de desarrolladores. Rol combina trabajo técnico hands-on (code samples, herramientas Zyte API y Scrapy) con evangelización comunitaria (talks, demos, workshops, contenido educativo). Responsabilidades incluyen crear contenido técnico de alta calidad, engagement en plataformas sociales/foros, hablar en eventos de industria, escribir blog posts, identificar community advocates, y proveer feedback de comunidad a equipos internos. Requisitos: 3+ años experiencia en developer advocacy/relations/technical evangelism, comprensión sólida de web technologies/APIs/web scraping/data extraction, proficiencia en Python, excelentes habilidades de comunicación (escrita, verbal, presentación), experiencia demostrada en community engagement/content creation/technical writing, experiencia con public speaking.

\textbf{Hard Skills anotadas}:
\begin{lstlisting}[basicstyle=\ttfamily\footnotesize, frame=none, numbers=none]
Python
Scrapy
Web scraping
REST API
API
Navegadores headless
Git
Documentacion tecnica
\end{lstlisting}

\textbf{Soft Skills anotadas}:
\begin{lstlisting}[basicstyle=\ttfamily\footnotesize, frame=none, numbers=none]
Comunicacion
Oratoria
Presentaciones
Creacion de contenido
Escritura tecnica
Automotivacion
Organizacion
Trabajo en equipo
\end{lstlisting}

\textbf{Comentarios}: Developer Advocate en Zyte (empresa de web scraping). Rol técnico combinado con evangelización y trabajo comunitario. Válido para gold standard.

\subsection*{Ejemplo 2: IBM ACE Developer (Colombia, Español, QA, Mid)}

\textbf{Job ID}: \texttt{4ded4226-c1fa-41f3-a75a-b804f6a01e24}

\textbf{Título}: IBM ACE Developer

\textbf{Metadata}: CO / es, QA, Mid, 460 palabras

\textbf{Descripción resumida}: Imagemaker (Colombia) busca Ingeniero IBM ACE responsable del diseño, desarrollo y despliegue de flujos de integración que conecten sistemas críticos del negocio, garantizando interoperabilidad entre aplicaciones internas y externas. Trabajo bajo metodologías ágiles junto con equipos multidisciplinarios de desarrollo y operaciones, asegurando calidad, escalabilidad y seguridad de servicios implementados. Rol requiere sólida base técnica en IBM App Connect Enterprise (ACE) y experiencia práctica en integración de datos, automatización de procesos y creación de APIs REST. Responsabilidades clave: diseñar/desarrollar/mantener flujos de integración en IBM ACE, configurar conexiones a bases de datos (ODBC.ini, mqsisetdbparms), programar nodos compute usando ESQL o Java Compute Nodes, implementar/consumir servicios REST y APIs, ejecutar consultas SQL y pruebas funcionales con Postman, configurar certificados/llaves de seguridad (JKS), documentar desarrollos técnicos/funcionales, colaborar con equipos bajo enfoque ágil. Skills requeridas: experiencia comprobada IBM ACE v12/v11, dominio ESQL/Java, metodologías ágiles (Scrum/Kanban), configuración ODBC/JDBC, creación/consumo APIs REST, herramientas de prueba (Postman), despliegue en productivo, documentación técnica. Plus: Kafka, integraciones seguras (JKS, SSL), GitHub/Jenkins/CI-CD, cloud híbrido (AWS, Azure), proyectos retail/banca.

\textbf{Hard Skills anotadas}:
\begin{lstlisting}[basicstyle=\ttfamily\footnotesize, frame=none, numbers=none]
IBM App Connect Enterprise
IBM ACE
ESQL
Java
Java Compute Nodes
REST API
ODBC
JDBC
SQL
Postman
JKS
Certificados SSL
Scrum
Kanban
Kafka
GitHub
Jenkins
CI/CD
AWS
Azure
Cloud hibrido
Integracion de sistemas
Automatizacion de procesos
DevOps
Documentacion tecnica
\end{lstlisting}

\textbf{Soft Skills anotadas}:
\begin{lstlisting}[basicstyle=\ttfamily\footnotesize, frame=none, numbers=none]
Trabajo en equipo
Colaboracion
Orientacion a resultados
\end{lstlisting}

\textbf{Comentarios}: IBM ACE Developer en Imagemaker (Colombia). Rol de integración de sistemas con IBM App Connect Enterprise. Válido.

\subsection*{Observaciones Metodológicas}

Los dos ejemplos ilustran características clave del protocolo de anotación:

\begin{itemize}
    \item \textbf{Variabilidad en número de skills}: Job \#1 (16 skills totales: 8 hard + 8 soft), Job \#46 (28 skills: 25 hard + 3 soft), reflejando que el número de skills anotadas depende de la especificidad y detalle de cada oferta, no de un target predefinido.
    \item \textbf{Predominancia de hard skills}: Promedio 78.7\% hard skills vs 21.3\% soft skills en las 300 ofertas, consistente con el enfoque técnico de roles de desarrollo de software.
    \item \textbf{Términos atómicos estrictos}: Ausencia de construcciones compuestas (``Python/Django'', ``AWS o Azure'') o paréntesis explicativos, garantizando comparabilidad directa con outputs de pipelines.
    \item \textbf{Granularidad técnica}: Inclusión de términos específicos como ``Java Compute Nodes'', ``ODBC'', ``Navegadores headless'' en lugar de categorías genéricas como ``Desarrollo backend'' o ``Herramientas de integración''.
    \item \textbf{Captura de soft skills contextuales}: Identificación de soft skills implícitas en descripción de responsabilidades (``colaborar con equipos multidisciplinarios'' se extrae como ``Colaboración'', ``deliver talks and demos'' se extrae como ``Oratoria'').
    \item \textbf{Diversidad geográfica e idiomática}: Job \#1 (Argentina, inglés, sector fintech/web scraping) vs Job \#46 (Colombia, español, sector enterprise integration), ilustrando la representatividad del dataset.
\end{itemize}

El formato garantizó consistencia a lo largo de las 300 anotaciones, facilitando la evaluación automatizada de Pipeline A (NER+Regex) y Pipeline B (LLM) mediante comparación directa de conjuntos de términos.
