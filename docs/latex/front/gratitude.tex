\thispagestyle{fancy}

\begin{center}
    {\large\bfseries AGRADECIMIENTOS}
\end{center}

\vspace{1cm}

\noindent\textbf{Nicolás Francisco Camacho Alarcón}

\vspace{0.5cm}

Llegar hasta aquí ha sido un camino largo y sinuoso, lleno de dudas, decisiones difíciles y momentos en los que pensé que no podría seguir adelante. Tras un camino de búsqueda que me llevó por tres carreras diferentes y siete años de aprendizaje constante, hoy puedo decir que cada paso, cada tropiezo y cada nueva dirección me trajeron exactamente donde debía estar.

En primer lugar, quiero agradecer a mis padres, quienes desde pequeño me impulsaron a crecer académicamente, a alcanzar nuevas metas y a nunca conformarme. Gracias por apoyarme en cada cambio de rumbo, por entender mis búsquedas y por nunca dejar de creer en mí, incluso cuando yo mismo dudaba. Su amor y esfuerzo constante han sido el pilar que me sostuvo en los momentos más difíciles. A mi hermana, mi apoyo incondicional y prácticamente mi mejor amiga, gracias por estar siempre ahí, por escucharme, por animarme y por ser esa compañía invaluable en cada etapa de este proceso. A mi abuela, por su cariño y por ser parte fundamental de esta familia que me ha dado todo. A toda mi familia, gracias por comprenderme, por acompañarme en cada gran decisión, por ayudarme a levantarme cuando sentí que no podía más, y por estar presentes en cada momento importante de mi vida.

Un agradecimiento muy especial a Lucy, mi perrita, quien con su compañía silenciosa y su alegría incondicional trajo luz a cada día, sin importar lo difícil que fuera. Su presencia fue un refugio de paz y felicidad en medio de las noches largas y los momentos de mayor estrés.

A todos los amigos que he ido conociendo semestre a semestre desde que entré a primer semestre de música, pasando por electrónica y llegando a sistemas: gracias por hacer de estos años una experiencia llena de aprendizajes, risas y compañía. Cada uno de ustedes aportó algo valioso a mi camino y me ayudó a convertirme en quien soy hoy.

A todos mis profesores, gracias por inspirarme semestre tras semestre a ser mejor, a alcanzar nuevos niveles y a recordarme constantemente que amo aprender y que me apasiona sentirme retado. Su dedicación y enseñanzas fueron fundamentales para mantener viva esa chispa de curiosidad que me impulsa.

A Alejandro, mi compañero de tesis, gracias por haberme acompañado este año y por haber sido el apoyo que necesité en el tramo final para lograrlo. Tu paciencia y tu apoyo emocional fueron invaluables, y estoy profundamente agradecido por haber compartido este proceso contigo.

A Gabriel, nuestro director de tesis, por inspirarnos y ayudarnos a encontrar algo realmente retador e innovador que nos permitiera hacer algo nuevo en la industria. Tu guía fue clave para que este proyecto tomara forma y alcanzara su verdadero potencial.

A Sezzle, mi empresa, gracias por permitirme terminar mi carrera al tiempo que trabajo, y por apoyarme para salir adelante. Su comprensión y flexibilidad fueron fundamentales para lograr este objetivo.

Finalmente, quiero agradecer a ese Nicolás que decidió levantarse una vez más, por haber tenido la valentía de buscar su verdadero camino y por haber confiado en que encontraría su lugar, incluso cuando el camino no estaba claro. Este logro también es tuyo.

A todos los que han sido parte de este camino: gracias. Este trabajo es el resultado no solo de mi esfuerzo, sino del amor, la paciencia y el acompañamiento de cada uno de ustedes.

\vspace{1cm}

\noindent\textbf{Alejandro Pinzón Fajardo}

\vspace{0.5cm}

La culminación de este trabajo representa mucho más que un logro académico; es el resultado de años de esfuerzo, constancia y del apoyo invaluable de las personas que me han acompañado a lo largo de este camino.

Mi familia ha sido el pilar fundamental de este proceso. Su amor, paciencia y fe inquebrantable me brindaron la fortaleza necesaria para avanzar incluso en los momentos más difíciles. Gracias por enseñarme el valor del trabajo honesto, la disciplina y la perseverancia que hoy hacen posible este resultado.

Durante mi formación en la Pontificia Universidad Javeriana, tuve el privilegio de aprender de profesores que, más allá de impartir conocimiento, despertaron en mí la curiosidad, el pensamiento crítico y la pasión por la ingeniería. Su dedicación y compromiso fueron esenciales para mi crecimiento profesional y personal.

El desarrollo de esta tesis no habría sido posible sin la guía del profesor Gabriel, cuya orientación y criterio fueron determinantes para dar forma y sentido al proyecto. Su apoyo y su exigencia académica nos permitieron alcanzar un resultado del que me siento profundamente orgulloso.

Finalmente, compartir este proceso con Nicolás fue una experiencia gratificante. Su compromiso, su disposición al trabajo en equipo y su acompañamiento hicieron de esta etapa un recorrido más llevadero y enriquecedor, en el que aprendimos tanto del proyecto como de nosotros mismos.

A todos ellos, mi gratitud más sincera.
