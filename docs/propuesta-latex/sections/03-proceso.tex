\chapter{Proceso}

Este proyecto de grado adopta una estrategia metodológica mixta inspirada en el modelo CRISP-DM y en prácticas ágiles derivadas de Scrum. En lugar de aplicar estos marcos de manera estricta, se toma de CRISP-DM la lógica iterativa por fases encadenadas, desde la comprensión del dominio hasta la validación y documentación, mientras que de Scrum se retoman prácticas de revisión incremental y planificación flexible por semanas de trabajo. El enfoque favorece un desarrollo modular, adaptable y basado en resultados verificables, articulando técnicas de scraping, procesamiento de lenguaje natural, embeddings semánticos, clustering y validación de resultados.

\section{Fase metodológica 1 - Diseño y arquitectura}

\subsection{Método}

Esta fase se basa en el paso de comprensión del negocio de CRISP-DM y en el inicio de un primer sprint de planificación según Scrum. Se enfoca en el análisis extensivo del estado del arte y en la definición conceptual y técnica del sistema a construir. Se realiza una revisión crítica de estudios previos, tecnologías disponibles y taxonomías laborales existentes (como ESCO o CIUO) para seleccionar las herramientas y flujos más adecuados para el contexto latinoamericano (Colombia, México y Argentina). Paralelamente, se inician los primeros módulos de documentación estructurada (guía metodológica, esquema de arquitectura) que serán alimentados en cada fase.

\subsection{Actividades}

\begin{itemize}
    \item Revisión crítica y sistemática del estado del arte (papers, benchmarks, pipelines existentes).
    \item Definición de módulos y flujos de datos (scraping, NLP, embeddings, clustering, validación).
    \item Selección preliminar de tecnologías, modelos pre entrenados y fuentes de datos.
    \item Diseño del pipeline general y planificación por iteraciones.
    \item Inicio de la documentación UML y del repositorio metodológico.
\end{itemize}

\subsection{Resultados esperados}

\begin{itemize}
    \item Diagrama del sistema propuesto (pipeline modular).
    \item Lista de portales de empleo seleccionados y características técnicas de acceso.
    \item Decisión justificada de tecnologías y taxonomías a utilizar.
    \item Documento metodológico inicial con justificación de diseño y planificación preliminar.
\end{itemize}

\section{Fase metodológica 2 - Extracción de ofertas laborales}

\subsection{Método}

Inspirado en la fase de recolección de datos de CRISP-DM, esta fase implementa un sistema de scraping automatizado de portales de empleo relevantes en español (como elempleo.com, Bumeran, Computrabajo). Se aprovechan herramientas como Scrapy y Selenium para capturar datos dinámicos de forma robusta. Desde Scrum se aplica una lógica de entrega continua, con iteraciones de prueba para validar la estabilidad de los spiders y ajustar el crawler por portal.

\subsection{Actividades}

\begin{itemize}
    \item Implementación de spiders con Scrapy + fallback con Selenium para contenido dinámico.
    \item Extracción de datos clave: título, descripción, ubicación, modalidad, requisitos.
    \item Normalización básica de campos y almacenamiento en base de datos estructurada.
    \item Validación de calidad del scraping (frecuencia de actualización, duplicados, errores).
    \item Registro continuo de avances en la documentación metodológica.
\end{itemize}

\subsection{Resultados esperados}

\begin{itemize}
    \item Base de datos actualizada con vacantes recolectadas de los tres países objetivo.
    \item Spiders funcionales y adaptables a cambios de formato por portal.
    \item Informe de scraping con cobertura, precisión y errores detectados.
    \item Revisión de licitud y ética del scraping conforme a estándares locales.
\end{itemize}

\section{Fase metodológica 3 - Procesamiento y análisis semántico}

\subsection{Método}

Corresponde a la fase de preparación de los datos y modelado en CRISP-DM. En esta etapa se realizan tareas de extracción de habilidades explícitas e implícitas usando un enfoque híbrido: se combinan técnicas clásicas de NER adaptado al dominio laboral en español con razonamiento semántico vía LLMs en modo few-shot. La fase incorpora un proceso de normalización de términos basado en taxonomías estandarizadas (como ESCO), aplicadas en el idioma original sin traducción previa. A través de actividades similares a los sprints de Scrum, se permite iterar sobre los modelos y prompts hasta alcanzar resultados robustos.

\subsection{Actividades}

\begin{itemize}
    \item Aplicación de NER entrenado o adaptado al dominio laboral en español.
    \item Curación y ajuste de regex específicas para secciones clave (requisitos, funciones).
    \item Implementación de prompts few-shot en LLMs (GPT, LLaMA, Claude) para detección implícita.
    \item Validación y filtrado de habilidades mediante taxonomías laborales estandarizadas.
    \item Documentación de los prompts, razonamientos y mejoras obtenidas.
\end{itemize}

\subsection{Resultados esperados}

\begin{itemize}
    \item Corpus anotado automáticamente con habilidades por vacante.
    \item Registro de habilidades explícitas (vía NER) e implícitas (vía LLM).
    \item Vocabulario laboral multilingüe alineado a taxonomías como ESCO o CIUO.
    \item Informe de cobertura y rendimiento de los métodos de extracción utilizados.
\end{itemize}

\section{Fase metodológica 4 - Segmentación de perfiles laborales}

\subsection{Método}

Inspirado en la fase de modelado y análisis de CRISP-DM, esta etapa consiste en representar semánticamente las habilidades detectadas y agruparlas en perfiles o clústeres funcionales. Se utilizan embeddings multilingües (como BETO, LaBSE o E5), reducción de dimensionalidad con UMAP y clustering con HDBSCAN. No se realizan visualizaciones aún; el objetivo es estructurar internamente los grupos y patrones emergentes para su posterior evaluación.

\subsection{Actividades}

\begin{itemize}
    \item Vectorización de habilidades mediante modelos multilingües preentrenados.
    \item Reducción de dimensionalidad con UMAP para preservar relaciones semánticas.
    \item Aplicación de HDBSCAN para identificación de grupos latentes de perfiles laborales.
    \item Revisión y depuración de clusters generados (eliminación de ruido, interpretación manual inicial).
    \item Actualización de la documentación con resultados intermedios.
\end{itemize}

\subsection{Resultados esperados}

\begin{itemize}
    \item Representación vectorial de habilidades por vacante.
    \item Clústeres de perfiles laborales con características técnicas y semánticas diferenciadas.
    \item Informe técnico sobre la estabilidad, coherencia e interpretabilidad de los agrupamientos.
\end{itemize}

\section{Fase metodológica 5 - Validación técnica y visualización macro}

\subsection{Método}

Equivalente a la fase de evaluación de CRISP-DM, esta etapa se enfoca en validar la calidad de las salidas del sistema: tanto a nivel de precisión del reconocimiento de habilidades como de coherencia en los agrupamientos obtenidos. Se aplican métricas cuantitativas (precisión, recall, silhouette score) y validaciones cualitativas. Adicionalmente, se generan visualizaciones macro que permiten evaluar tendencias emergentes, sin llegar a construir dashboards interactivos.

\subsection{Actividades}

\begin{itemize}
    \item Evaluación de los resultados de extracción con muestras revisadas manualmente.
    \item Cálculo de métricas para los clusters (densidad, separación, coherencia).
    \item Generación de visualizaciones macro estáticas (por frecuencia, temporalidad, distribución regional).
    \item Revisión crítica con usuarios técnicos (Profesores, revisores) para retroalimentación.
    \item Registro de hallazgos, mejoras y limitaciones detectadas.
\end{itemize}

\subsection{Resultados esperados}

\begin{itemize}
    \item Reporte técnico de validación con métricas y gráficos interpretables.
    \item Visualizaciones macro de perfiles, habilidades, y patrones relevantes.
    \item Justificación del valor informativo del sistema construido.
\end{itemize}

\section{Fase metodológica 6 - Documentación y guía metodológica}

\subsection{Método}

Inspirada en la fase final de despliegue de CRISP-DM, esta etapa compila todo el conocimiento generado en las fases anteriores y estructura una guía metodológica que permita replicar, adaptar o escalar el sistema en otros contextos. Se produce documentación técnica, análisis de replicabilidad, y propuestas de mejora futura.

\subsection{Actividades}

\begin{itemize}
    \item Consolidación de resultados técnicos por fase.
    \item Estructuración de la guía metodológica completa del observatorio.
    \item Elaboración de anexos: scripts, configuraciones, prompts, logs de scraping, visualizaciones.
    \item Preparación del entregable final para sustentación.
    \item Revisión general con enfoque en claridad, replicabilidad y escalabilidad.
\end{itemize}

\subsection{Resultados esperados}

\begin{itemize}
    \item Documento técnico y guía metodológica del sistema desarrollado.
    \item Repositorio estructurado con código y recursos.
    \item Validación interna del sistema como solución replicable en Latinoamérica.
\end{itemize}
