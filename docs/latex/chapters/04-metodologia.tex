\chapter{METODOLOGÍA}

El proyecto adoptó un enfoque metodológico basado en la integración entre la metodología CRISP-DM y un esquema de desarrollo modular iterativo. Esta combinación permitió gestionar simultáneamente la complejidad analítica del proceso de minería de datos y las necesidades prácticas de construir un sistema compuesto por módulos independientes. CRISP-DM proporcionó la estructura conceptual que dio orden al ciclo de vida del proyecto, mientras que la modularidad garantizó que cada componente pudiera desarrollarse, validarse y refinarse de manera aislada antes de incorporarse al pipeline completo. El resultado fue un proceso sistemático capaz de equilibrar diseño metodológico y evolución técnica.

La aplicación del enfoque comenzó con una etapa de análisis en la que se clarificó el problema que motiva el observatorio: la falta de herramientas automatizadas que permitan capturar habilidades técnicas emergentes en el mercado laboral latinoamericano. A través de revisión documental, análisis de vacantes y evaluación del estado del arte, se estableció un conjunto de objetivos medibles y criterios de éxito que guiaron el desarrollo. Este entendimiento inicial fue determinante para definir el alcance, las métricas y las decisiones metodológicas posteriores.

Con los objetivos concretados, se avanzó hacia la caracterización de las fuentes de datos disponibles. Este proceso permitió identificar la estructura y variabilidad de los portales de empleo utilizados, así como la presencia de retos recurrentes relacionados con la heterogeneidad del formato, la mezcla de español e inglés técnico y la ausencia de campos estandarizados. A partir de este análisis se definió la necesidad de construir un conjunto de referencia manualmente anotado, así como la pertinencia de emplear ESCO como taxonomía base para la normalización de habilidades. Estas decisiones metodológicas sentaron las bases para garantizar evaluaciones objetivas en las etapas posteriores.

La preparación de datos se abordó mediante el desarrollo iterativo de los módulos de scraping, limpieza, normalización y deduplicación. Cada módulo atravesó ciclos de ajustes basados en inspección manual, pruebas controladas y verificación de coherencia, lo que permitió corregir errores, fortalecer la robustez del procesamiento y asegurar que los datos resultantes cumplieran criterios de calidad adecuados para el modelado. Paralelamente, se construyó el conjunto de 300 vacantes anotadas que funcionó como gold standard para la evaluación formal de los modelos.

El modelado se desarrolló siguiendo una lógica igualmente iterativa. Se diseñaron dos pipelines complementarios: uno basado en técnicas tradicionales expresiones regulares, reglas lingüísticas y reconocimiento de entidades y otro sustentado en modelos de lenguaje y mapeo semántico hacia ESCO. Ambos pipelines evolucionaron mediante ciclos constantes de experimentación y análisis de errores, donde cada ajuste se validó respecto al gold standard. Este proceso permitió mejorar de manera progresiva la precisión, cobertura y coherencia en la extracción de habilidades.

La evaluación del sistema integró métricas cuantitativas, verificación cualitativa y análisis de robustez operativa. Los resultados obtenidos revelaron fortalezas y debilidades específicas, lo que habilitó nuevos ciclos de refinamiento en los modelos y los componentes del pipeline. Esta interacción continua entre modelado y evaluación consolidó un proceso de mejora incremental que fortaleció la calidad final del sistema.

Finalmente, el proyecto alcanzó su fase operativa mediante la integración de todos los módulos en una herramienta ejecutable con programación automática, interfaz de línea de comandos y documentación técnica completa. Esta etapa permitió validar el funcionamiento del observatorio en un entorno real, asegurando estabilidad, trazabilidad y capacidad de ejecución recurrente sin supervisión.

\section{Diagrama de Flujo Metodológico}

El flujo metodológico completo del proyecto integró las seis fases de CRISP-DM con las siete etapas del pipeline de software mediante ciclos iterativos de refinamiento. Cada fase de CRISP-DM se tradujo en actividades concretas de desarrollo de software, estableciendo un proceso sistemático que equilibró diseño metodológico y evolución técnica. Los resultados de la fase de Evaluación retroalimentaron continuamente la fase de Modelado, generando versiones mejoradas de los pipelines de extracción mediante ajuste de parámetros, refinamiento de reglas lingüísticas y optimización de prompts para LLMs. La representación visual completa del flujo metodológico, incluyendo las interconexiones entre fases, las iteraciones de mejora continua y los puntos de validación experimental, se presenta detalladamente en el Apéndice D (Figura \ref{fig:flujo-metodologico-anexo}).
