\chapter{Análisis de impacto}

\section*{Impacto a corto plazo}

\begin{itemize}
    \item \textbf{Disponibilidad de una arquitectura técnica replicable}: El proyecto entregará un pipeline modular, escalable y validado para el análisis de demanda de habilidades en español, accesible a universidades, gobiernos o centros de datos laborales.

    \item \textbf{Innovación metodológica en el uso de LLMs en español}: Se aportarán resultados empíricos sobre el uso de modelos como GPT o T5 en tareas de enriquecimiento de datos, corrección de errores de NER y detección de habilidades implícitas, abriendo camino a nuevas líneas de investigación.

    \item \textbf{Reducción de la dependencia conceptual de reportes estáticos}: Se demostrará que es posible construir análisis dinámicos de demanda con scraping + NLP + clustering, generando una alternativa a métodos tradicionales como encuestas o informes anuales.
\end{itemize}

\section*{Impacto a mediano plazo}

\begin{itemize}
    \item \textbf{Toma de decisiones informada por terceros}: Actores del ecosistema educativo y laboral podrán reutilizar el sistema propuesto para ajustar programas académicos, estrategias de formación o diagnósticos institucionales, con base en datos reales del mercado.

    \item \textbf{Adaptación de currículos académicos basada en evidencia}: Instituciones educativas podrán incorporar resultados derivados del pipeline o usar su versión adaptada para identificar lagunas formativas y tendencias emergentes en IA, Ciencia de Datos y tecnología.

    \item \textbf{Mejor conexión entre oferta y demanda laboral}: Al proporcionar una estructura replicable de análisis, se reduce la brecha entre lo que enseñan las instituciones y lo que requiere el mercado, permitiendo acciones más precisas para disminuir el desempleo técnico.

    \item \textbf{Fortalecimiento del talento regional}: Profesionales de Colombia, México y Argentina contarán con un marco claro, si el sistema es implementado por terceros, sobre las habilidades más demandadas, fortaleciendo su empleabilidad y adaptabilidad.
\end{itemize}

\section*{Impacto a largo plazo}

\begin{itemize}
    \item \textbf{Transformación estructural en políticas educativas y laborales}: Gobiernos, observatorios nacionales y organismos multilaterales podrían utilizar el sistema como insumo para estrategias públicas de formación, reconversión laboral o fomento a la empleabilidad tecnológica.

    \item \textbf{Impulso a una digitalización regional más armónica}: Al permitir análisis comparables entre países hispanohablantes, el proyecto puede contribuir al diseño de estrategias de transformación digital más sensibles a la evolución real del mercado.

    \item \textbf{Reducción de desigualdades estructurales en el acceso a oportunidades laborales}: El enfoque abierto, modular y multilingüe facilita que actores sin grandes recursos técnicos, como regiones no capitalinas o instituciones pequeñas, puedan beneficiarse de los hallazgos y construir sus propios análisis.
\end{itemize}
