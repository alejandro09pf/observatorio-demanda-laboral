% ============================================================================
% PONTIFICIA UNIVERSIDAD JAVERIANA
% FACULTAD DE INGENIERIA - SISTEMAS
% Política de Tratamiento de Datos Personales
% Observatorio de Demanda Laboral en América Latina
% ============================================================================

\documentclass[11pt,oneside,letterpaper]{article}

% ============================================================================
% PAQUETES
% ============================================================================
\usepackage[utf8]{inputenc}
\usepackage[spanish,es-tabla]{babel}
\usepackage[letterpaper,top=3cm,bottom=3cm,left=3cm,right=3cm]{geometry}
\usepackage{times}
\usepackage{graphicx}
\usepackage{setspace}
\usepackage{fancyhdr}
\usepackage{titlesec}
\usepackage[hidelinks]{hyperref}

% ============================================================================
% CONFIGURACIÓN DE INTERLINEADO
% ============================================================================
\onehalfspacing

% ============================================================================
% CONFIGURACIÓN DE ENCABEZADOS Y PIE DE PÁGINA
% ============================================================================
\pagestyle{fancy}
\fancyhf{}
\fancyhead[L]{Observatorio de Demanda Laboral}
\fancyhead[R]{Política de Tratamiento de Datos}
\fancyfoot[R]{Página \thepage}
\renewcommand{\headrulewidth}{0.4pt}
\renewcommand{\footrulewidth}{0.4pt}

% Estilo para primera página
\fancypagestyle{plain}{%
  \fancyhf{}%
  \fancyfoot[R]{Página \thepage}
  \renewcommand{\headrulewidth}{0pt}
  \renewcommand{\footrulewidth}{0.4pt}
}

% ============================================================================
% CONFIGURACIÓN DE TÍTULOS
% ============================================================================
\titleformat{\section}
{\normalfont\Large\bfseries}{\thesection}{1em}{}

\titleformat{\subsection}
{\normalfont\large\bfseries}{\thesubsection}{1em}{}

\titleformat{\subsubsection}
{\normalfont\normalsize\bfseries}{\thesubsubsection}{1em}{}

% ============================================================================
% DOCUMENTO
% ============================================================================
\begin{document}

% ============================================================================
% PORTADA
% ============================================================================
\begin{titlepage}
    \centering
    \vspace*{2cm}

    {\LARGE\bfseries PONTIFICIA UNIVERSIDAD JAVERIANA\par}

    \vspace{0.5cm}

    {\large FACULTAD DE INGENIERÍA\par}
    {\large CARRERA DE INGENIERÍA DE SISTEMAS\par}

    \vspace{2.5cm}

    {\Large\itshape Observatorio de demanda laboral en América Latina\par}

    \vspace{2cm}

    {\huge\bfseries Política de Tratamiento de\par}
    {\huge\bfseries Datos Personales\par}

    \vspace{2cm}

    {\Large Octubre 2025\par}

    \vspace{0.5cm}

    {\large Versión 1.0\par}

    \vspace{1.5cm}

    {\large Alejandro Pinzón Fajardo\par}
    \vspace{0.3cm}
    {\large Nicolás Francisco Camacho Alarcón\par}

    \vfill

    {\large Proyecto de Grado\par}

    \vspace{0.5cm}

    {\large BOGOTÁ, D.C.\par}

\end{titlepage}

% ============================================================================
% TABLA DE CONTENIDOS
% ============================================================================
\tableofcontents
\newpage

% ============================================================================
% CONTENIDO
% ============================================================================

\section{INTRODUCCIÓN}

\subsection{Presentación}

El Observatorio de Demanda Laboral en América Latina es un sistema automatizado diseñado para el monitoreo y análisis de la demanda de habilidades técnicas en los mercados laborales de Colombia, México y Argentina. A través de tecnologías de web scraping, procesamiento de lenguaje natural (NLP) e inteligencia artificial, el sistema recopila ofertas de empleo públicas de diversos portales especializados, extrae habilidades técnicas utilizando la taxonomía ESCO (European Skills, Competences, Qualifications and Occupations), y genera análisis de tendencias del mercado laboral.

Dado que el sistema recopila y procesa información de ofertas de empleo que puede contener datos personales, se establece la presente Política de Tratamiento de Datos Personales con el fin de garantizar la protección, seguridad y uso adecuado de la información, en cumplimiento de la normativa colombiana vigente en materia de protección de datos personales.

\subsection{Objetivo de la Política}

La presente política tiene como objetivo establecer los principios, procedimientos y derechos de los titulares de los datos personales que puedan ser recolectados y tratados por el Observatorio de Demanda Laboral. En ella se describen las medidas adoptadas para garantizar la seguridad de la información y el respeto a la privacidad de todas las personas cuyos datos puedan ser procesados.

\subsection{Ámbito de Aplicación}

Esta política se aplica a:

\begin{itemize}
    \item Todos los datos recopilados mediante técnicas de web scraping de portales de empleo públicos en Colombia, México y Argentina.
    \item Información contenida en ofertas de empleo, incluyendo descripciones de puestos, requisitos y otra información publicada por empleadores.
    \item Habilidades técnicas extraídas y analizadas mediante algoritmos de NLP y machine learning.
    \item Datos almacenados en la base de datos PostgreSQL del sistema.
    \item Desarrolladores, investigadores y terceros que, en el desarrollo de sus actividades académicas o de investigación, puedan acceder a información contenida en las bases de datos del Observatorio.
\end{itemize}

\subsection{Responsable}

El responsable del tratamiento de los datos personales recolectados a través del Observatorio de Demanda Laboral es el equipo de desarrollo del proyecto, un grupo de trabajo académico de la Pontificia Universidad Javeriana.

\begin{itemize}
    \item \textbf{Nombre del responsable:} Proyecto Observatorio de Demanda Laboral en América Latina
    \item \textbf{Naturaleza:} Proyecto académico de investigación
    \item \textbf{Institución:} Pontificia Universidad Javeriana
    \item \textbf{Domicilio:} Bogotá D.C., Colombia
    \item \textbf{Correos electrónicos de contacto:} alejandro\_pinzon@javeriana.edu.co o camachoa.nicolas@javeriana.edu.co
\end{itemize}

El equipo responsable se compromete a garantizar la protección, confidencialidad y correcto tratamiento de los datos personales recolectados, conforme a lo establecido en la Ley 1581 de 2012 y demás normas concordantes sobre protección de datos personales en Colombia.

\newpage

\section{MARCO LEGAL}

\subsection{Legislación Aplicable}

El tratamiento de los datos personales en el Observatorio de Demanda Laboral se rige por las siguientes normas de la República de Colombia:

\begin{itemize}
    \item \textbf{Ley 1581 de 2012:} Régimen General de Protección de Datos Personales, que establece principios y derechos de los titulares de datos.
    \item \textbf{Decreto 1377 de 2013:} Reglamentación parcial de la Ley 1581 de 2012, en lo relacionado con las políticas de tratamiento de datos y la autorización de los titulares.
    \item \textbf{Constitución Política de Colombia (Artículos 15 y 20):} Protección del derecho a la intimidad, habeas data y acceso a la información.
    \item \textbf{Reglamentos y circulares de la Superintendencia de Industria y Comercio (SIC):} Ente encargado de supervisar el cumplimiento de la normativa de protección de datos en Colombia.
\end{itemize}

\subsection{Principios Rectores del Tratamiento de Datos}

De acuerdo con la legislación vigente, el tratamiento de datos personales en el Observatorio de Demanda Laboral se basa en los siguientes principios:

\begin{itemize}
    \item \textbf{Legalidad:} El tratamiento de datos es una actividad regulada que debe cumplir con la normativa aplicable.
    \item \textbf{Finalidad:} Los datos se recolectan con propósitos legítimos de investigación académica y análisis del mercado laboral, previamente definidos.
    \item \textbf{Libertad:} Se respeta el principio de que la recolección de datos debe realizarse de manera transparente y con fines legítimos.
    \item \textbf{Veracidad o Calidad:} La información tratada proviene de fuentes públicas y se procesa sin alteración de su contenido original.
    \item \textbf{Transparencia:} Los interesados tienen derecho a conocer el uso que se le está dando a los datos procesados por el sistema.
    \item \textbf{Seguridad:} El Observatorio adopta medidas de protección técnicas y administrativas para evitar el acceso no autorizado, pérdida o uso indebido de los datos.
    \item \textbf{Confidencialidad:} Los datos personales no serán divulgados de manera que permitan la identificación de personas específicas, salvo autorización expresa o cuando exista una orden legal que lo justifique.
\end{itemize}

\newpage

\section{RESPONSABLE DEL TRATAMIENTO Y TIPOS DE DATOS TRATADOS}

\subsection{Responsable del Tratamiento}

El Observatorio de Demanda Laboral, como responsable del tratamiento de datos personales, se compromete a garantizar la confidencialidad, integridad y seguridad de la información procesada, en cumplimiento de la normativa vigente y con fines exclusivamente académicos y de investigación.

\subsection{Tipos de Datos Tratados}

El Observatorio recopila y trata los siguientes tipos de datos:

\subsubsection{Datos Recopilados de Ofertas de Empleo Públicas}

\begin{itemize}
    \item Título del puesto de trabajo
    \item Descripción de funciones y responsabilidades
    \item Requisitos técnicos y profesionales
    \item Información de la empresa empleadora (cuando es pública)
    \item Ubicación geográfica del empleo (país, ciudad)
    \item Portal de origen de la oferta laboral
    \item Fecha de publicación
\end{itemize}

\subsubsection{Datos Derivados del Procesamiento}

\begin{itemize}
    \item Habilidades técnicas extraídas mediante NLP y análisis de texto
    \item Mapeo de habilidades a la taxonomía ESCO
    \item Scores de confianza de las extracciones realizadas
    \item Vectores de embeddings (representaciones numéricas de habilidades)
    \item Resultados de clustering y análisis dimensional
    \item Tendencias agregadas del mercado laboral
\end{itemize}

\subsubsection{Datos Sensibles}

El Observatorio está diseñado para procesar información pública de ofertas de empleo y \textbf{no recopila intencionalmente datos sensibles} tales como:

\begin{itemize}
    \item Datos biométricos
    \item Información de salud
    \item Orientación sexual
    \item Afiliación política o sindical
    \item Datos financieros personales
\end{itemize}

En caso de que, por la naturaleza pública de las ofertas de empleo, algún dato sensible sea inadvertidamente recopilado, el sistema implementa mecanismos de anonimización y agregación que impiden la identificación de personas específicas.

\newpage

\section{FINALIDADES DEL TRATAMIENTO DE DATOS Y DERECHOS DE LOS TITULARES}

\subsection{Finalidades del Tratamiento de Datos}

Los datos personales recolectados por el Observatorio de Demanda Laboral serán utilizados exclusivamente para las siguientes finalidades:

\begin{itemize}
    \item \textbf{Análisis del mercado laboral:} Identificar tendencias de demanda de habilidades técnicas en Colombia, México y Argentina.
    \item \textbf{Investigación académica:} Generar conocimiento sobre las necesidades del mercado laboral latinoamericano para fines educativos y de investigación.
    \item \textbf{Extracción de habilidades:} Aplicar técnicas de NLP para identificar y clasificar competencias técnicas requeridas por los empleadores.
    \item \textbf{Generación de reportes agregados:} Crear visualizaciones, estadísticas y reportes sobre tendencias del mercado, garantizando la no identificación de personas o empresas específicas.
    \item \textbf{Mejora del sistema:} Optimizar los algoritmos de extracción, clasificación y análisis mediante técnicas de machine learning.
    \item \textbf{Cumplimiento legal:} Atender requerimientos de entidades reguladoras y cumplir con normativas de protección de datos.
\end{itemize}

\subsection{Derechos de los Titulares}

De acuerdo con la normativa vigente, las personas cuyos datos puedan estar siendo procesados tienen los siguientes derechos:

\begin{itemize}
    \item \textbf{Acceder:} Conocer de forma gratuita los datos personales que estén siendo tratados y el procesamiento aplicado.
    \item \textbf{Actualizar y Rectificar:} Solicitar la corrección de datos parciales, inexactos, incompletos o que induzcan a error.
    \item \textbf{Revocar la Autorización:} Retirar el consentimiento para el tratamiento de datos cuando aplique, y/o solicitar la supresión del dato cuando en el tratamiento no se respeten los principios, derechos y garantías constitucionales y legales.
    \item \textbf{Eliminar sus Datos:} Solicitar la supresión de información personal cuando no sea necesaria para las finalidades establecidas.
    \item \textbf{Solicitar prueba de la autorización:} Cuando aplique, solicitar evidencia del consentimiento otorgado para el tratamiento de datos personales.
    \item \textbf{Ser informado:} Previa solicitud, conocer el uso que se le ha dado a los datos personales.
    \item \textbf{Presentar quejas:} Presentar ante la Superintendencia de Industria y Comercio quejas por infracciones a lo dispuesto en la Ley 1581 de 2012 y las demás normas concordantes.
\end{itemize}

\newpage

\section{MEDIDAS DE SEGURIDAD Y TRANSFERENCIA DE DATOS}

\subsection{Medidas de Seguridad}

El Observatorio de Demanda Laboral implementa diversas medidas de seguridad técnicas y administrativas para proteger los datos procesados:

\begin{itemize}
    \item \textbf{Cifrado de datos:} La información almacenada en la base de datos PostgreSQL está protegida mediante cifrado en tránsito y en reposo.
    \item \textbf{Control de acceso:} Se emplean sistemas de autenticación y autorización para garantizar que solo el personal autorizado del proyecto pueda acceder a la información.
    \item \textbf{Anonimización y agregación:} Los reportes y visualizaciones generados presentan datos agregados que impiden la identificación de personas o empresas específicas.
    \item \textbf{Monitoreo y auditoría:} Se realizan revisiones periódicas de los sistemas para identificar vulnerabilidades y prevenir incidentes de seguridad.
    \item \textbf{Almacenamiento seguro:} Los datos se guardan en infraestructura con protocolos de seguridad certificados y con acceso restringido.
    \item \textbf{Minimización de datos:} El sistema solo procesa la información estrictamente necesaria para cumplir con las finalidades académicas y de investigación.
    \item \textbf{Pseudonimización:} Cuando es posible, se utilizan técnicas de pseudonimización para reducir el riesgo de identificación.
\end{itemize}

En caso de detectar una vulnerabilidad o posible filtración de datos, el equipo responsable tomará acciones inmediatas y notificará a las partes afectadas y a las autoridades competentes, según lo estipulado por la normativa vigente.

\subsection{Transferencia y Transmisión de Datos}

La información procesada por el Observatorio no será compartida con terceros comerciales. Sin embargo, podrá ser compartida en los siguientes casos:

\begin{itemize}
    \item \textbf{Cumplimiento Legal:} Cuando sea requerido por una autoridad judicial o administrativa, conforme a la legislación colombiana.
    \item \textbf{Fines Académicos:} Con otros investigadores o instituciones académicas, siempre garantizando la anonimización y agregación de datos para impedir la identificación de personas específicas.
    \item \textbf{Prestadores de Servicios:} En caso de que terceros sean contratados para proveer infraestructura tecnológica (servidores, servicios de nube), siempre bajo acuerdos que garanticen la protección de los datos.
    \item \textbf{Publicaciones Científicas:} Datos anonimizados y agregados podrán ser incluidos en artículos académicos, reportes de investigación o presentaciones científicas.
\end{itemize}

Los datos personales recolectados por el Observatorio serán sometidos a tratamiento automatizado mediante herramientas tecnológicas como algoritmos de NLP, modelos de inteligencia artificial, sistemas de embeddings, clustering dimensional (UMAP, HDBSCAN) y búsqueda vectorial (FAISS), con fines de análisis de habilidades, visualización de tendencias y mejora del sistema.

\newpage

\section{REVOCATORIA DE AUTORIZACIÓN Y SUPRESIÓN DE DATOS}

\subsection{Derecho a la Supresión}

Los titulares de datos pueden solicitar la eliminación de su información personal cuando:

\begin{itemize}
    \item No se respeten los principios, derechos y garantías establecidos en la normativa vigente.
    \item La información ya no sea necesaria para las finalidades académicas y de investigación con las que fue recopilada.
    \item Haya un uso indebido de los datos personales.
\end{itemize}

En estos casos, el Observatorio procederá con la supresión de los datos personales siempre que no exista una obligación legal, académica o de investigación que requiera su conservación.

\subsection{Procedimiento de Solicitud}

Para solicitar la revocatoria del consentimiento o la supresión de datos, los titulares deben:

\begin{enumerate}
    \item Enviar una solicitud escrita a los correos electrónicos: alejandro\_pinzon@javeriana.edu.co o camachoa.nicolas@javeriana.edu.co
    \item Incluir información de contacto y especificar claramente la solicitud
    \item Proporcionar identificación que permita verificar la titularidad de los datos
\end{enumerate}

El equipo responsable responderá en un plazo máximo de 15 días hábiles y procederá con la acción solicitada cuando sea procedente.

Si un titular no recibe respuesta a sus solicitudes o considera que sus derechos han sido vulnerados, puede presentar una queja ante la Superintendencia de Industria y Comercio (SIC), entidad encargada de la protección de datos personales en Colombia.

\newpage

\section{CONSULTAS Y RECLAMOS}

\subsection{Consultas}

Los titulares de los datos personales pueden realizar consultas para conocer la información que el Observatorio ha recopilado sobre ellos. Para ello, deben enviar una solicitud a través de los siguientes canales de contacto:

\begin{itemize}
    \item \textbf{Correos Electrónicos:} alejandro\_pinzon@javeriana.edu.co o camachoa.nicolas@javeriana.edu.co
    \item \textbf{Institución:} Pontificia Universidad Javeriana - Facultad de Ingeniería
    \item \textbf{Dirección:} Carrera 7 \# 40-62, Bogotá D.C., Colombia
\end{itemize}

El Observatorio responderá a las consultas en un plazo máximo de 10 días hábiles contados a partir de la recepción de la solicitud. Si no es posible atender la consulta en este tiempo, se informará al titular sobre la razón del retraso y la nueva fecha de respuesta, la cual no podrá superar los 5 días hábiles adicionales, conforme a la legislación vigente.

\subsection{Reclamos}

Si un titular considera que la información contenida en las bases de datos del Observatorio debe ser corregida, actualizada o eliminada, o si considera que se ha dado un uso inadecuado a sus datos, puede presentar un reclamo siguiendo este procedimiento:

\begin{enumerate}
    \item \textbf{Presentación del Reclamo:}
    \begin{itemize}
        \item Enviar la solicitud a través de los canales de contacto mencionados.
        \item Incluir nombre completo y datos de contacto.
        \item Especificar el motivo del reclamo y adjuntar documentos que respalden la solicitud, si aplica.
    \end{itemize}

    \item \textbf{Verificación y Respuesta:}
    \begin{itemize}
        \item Una vez recibido el reclamo completo, el Observatorio contará con 15 días hábiles para dar una respuesta.
        \item Si se requiere más tiempo, se notificará al usuario, con una prórroga máxima de 8 días hábiles adicionales.
    \end{itemize}
\end{enumerate}

\newpage

\section{VIGENCIA DE LA POLÍTICA Y ALMACENAMIENTO DE LOS DATOS}

\subsection{Vigencia}

La presente Política de Tratamiento de Datos Personales entra en vigor a partir del \textbf{30 de octubre de 2025} y permanecerá vigente mientras se mantenga en funcionamiento el Observatorio de Demanda Laboral en América Latina.

El equipo responsable se reserva el derecho de modificar esta política cuando sea necesario para adaptarse a cambios en la legislación, en la tecnología utilizada o en las finalidades del proyecto. Cualquier modificación será publicada y comunicada de manera oportuna.

\subsection{Almacenamiento de los Datos}

Los datos personales recolectados serán tratados durante el tiempo que sea necesario para cumplir con las finalidades descritas en esta política, específicamente:

\begin{itemize}
    \item \textbf{Datos de ofertas de empleo:} Se almacenarán mientras sean relevantes para el análisis de tendencias del mercado laboral, típicamente hasta 24 meses desde su recopilación.
    \item \textbf{Datos agregados y anonimizados:} Podrán conservarse indefinidamente con fines de investigación académica, ya que no permiten la identificación de personas específicas.
    \item \textbf{Datos de análisis histórico:} Se conservarán para estudios longitudinales de evolución del mercado laboral, siempre en formato anonimizado y agregado.
\end{itemize}

Posteriormente, los datos que permitan identificación serán eliminados de forma segura mediante:

\begin{itemize}
    \item Borrado permanente de registros de bases de datos
    \item Sobrescritura de información sensible
    \item Destrucción segura de respaldos que contengan información identificable
\end{itemize}

\subsection{Naturaleza de la Información Recopilada}

Es importante destacar que el Observatorio de Demanda Laboral procesa principalmente información de carácter público, proveniente de ofertas de empleo disponibles en portales especializados de acceso público en Colombia, México y Argentina. El sistema está diseñado para:

\begin{itemize}
    \item Realizar análisis agregados y estadísticos
    \item Generar insights sobre tendencias del mercado laboral
    \item No crear perfiles individuales de personas
    \item No utilizar la información con fines comerciales
    \item Mantener el enfoque exclusivamente académico y de investigación
\end{itemize}

\vfill

\begin{center}
\textbf{Observatorio de Demanda Laboral en América Latina}\\
Pontificia Universidad Javeriana\\
Facultad de Ingeniería - Departamento de Sistemas\\
Bogotá D.C., Colombia\\
\vspace{0.3cm}
alejandro\_pinzon@javeriana.edu.co | camachoa.nicolas@javeriana.edu.co
\end{center}

\end{document}
