% ============================================================================
% ESPECIFICACIÓN DE REQUERIMIENTOS DE SOFTWARE (SRS)
% Observatorio de Demanda Laboral en Tecnología en Latinoamérica
% ============================================================================

\documentclass[11pt,oneside,letterpaper]{report}

% ============================================================================
% PAQUETES
% ============================================================================
\usepackage[utf8]{inputenc}
\usepackage[spanish,es-tabla]{babel}
\usepackage[letterpaper,top=3cm,bottom=3cm,left=3cm,right=3cm]{geometry}
\usepackage{times}
\usepackage{graphicx}
\usepackage{amsmath,amssymb}
\usepackage{setspace}
\usepackage{fancyhdr}
\usepackage{titlesec}
\usepackage{tocloft}
\usepackage[hidelinks]{hyperref}
\usepackage{listings}
\usepackage{xcolor}
\usepackage{float}
\usepackage{longtable}
\usepackage{multirow}
\usepackage{array}
\usepackage{booktabs}
\usepackage[backend=biber,style=ieee,citestyle=numeric-comp,sorting=none]{biblatex}

% TikZ para diagramas
\usepackage{tikz}
\usetikzlibrary{shapes.geometric, arrows.meta, positioning, shadows, fit, shapes.multipart}

% ============================================================================
% CONFIGURACIÓN DE BIBLIOGRAFÍA
% ============================================================================
\addbibresource{bibliografia.bib}

% ============================================================================
% CONFIGURACIÓN DE CÓDIGO
% ============================================================================
\lstset{
    basicstyle=\ttfamily\small,
    breaklines=true,
    frame=single,
    numbers=left,
    numberstyle=\tiny\color{gray},
    keywordstyle=\color{blue},
    commentstyle=\color{green!60!black},
    stringstyle=\color{orange},
    showstringspaces=false
}

% ============================================================================
% CONFIGURACIÓN DE INTERLINEADO
% ============================================================================
\onehalfspacing

% ============================================================================
% CONFIGURACIÓN DE ENCABEZADOS Y PIE DE PÁGINA
% ============================================================================
\pagestyle{fancy}
\fancyhf{}
\fancyhead[L]{Pontificia Universidad Javeriana}
\fancyhead[R]{SRS - Observatorio de Demanda Laboral}
\fancyfoot[R]{Página \thepage}
\renewcommand{\headrulewidth}{0.4pt}
\renewcommand{\footrulewidth}{0.4pt}

\fancypagestyle{plain}{%
  \fancyhf{}%
  \fancyhead[L]{Pontificia Universidad Javeriana}
  \fancyhead[R]{SRS - Observatorio de Demanda Laboral}
  \fancyfoot[R]{Página \thepage}
  \renewcommand{\headrulewidth}{0.4pt}
  \renewcommand{\footrulewidth}{0.4pt}
}

% ============================================================================
% CONFIGURACIÓN DE TÍTULOS
% ============================================================================
\titleformat{\chapter}[display]
{\normalfont\Large\bfseries\centering}
{}{0pt}{\Large}

\titlespacing*{\chapter}{0pt}{20pt}{20pt}

\titleformat{\section}
{\normalfont\large\bfseries}{\thesection}{1em}{}

\titleformat{\subsection}
{\normalfont\normalsize\bfseries}{\thesubsection}{1em}{}

\titleformat{\subsubsection}
{\normalfont\normalsize\bfseries}{\thesubsubsection}{1em}{}

% ============================================================================
% INFORMACIÓN DEL DOCUMENTO
% ============================================================================
\newcommand{\proyectoTitulo}{Observatorio de Demanda Laboral en Tecnología en Latinoamérica}
\newcommand{\proyectoCodigo}{CIS2025CP08}
\newcommand{\autorUno}{Nicolas Francisco Camacho Alarcón}
\newcommand{\autorDos}{Alejandro Pinzón Fajardo}
\newcommand{\director}{Ing. Luis Gabriel Moreno Sandoval}
\newcommand{\anio}{2025}
\newcommand{\mes}{Noviembre}
\newcommand{\version}{Versión 2.1 - Fase 0 Implementada}
\newcommand{\fecha}{Noviembre 2025}

% ============================================================================
% DOCUMENTO
% ============================================================================
\begin{document}

% ============================================================================
% PORTADA
% ============================================================================
% ============================================================================
% PORTADA - SRS
% Especificación de Requerimientos de Software
% ============================================================================

\begin{titlepage}
\centering

\vspace*{1cm}

{\Large\bfseries \proyectoTitulo\par}
\vspace{0.5cm}
{[Grupo 8]\par}

\vspace{2cm}

{\Large\bfseries ESPECIFICACIÓN DE REQUERIMIENTOS DE SOFTWARE\par}

\vspace{1cm}

{[\fecha]\par}
\vspace{0.5cm}


\vspace{2cm}

\includegraphics[width=0.4\textwidth]{logo-javeriana.png}

\vfill

{\large Autores:\par}
\vspace{0.5cm}
{\large \autorUno\par}
{\large \autorDos\par}

\vspace{1cm}

{\large Pontificia Universidad Javeriana\par}
{\large Facultad de Ingeniería\par}
{\large Bogotá, Colombia\par}
{\large \mes{} de \anio\par}

\end{titlepage}


% ============================================================================
% TABLA DE CONTENIDOS
% ============================================================================
\renewcommand{\contentsname}{CONTENIDO}
\tableofcontents
\newpage

% ============================================================================
% CAPÍTULOS
% ============================================================================
\chapter{INTRODUCCIÓN}

\section{Propósito}

El presente documento tiene como propósito especificar de manera detallada los requerimientos funcionales y no funcionales del sistema denominado Observatorio de Demanda Laboral en Tecnología en Latinoamérica, una herramienta de análisis automatizado orientada a procesar, extraer y segmentar habilidades tecnológicas desde portales de empleo en línea, mediante técnicas modernas de procesamiento de lenguaje natural, scraping, embeddings semánticos y clustering no supervisado.

Este documento está dirigido principalmente a tres audiencias complementarias. En primer lugar, sirve como guía estructurada para el equipo de desarrollo del proyecto, compuesto por los estudiantes \autorUno{} y \autorDos{}, orientando la implementación y validación del sistema. En segundo lugar, constituye evidencia formal del entendimiento técnico y conceptual del producto para el director del proyecto, \director{}, y los jurados evaluadores. Finalmente, proporciona un marco de referencia para otros actores académicos o institucionales interesados en replicar o adaptar el sistema en contextos similares, tales como universidades, centros de investigación o entidades públicas vinculadas al análisis del mercado laboral.

El documento cubre la totalidad del sistema propuesto, sin limitarse a un solo módulo o subsistema. Por tanto, especifica requerimientos relacionados con la adquisición de datos mediante scraping, su procesamiento semántico, análisis estadístico y segmentación por perfiles laborales, así como aspectos de validación, modularidad, documentación técnica y estándares de calidad.

La importancia de este documento radica en su papel como contrato técnico entre los actores involucrados, asegurando una visión compartida del comportamiento esperado del sistema, las restricciones existentes, los criterios de aceptación y los estándares metodológicos adoptados. Además, facilita la trazabilidad entre los objetivos definidos en la propuesta de grado y las funcionalidades implementadas en cada fase, garantizando coherencia metodológica, control de calidad y sostenibilidad del desarrollo.

\section{Alcance}

El sistema propuesto, titulado Observatorio de Demanda Laboral en Tecnología en Latinoamérica, tiene como propósito desarrollar una herramienta automatizada capaz de analizar la evolución de las habilidades tecnológicas demandadas en el mercado laboral digital, específicamente en los países de Colombia (CO), México (MX) y Argentina (AR). El sistema abarca desde la recolección periódica de datos a través de scraping en portales de empleo hasta el procesamiento semántico y la segmentación de perfiles laborales utilizando técnicas avanzadas de NLP y clustering no supervisado.

El alcance geográfico del observatorio comprende tres países latinoamericanos: Colombia, México y Argentina. Las fuentes de datos corresponden a 7 portales de empleo principales que concentran la mayor parte de las ofertas laborales tecnológicas en la región, incluyendo Computrabajo, Bumeran, ElEmpleo, HiringCafe, OCC Mundial, ZonaJobs e Indeed.

La taxonomía base del sistema constituye una integración unificada de 14,174 habilidades, compuesta por 13,939 competencias de la base europea ESCO v1.1.0, complementadas con 152 tecnologías emergentes del catálogo O*NET Hot Technologies del sector IT, y 83 habilidades curadas manualmente específicas para el mercado tecnológico latinoamericano. El stack tecnológico seleccionado incluye Python 3.10 o superior como lenguaje base, Scrapy para extracción automatizada, spaCy para procesamiento de lenguaje natural, PostgreSQL como sistema de gestión de base de datos, FAISS para búsqueda vectorial eficiente, el modelo de embeddings multilingües E5 de 768 dimensiones, y los algoritmos HDBSCAN y UMAP para clustering y reducción dimensional respectivamente.

El producto incluye siete funcionalidades principales integradas en un pipeline secuencial. La primera funcionalidad corresponde a la extracción automatizada de vacantes desde los portales web mediante spiders especializados que respetan las normas de uso de cada sitio. La segunda funcionalidad implementa preprocesamiento y limpieza textual mediante tokenización, lematización y normalización de formatos, preparando los datos para análisis posterior. La tercera funcionalidad ejecuta la detección de habilidades explícitas e implícitas mediante técnicas híbridas que combinan reconocimiento de entidades nombradas, expresiones regulares y modelos de lenguaje.

La cuarta funcionalidad genera representaciones semánticas mediante el modelo de embeddings E5 de 768 dimensiones, seguido de reducción dimensional con UMAP para visualización y agrupación eficiente. La quinta funcionalidad mapea las habilidades extraídas contra la taxonomía ESCO mediante una estrategia de tres capas que incluye matching exacto, fuzzy matching con umbral de similitud del 85 por ciento, y semantic matching mediante búsqueda vectorial con FAISS. La sexta funcionalidad agrupa perfiles laborales mediante el algoritmo HDBSCAN, que permite segmentar la demanda en grupos funcionales coherentes sin requerir especificación previa del número de clusters. La séptima funcionalidad genera visualizaciones macro mediante gráficos interpretables y reportes estáticos, complementados por un dashboard web interactivo desarrollado con Next.js que permite consulta dinámica de métricas y tendencias. Finalmente, el sistema proporciona documentación metodológica completa y código reproducible que permite replicar o adaptar la solución a otras regiones o sectores bajo principios de ética, apertura científica y eficiencia computacional.

El alcance funcional se circunscribe al dominio de las ofertas de empleo tecnológicas publicadas en español en los países mencionados, sin contemplar vacantes en otros idiomas ni otros sectores económicos. Sin embargo, el diseño modular del sistema permitirá su adaptación futura a nuevos contextos geográficos o temáticos.

\section{Definiciones, Acrónimos y Abreviaciones}

\subsection{Portales de empleo}
Son plataformas web donde empresas publican vacantes laborales y profesionales buscan oportunidades. En este proyecto se consideran fuentes como LinkedIn, Computrabajo, Bumeran, ZonaJobs e Indeed, que constituyen insumos primarios para los procesos de scraping y análisis \cite{aguilera2018, cardenas2015}.

\subsection{Web Scraping}
Técnica de recolección automatizada de datos desde páginas web, utilizando librerías como BeautifulSoup, Selenium o Playwright. Permite extraer de forma estructurada información relevante de las ofertas publicadas \cite{orozco2019}.

\subsection{Oferta laboral}
Se refiere al anuncio publicado por una organización donde se describe el perfil buscado, incluyendo título del cargo, funciones, requisitos y habilidades deseadas \cite{rubio2024}.

\subsection{Base de datos relacional (PostgreSQL)}
Sistema que organiza los datos recolectados en tablas interconectadas, facilitando su consulta, limpieza y posterior análisis mediante estructuras SQL \cite{martinez2024}.

\subsection{Normalización de datos}
Proceso de limpieza, estandarización y unificación de formatos para reducir ambigüedad, errores y duplicados, y mejorar la coherencia del análisis posterior \cite{martinez2024}.

\subsection{Expresiones regulares (Regex)}
Lenguaje sintáctico utilizado para identificar y extraer patrones textuales específicos (como frases que contengan habilidades o requisitos) en grandes volúmenes de texto.

\subsection{Named Entity Recognition (NER)}
Técnica de procesamiento de lenguaje natural (NLP) que identifica y clasifica entidades en un texto, como nombres de habilidades, empresas o tecnologías \cite{vasquez2024, aitoskillner}.

\subsection{Tokenización}
Consiste en dividir un texto en unidades mínimas llamadas ``tokens'' (palabras, signos u oraciones), facilitando el análisis lingüístico automatizado.

\subsection{Lematización}
Proceso que transforma las palabras a su forma canónica o raíz gramatical, permitiendo uniformar variaciones morfológicas del lenguaje.

\subsection{Stopwords}
Términos frecuentes sin valor informativo (como ``de'', ``por'', ``la''), comúnmente eliminados en tareas de procesamiento textual.

\subsection{Co-ocurrencia}
Medida estadística que indica la frecuencia con que dos o más términos aparecen juntos en un texto, útil para detectar relaciones semánticas \cite{lukauskas2023}.

\subsection{Bigramas y trigramas}
Secuencias de dos o tres palabras consecutivas utilizadas para capturar patrones de lenguaje más complejos que las palabras individuales.

\subsection{LLM (Large Language Models)}
Modelos de lenguaje de gran escala (como GPT o T5) entrenados sobre corpus masivos, capaces de generar texto, extraer conocimiento implícito y realizar razonamiento contextualizado \cite{nguyen2024}.

\subsection{Prompt Engineering}
Diseño estratégico de instrucciones o ejemplos para guiar la salida de un LLM, crucial en tareas de extracción de habilidades o clasificación de ocupaciones \cite{nguyen2024}.

\subsection{Few-shot learning}
Habilidad de los LLMs para realizar tareas complejas con pocos ejemplos, lo cual resulta clave cuando se carece de datasets etiquetados masivamente en español \cite{nguyen2024}.

\subsection{Embeddings semánticos}
Representaciones numéricas de textos que capturan similitudes semánticas, permitiendo análisis cuantitativos y clustering. Ejemplos incluyen word2vec, BERT y E5 \cite{kavas2024}.

\subsection{Embeddings multilingües}
Vectores entrenados para representar texto en múltiples idiomas en un mismo espacio semántico. Son esenciales para manejar contenido mixto español-inglés en ofertas laborales \cite{kavas2025}.

\subsection{UMAP (Reducción de dimensionalidad)}
Técnica que transforma espacios de alta dimensionalidad en representaciones más simples, conservando la estructura semántica subyacente para facilitar análisis y visualización.

\subsection{Clustering (HDBSCAN)}
Algoritmo no supervisado que detecta grupos naturales de observaciones (como habilidades o perfiles laborales) según su similitud semántica, sin requerir número de clusters predefinido \cite{lukauskas2023}.

\subsection{Taxonomía de habilidades (ESCO, CIUO-08, O*NET)}
Sistemas jerárquicos y normalizados de clasificación de habilidades y ocupaciones, fundamentales para anclar el análisis a estándares internacionales y mejorar interoperabilidad de los resultados \cite{echeverria2022, campos2024}.

\subsection{FAISS (Facebook AI Similarity Search)}
Biblioteca de código abierto para búsqueda eficiente de similitud en espacios vectoriales de alta dimensionalidad. El sistema utiliza FAISS IndexFlatIP para búsqueda exacta de vecinos más cercanos con producto interno, logrando velocidades de 30,147 consultas por segundo, aproximadamente 25 veces más rápido que PostgreSQL con pgvector.

\subsection{Estrategia de tres capas (Three-layer matching)}
Metodología implementada para mapear habilidades extraídas contra la taxonomía ESCO:
\begin{itemize}
    \item Layer 1 - Exact Match: Búsqueda exacta mediante SQL ILIKE con confianza 1.0
    \item Layer 2 - Fuzzy Match: Similitud difusa con fuzzywuzzy, threshold 0.85, confianza 0.85-1.0
    \item Layer 3 - Semantic Match: Búsqueda semántica con FAISS, threshold 0.87, confianza 0.87-1.0 (actualmente deshabilitado debido a limitaciones del modelo E5 con vocabulario técnico)
\end{itemize}

\subsection{Skills emergentes}
Habilidades extraídas de ofertas laborales que no pueden ser mapeadas a la taxonomía ESCO existente. Representan el 87.4\% de las skills extraídas y constituyen una señal valiosa sobre tendencias emergentes del mercado tech latinoamericano, no un fallo del sistema.

\subsection{Natural Language Processing (NLP)}
Conjunto de técnicas de inteligencia artificial, combinando modelos de lingüística computacional, machine learning y aprendizaje profundo, para poder procesar lenguaje humano.

\subsection{Python}
Lenguaje de programación ampliamente utilizado en ciencia de datos y NLP, por su sintaxis sencilla y librerías especializadas como scikit-learn, spaCy, transformers y pandas.

\section{Apreciación Global}

El presente documento de Especificación de Requerimientos del Software tiene como objetivo presentar de manera estructurada y detallada los aspectos fundamentales del sistema Observatorio de Demanda Laboral en Tecnología en Latinoamérica. La organización del documento se ha realizado con el propósito de facilitar su comprensión tanto para usuarios técnicos como no técnicos, brindando una visión progresiva desde el contexto general hasta los requerimientos específicos del sistema.

El contenido del documento se distribuye en tres secciones principales. La Sección 1 expone la introducción general del proyecto, incluyendo su propósito, alcance, definiciones clave, referencias utilizadas y una apreciación global de su contenido. La Sección 2 describe de manera general los factores que afectan al producto, incluyendo su perspectiva dentro del ecosistema tecnológico, interfaces con otros sistemas y con el usuario, consideraciones de hardware y software, restricciones operativas, y requerimientos de adaptación al entorno de despliegue. La Sección 3 presenta los requerimientos funcionales y no funcionales del sistema, detallando exhaustivamente cada funcionalidad esperada, las restricciones técnicas y operativas, las condiciones necesarias para su correcto funcionamiento, los criterios de validación y verificación, los atributos de calidad del software, los requerimientos de base de datos, la trazabilidad entre objetivos y requerimientos, y el estado actual de implementación de cada módulo del sistema.

Este documento servirá como base para el diseño, desarrollo, validación y evaluación del sistema propuesto, asegurando que todos los actores involucrados compartan una visión clara y consensuada de los objetivos, alcances y funcionalidades del software a implementar. Además, constituye un instrumento de trazabilidad que permite verificar el cumplimiento de cada requerimiento especificado durante el ciclo de vida del proyecto.

% ============================================================================

\chapter{DESCRIPCIÓN GENERAL}

\section{Oportunidad y problema}

\subsection{Contexto del problema}

El mercado laboral en América Latina se encontró, durante la última década, en una compleja encrucijada definida por la confluencia de dos fuerzas a menudo contrapuestas: una acelerada transformación digital y la persistencia de desafíos estructurales, como una elevada informalidad laboral y brechas de capital humano \cite{echeverria2022}. La pandemia de COVID-19 actuó como un catalizador sin precedentes, intensificando la adopción de tecnologías y, con ello, la demanda de competencias digitales, al tiempo que exponía la vulnerabilidad de los mercados de trabajo de la región \cite{azuara2022}. Este dinamismo generó el riesgo de que la automatización y la digitalización, de no ser gestionadas estratégicamente, pudiesen exacerbar las desigualdades existentes, conduciendo a una mayor polarización y segmentación social \cite{echeverria2022}.

Para analizar este fenómeno regional de manera tangible y robusta, este proyecto seleccionó como casos de estudio a tres de las economías más grandes y digitalmente activas de habla hispana: Colombia, México y Argentina. La elección de estos países respondió a tres criterios estratégicos. Primero, su alto volumen de publicaciones de ofertas laborales en portales digitales aseguró la viabilidad de una recolección masiva de datos (web scraping), fundamental para el entrenamiento de modelos de lenguaje robustos \cite{aguilera2018, martinez2024, rubio2025}. Segundo, la existencia de estudios previos en cada país, aunque metodológicamente limitados, confirmó la pertinencia del problema y proporcionó una línea de base para la comparación \cite{cardenas2015, campos2024}. Y tercero, su diversidad en términos de realidades económicas, territoriales y de madurez digital permitió validar que la solución desarrollada fuese portable y adaptable a los distintos contextos que caracterizan a América Latina.

El caso de Colombia sirvió como una ilustración profunda de esta dinámica. El diagnóstico nacional previo al proyecto ya indicaba que el principal cuello de botella para la inclusión digital no era la falta de infraestructura, sino la brecha de capital humano. Específicamente, el ``Índice de Brecha Digital'' (IBD) del Ministerio de Tecnologías de la Información y las Comunicaciones reveló que la dimensión de ``Habilidades Digitales'' constituía el mayor componente individual de la brecha en el país. Esta evidencia fue posteriormente corroborada y cuantificada por el análisis empírico de la demanda laboral, el cual demostró que la pandemia generó un cambio estructural y persistente en el mercado. Se encontró que, en los 18 meses posteriores al inicio de la crisis sanitaria, las vacantes tecnológicas aumentaron en un 50\% en comparación con las no tecnológicas \cite{rubio2025}. Este cambio no fue solo cuantitativo, sino también cualitativo: se observó una marcada caída en la demanda de herramientas ofimáticas tradicionales como Excel (cuya mención en ofertas cayó del 35.8\% en 2018 al 17.4\% en 2023) y un surgimiento exponencial de tecnologías especializadas asociadas al desarrollo web y la gestión de datos, como bases de datos NoSQL (12.3\%), el framework Django (5.5\%) y la librería React (5.3\%) para el año 2023 \cite{rubio2025}.

\subsection{Formulación del problema}

A pesar de que el contexto del problema ---la creciente e insatisfecha demanda de habilidades tecnológicas--- estaba claramente identificado, los métodos existentes en la región para analizarlo presentaban limitaciones metodológicas significativas que impedían una comprensión profunda y ágil del fenómeno. Los estudios de referencia en los países seleccionados, si bien valiosos para establecer tendencias macro, se basaron en enfoques de análisis léxico y reglas manuales. En Colombia, el análisis se centró en un sistema de clasificación basado en la Clasificación Internacional Uniforme de Ocupaciones (CIUO), utilizando algoritmos de emparejamiento de texto con tokenización y métricas de similitud basadas en n-gramas \cite{rubio2025}. De forma análoga, en Argentina, los estudios se concentraron en técnicas de minería de texto con análisis de frecuencias y bigramas para identificar patrones en las ofertas del sector TI \cite{aguilera2018}. En México, el enfoque combinó datos de encuestas con scraping de portales, apoyándose en el análisis de frecuencia de términos y la creación de tipologías manuales para segmentar las habilidades \cite{martinez2024}.

La limitación fundamental compartida por estos enfoques es su dependencia de la correspondencia léxica explícita, lo que los hace incapaces de capturar la riqueza semántica del lenguaje. Estos métodos no podían detectar habilidades implícitas (aquellas que se infieren del contexto de un cargo pero no se mencionan directamente), gestionar la ambigüedad del lenguaje informal o el uso de anglicismos técnicos (``Spanglish''), ni identificar clústeres de competencias emergentes que aún no forman parte de taxonomías estandarizadas. La alta variabilidad en la redacción de las ofertas laborales, la falta de estructuras normalizadas y la rápida aparición de nuevas tecnologías hacían que estos sistemas fueran metodológicamente frágiles y requirieran un constante mantenimiento manual \cite{echeverria2022, lukauskas2023}.

En consecuencia, el problema específico que este proyecto abordó fue la ausencia de una herramienta automatizada y de extremo a extremo que, adaptada a las particularidades lingüísticas y estructurales del español latinoamericano, permitiera superar las limitaciones de los análisis léxicos tradicionales. Se identificó la necesidad de un sistema capaz de extraer, estructurar y analizar la evolución de las habilidades tecnológicas de manera semántica, escalable y con un mayor grado de autonomía, integrando para ello técnicas avanzadas de Procesamiento de Lenguaje Natural (NLP), enriquecimiento contextual con Large Language Models (LLMs) y algoritmos de agrupamiento no supervisado.

\subsection{Propuesta de solución}

Para dar respuesta al problema formulado, se diseñó e implementó un observatorio de demanda laboral tecnológica basado en un pipeline modular y automatizado, un proyecto enmarcado en las áreas de Ingeniería de Sistemas y Ciencia de Datos. El sistema fue concebido como una solución de extremo a extremo que integró las etapas de recolección, procesamiento, análisis semántico y segmentación de ofertas de empleo publicadas en Colombia, México y Argentina. El objetivo fue crear una arquitectura robusta, replicable y adaptada a las complejidades del contexto latinoamericano, superando las limitaciones de los enfoques puramente léxicos o manuales.

La solución se materializó a través de un sistema compuesto por módulos secuenciales y cohesivos. El primer módulo consistió en un motor de adquisición de datos que, mediante técnicas de web scraping, extrajo de forma sistemática y ética decenas de miles de ofertas laborales de portales de empleo clave en la región. El núcleo del sistema fue su arquitectura de extracción dual, compuesta por dos pipelines paralelos.

El primero, denominado Pipeline A (tradicional), implementó un método de extracción basado en Reconocimiento de Entidades Nombradas (NER) utilizando un EntityRuler de spaCy, poblado con la taxonomía completa de ESCO, combinado con expresiones regulares para capturar un baseline de habilidades explícitas de alta precisión.

El segundo, Pipeline B (basado en LLMs), empleó Large Language Models (LLMs) como Llama 3 para realizar una extracción semántica, capaz de identificar no solo habilidades explícitas sino también de inferir competencias implícitas a partir del contexto de la vacante, siguiendo enfoques de vanguardia \cite{herandi2024, nguyen2024}.

Posteriormente, un módulo de mapeo de dos capas normalizó las habilidades extraídas por ambos pipelines contra la taxonomía ESCO. La primera capa realizó una coincidencia léxica (exacta y difusa), mientras que la segunda ejecutó una búsqueda de similitud semántica de alto rendimiento, utilizando embeddings multilingües (E5) y un índice FAISS pre-calculado, inspirado en las arquitecturas de herramientas como ESCOX \cite{kavargyris2025}. Finalmente, un módulo de análisis no supervisado aplicó una secuencia metodológica de embeddings, reducción de dimensionalidad con UMAP y agrupamiento con HDBSCAN para identificar clústeres de habilidades y perfiles emergentes, un enfoque validado por la literatura para el descubrimiento de estructuras en el mercado laboral \cite{lukauskas2023}.

\subsection{Justificación de la solución}

La solución implementada se justificó como una alternativa superior y mejor adaptada para el análisis de la demanda de habilidades en América Latina, ya que abordó directamente las debilidades metodológicas identificadas en los estudios previos. A diferencia de los enfoques basados exclusivamente en reglas léxicas \cite{aguilera2018, rubio2025} o en el uso aislado de LLMs \cite{nguyen2024}, la arquitectura de dos pipelines paralelos permitió una validación empírica cruzada: combinó la auditabilidad y alta precisión para habilidades conocidas del Pipeline A con la potencia inferencial y la capacidad de descubrir habilidades implícitas del Pipeline B. Este diseño comparativo proveyó un marco para evaluar objetivamente el rendimiento de los LLMs, en lugar de depender únicamente de su capacidad ``black-box''.

Técnicamente, el sistema representó un avance significativo en escalabilidad y eficiencia. La implementación de un índice FAISS para la búsqueda semántica de similitud (una mejora sobre la propuesta original de ESCOX) permitió procesar grandes volúmenes de datos a una velocidad órdenes de magnitud superior a las búsquedas en bases de datos vectoriales convencionales, haciendo factible el análisis de todo el corpus recolectado \cite{kavargyris2025, lukauskas2023}. Adicionalmente, el sistema fue diseñado explícitamente para la realidad del español latinoamericano. Este enfoque abordó directamente una limitación crítica de trabajos de vanguardia en LLMs, los cuales se han desarrollado y validado casi exclusivamente sobre datasets en inglés \cite{herandi2024}, ignorando las particularidades lingüísticas (como el ``Spanglish'') del dominio tecnológico en la región.

Finalmente, el valor agregado del proyecto residió en su síntesis estratégica de metodologías de vanguardia. El sistema no se limitó a una sola técnica, sino que articuló la cobertura del scraping regional, la potencia de los LLMs ajustados para generar salidas estructuradas \cite{herandi2024}, y la capacidad estructuradora del clustering semántico \cite{lukauskas2023}. Al hacerlo, se desarrolló un observatorio más completo, robusto y metodológicamente transparente que las alternativas existentes, estableciendo una base sólida y replicable para el monitoreo dinámico de la demanda laboral en la región.

\section{Descripción del proyecto}

El proyecto se concibió como un observatorio automatizado para capturar, normalizar y analizar avisos de empleo en Latinoamérica. Se integraron múltiples portales (CO, MX y AR), se diseñó una base de datos relacional con soporte vectorial, y se implementó un pipeline de extracción de habilidades (NER/regex/LLM) alineadas a ESCO, con generación de indicadores, visualizaciones y reportes. Operativamente, se planificó escalar hasta 600.000 avisos para la defensa, garantizando calidad, trazabilidad y reproducibilidad.

\subsection{Objetivo general}

Desarrollar un sistema que permita procesar y segmentar la demanda de habilidades tecnológicas en Colombia, México y Argentina, mediante técnicas de procesamiento de lenguaje natural.

\subsection{Objetivos específicos}

\begin{itemize}
    \item Construir un estado del arte exhaustivo para comparar trabajos existentes en el ámbito de observatorios laborales automatizados y técnicas de procesamiento de lenguaje natural en español.

    \item Diseñar una arquitectura modular, escalable y reutilizable para el observatorio laboral automatizado, fundamentada en las mejores prácticas identificadas en el estado del arte.

    \item Implementar e integrar técnicas de inteligencia artificial para la identificación, normalización y agrupación semántica de habilidades tecnológicas en ofertas laborales en español.

    \item Validar el desempeño y la robustez de la arquitectura y los modelos propuestos mediante métricas cuantitativas y estudios empíricos.
\end{itemize}

\subsection{Entregables, estándares y justificación}

El desarrollo del observatorio se materializó en un conjunto estructurado de entregables alineados con estándares de ingeniería de software y buenas prácticas de la industria. La Tabla \ref{tab:entregables-estandares} presenta cada componente desarrollado, los estándares técnicos que guiaron su implementación, y la justificación que fundamenta su adhesión a dichos estándares.

\begin{longtable}{|p{5cm}|p{5cm}|p{5cm}|}
\caption{Entregables, Estándares y Justificación Técnica}
\label{tab:entregables-estandares} \\
\hline
\textbf{Entregable} & \textbf{Estándares asociados} & \textbf{Justificación} \\
\hline
\endfirsthead

\multicolumn{3}{c}%
{\tablename\ \thetable\ -- \textit{Continuación de la página anterior}} \\
\hline
\textbf{Entregable} & \textbf{Estándares asociados} & \textbf{Justificación} \\
\hline
\endhead

\hline
\endfoot

\hline
\endlastfoot

Repositorio de código (spiders, orquestador, pipelines) & PEP 8/257/484; Conv. Commits; SemVer & Mantenibilidad, legibilidad y control de versiones. \\
\hline

Esquema BD y migraciones (PostgreSQL + pgvector) & Normalización (3NF); SQL best practices & Integridad, trazabilidad y soporte a consultas vectoriales. \\
\hline

Spiders y configuración de scraping & Polite crawling (delays/retries); manejo anti-bots & Captura estable a escala y resiliencia ante cambios UI. \\
\hline

Orquestador CLI + scheduler & CLI UX (Typer); jobs idempotentes & Operación reproducible, programable y auditable. \\
\hline

Módulo de extracción/normalización de habilidades & ISO/IEC/IEEE 29148 (requisitos); ESCO & Consistencia semántica y comparabilidad entre países. \\
\hline

Embeddings y análisis (E5, UMAP, HDBSCAN) & Procedimientos reproducibles; semillas fijas & Descubrimiento de patrones y replicabilidad experimental. \\
\hline

Datasets consolidados (CSV/JSON) + diccionario de datos & Esquemas declarativos; control de versiones & Consumo externo y verificación de calidad. \\
\hline

Documentación técnica y de proyecto (SRS, SPMP, VFP, manuales) & IEEE 1058 (plan de proyecto); 29148 (requisitos) & Alineación con buenas prácticas y transferencia de conocimiento. \\
\hline

Reportes y visualizaciones (PDF/PNG/CSV) & Principios de visualización; metadatos & Comunicación clara de hallazgos a públicos no técnicos. \\
\hline

Plan de operación y mantenimiento (Docker/monitoring) & Buenas prácticas Docker/Logging & Despliegue consistente y observabilidad del sistema. \\
\hline

\end{longtable}

\chapter{REQUERIMIENTOS ESPECÍFICOS}

Este capítulo detalla de manera exhaustiva los requerimientos funcionales y no funcionales del sistema, organizados según las categorías definidas en las secciones anteriores.

\section{Requerimientos de Interfaces Externas}

\subsection{Interfaces con el Usuario}

\begin{description}
    \item[REI-01] El sistema debe permitir la ejecución de scripts desde CLI para scraping y visualización.
    \begin{itemize}
        \item Prioridad: Alta
        \item Módulo: Interfaz Usuario
        \item Criterio: Ejecución funcional por línea de comandos
    \end{itemize}

    \item[REI-02] El sistema debe incluir notebooks Jupyter con celdas documentadas.
    \begin{itemize}
        \item Prioridad: Media
        \item Módulo: Interfaz Usuario
        \item Criterio: Visualización y reproducción de notebooks
    \end{itemize}

    \item[REI-03] El sistema debe generar reportes en formato PDF, PNG y HTML.
    \begin{itemize}
        \item Prioridad: Alta
        \item Módulo: Interfaz Usuario/Visualización
        \item Criterio: Visualización correcta de reportes generados
    \end{itemize}
\end{description}

\subsection{Interfaces con el Hardware}

\begin{description}
    \item[REI-04] El sistema debe operar en equipos sin GPU dedicada y mínimo 8 GB de RAM.
    \begin{itemize}
        \item Prioridad: Alta
        \item Módulo: Interfaz Hardware
        \item Criterio: Ejecución sin errores en equipos personales
    \end{itemize}
\end{description}

\subsection{Interfaces con el Software}

\begin{description}
    \item[REI-05] El sistema debe conectarse a una base de datos PostgreSQL local.
    \begin{itemize}
        \item Prioridad: Alta
        \item Módulo: Interfaz Software
        \item Criterio: Inserción y lectura desde PostgreSQL
    \end{itemize}

    \item[REI-06] El sistema debe acceder a portales web por HTTP/HTTPS para extraer datos.
    \begin{itemize}
        \item Prioridad: Alta
        \item Módulo: Interfaz Comunicación
        \item Criterio: Scraping exitoso desde URLs definidas
    \end{itemize}
\end{description}

\section{Requerimientos Funcionales}

\subsection{Funcionalidad 1: Extracción de Vacantes (Scraping)}

\begin{description}
    \item[RF-01] El sistema debe extraer vacantes desde portales como Computrabajo, Bumeran y elempleo.com.
    \begin{itemize}
        \item Prioridad: Alta
        \item Módulo: Scraping
        \item Criterio: Extracción visible y consistente de datos
    \end{itemize}
\end{description}

\subsection{Funcionalidad 2: Procesamiento de Texto}

\begin{description}
    \item[RF-02] El sistema debe almacenar las vacantes en PostgreSQL con deduplicación SHA256.
    \begin{itemize}
        \item Prioridad: Alta
        \item Módulo: Almacenamiento
        \item Criterio: Consultas e inserciones validadas, duplicados detectados
        \item Estado: IMPLEMENTADO (23,352 jobs, dedup rate 0.5\%)
    \end{itemize}

    \item[RF-03] El sistema debe preprocesar el texto: limpieza HTML, tokenización, normalización.
    \begin{itemize}
        \item Prioridad: Alta
        \item Módulo: Procesamiento NLP
        \item Criterio: Verificación de campos procesados, detección de jobs basura
        \item Estado: IMPLEMENTADO (23,188 jobs limpios, 99.5\% usable)
    \end{itemize}

    \item[RF-04] El sistema debe extraer habilidades explícitas mediante NER y Regex (Pipeline A).
    \begin{itemize}
        \item Prioridad: Alta
        \item Módulo: Extracción
        \item Criterio: Skills extraídas con método y confianza asociados
        \item Métodos:
        \begin{itemize}
            \item Regex: 200+ patrones tecnológicos (78-89\% precision)
            \item NER: spaCy + custom entity ruler (13\% precision con filtros)
        \end{itemize}
        \item Estado: IMPLEMENTADO (Test 100 jobs: 2,756 skills extraídas)
    \end{itemize}
\end{description}

\subsection{Funcionalidad 2.1: Mapeo contra Taxonomía ESCO}

\begin{description}
    \item[RF-04.1] El sistema debe cargar y mantener una taxonomía unificada de 14,174 skills.
    \begin{itemize}
        \item Prioridad: Alta
        \item Módulo: Taxonomía (Fase 0)
        \item Criterio: ESCO v1.1.0 (13,939) + O*NET Hot Tech (152) + Manual Curated (83)
        \item Estado: IMPLEMENTADO
    \end{itemize}

    \item[RF-04.2] El sistema debe mapear skills extraídas contra ESCO usando estrategia de dos capas.
    \begin{itemize}
        \item Prioridad: Alta
        \item Módulo: Matching
        \item Criterio: Layer 1 (Exact, SQL ILIKE, confidence 1.0), luego Layer 2 (Fuzzy, threshold 0.92). Layer 3 (Semantic) fue deshabilitada tras pruebas que revelaron falsos positivos en contexto técnico.
        \item Estado: IMPLEMENTADO (Layer 1 + 2 activos)
        \item Match rate actual: 12.6\% (esperado para taxonomías 2016-2017)
    \end{itemize}

    \item[RF-04.3] El sistema debe identificar y rastrear skills emergentes no mapeadas.
    \begin{itemize}
        \item Prioridad: Alta
        \item Módulo: Emergent Skills Tracking
        \item Criterio: Skills sin match ESCO almacenadas con frecuencia y contexto
        \item Estado: IMPLEMENTADO (87.4\% emergent rate en test de 100 jobs)
    \end{itemize}

    \item[RF-04.4] El sistema debe generar embeddings semánticos para todas las skills.
    \begin{itemize}
        \item Prioridad: Alta
        \item Módulo: Embeddings (Fase 0)
        \item Criterio: Modelo intfloat/multilingual-e5-base (768D), L2-normalized
        \item Performance: 721 skills/segundo (GPU), 30 skills/segundo (CPU)
        \item Estado: IMPLEMENTADO (14,133 embeddings, 94.6\% test pass)
    \end{itemize}

    \item[RF-04.5] El sistema debe construir y mantener índice FAISS para búsqueda semántica.
    \begin{itemize}
        \item Prioridad: Alta
        \item Módulo: FAISS Index (Fase 0)
        \item Criterio: IndexFlatIP (exact search), 30,147 queries/segundo
        \item Archivos: data/embeddings/esco.faiss (41.41 MB), esco\_mapping.pkl (545 KB)
        \item Estado: IMPLEMENTADO (25x más rápido que PostgreSQL pgvector)
    \end{itemize}
\end{description}

\subsection{Funcionalidad 3: Representación Semántica}

\begin{description}
    \item[RF-05] El sistema debe generar representaciones semánticas (embeddings) y realizar clustering automático.
    \begin{itemize}
        \item Prioridad: Alta
        \item Módulo: Agrupamiento
        \item Criterio: Clústeres coherentes generados
    \end{itemize}
\end{description}

\subsection{Funcionalidad 4: Visualización}

\begin{description}
    \item[RF-06] El sistema debe generar visualizaciones estáticas con gráficas de frecuencia de habilidades y comparativas.
    \begin{itemize}
        \item Prioridad: Alta
        \item Módulo: Visualización
        \item Criterio: Visualizaciones exportadas exitosamente
    \end{itemize}
\end{description}

\section{Requerimientos de Desempeño}

\begin{description}
    \item[RDP-01] El sistema debe extraer y almacenar vacantes desde múltiples portales de empleo.
    \begin{itemize}
        \item Prioridad: Alta
        \item Módulo: Scraping
        \item Criterio: Mínimo 300 vacantes por país desde dos portales distintos
        \item Performance actual: 23,352 jobs scraped (hiring.cafe: 23,313, elempleo: 38, zonajobs: 1)
        \item Estado: CUMPLIDO (7,784\% sobre mínimo requerido)
    \end{itemize}

    \item[RDP-02] El preprocesamiento textual debe ejecutarse sin errores críticos.
    \begin{itemize}
        \item Prioridad: Alta
        \item Módulo: Procesamiento
        \item Criterio: Pipeline completado con success rate > 95\%
        \item Performance actual: 99.5\% usable rate (23,188/23,352 jobs)
        \item Estado: CUMPLIDO
    \end{itemize}

    \item[RDP-03] La generación de embeddings debe completarse en tiempo razonable.
    \begin{itemize}
        \item Prioridad: Alta
        \item Módulo: Embeddings (Fase 0)
        \item Criterio: Completar 14K skills en < 5 minutos con GPU
        \item Performance actual: 19.65 segundos para 14,133 skills (721 skills/seg)
        \item Estado: CUMPLIDO (15x mejor que requerimiento)
    \end{itemize}

    \item[RDP-04] La búsqueda semántica con FAISS debe ser eficiente.
    \begin{itemize}
        \item Prioridad: Alta
        \item Módulo: FAISS Index
        \item Criterio: Mínimo 100 queries por segundo
        \item Performance actual: 30,147 queries/segundo (301x sobre requerimiento)
        \item Estado: CUMPLIDO (25x más rápido que PostgreSQL pgvector)
    \end{itemize}

    \item[RDP-05] El sistema de extracción de skills debe procesar jobs sin fallos.
    \begin{itemize}
        \item Prioridad: Alta
        \item Módulo: Extracción Pipeline A
        \item Criterio: Success rate > 95\% en procesamiento
        \item Performance actual: 100\% success rate en test de 100 jobs (2,756 skills extraídas)
        \item Tiempo promedio: 1.82 segundos por job
        \item Estado: CUMPLIDO
    \end{itemize}

    \item[RDP-06] El matching contra ESCO debe completarse para todas las skills extraídas.
    \begin{itemize}
        \item Prioridad: Alta
        \item Módulo: Matching (2-layer strategy)
        \item Criterio: Todas las skills procesadas por las 2 layers activas (exact + fuzzy con threshold 0.92)
        \item Performance actual: 12.6\% match rate (Layer 1: 5.4\%, Layer 2: 7.1\%)
        \item Estado: CUMPLIDO (Layer 3 semantic deshabilitada por falsos positivos)
        \item Nota: 87.4\% emergent skills es esperado para taxonomías 2016-2017
    \end{itemize}
\end{description}

\section{Restricciones de Diseño}

\begin{description}
    \item[RDZ-01] El sistema debe ejecutarse completamente en local, sin servicios web de pago.
    \begin{itemize}
        \item Prioridad: Alta
        \item Módulo: Arquitectura General
        \item Criterio: Funciona sin acceso a servicios externos
    \end{itemize}

    \item[RDZ-02] La ejecución de modelos LLM se limitará a versiones descargables.
    \begin{itemize}
        \item Prioridad: Alta
        \item Módulo: Enriquecimiento
        \item Criterio: Modelos configurados desde Hugging Face
    \end{itemize}

    \item[RDZ-03] El desarrollo deberá realizarse en Python 3.10 o superior.
    \begin{itemize}
        \item Prioridad: Alta
        \item Módulo: Infraestructura
        \item Criterio: Repositorio sin dependencias privativas
    \end{itemize}

    \item[RDZ-04] No se implementará una interfaz gráfica interactiva.
    \begin{itemize}
        \item Prioridad: Alta
        \item Módulo: Visualización
        \item Criterio: El sistema exporta reportes, no tiene frontend
    \end{itemize}

    \item[RDZ-05] El sistema debe ser modular.
    \begin{itemize}
        \item Prioridad: Alta
        \item Módulo: Arquitectura General
        \item Criterio: Cada módulo puede lanzarse individualmente
    \end{itemize}
\end{description}

\section{Atributos del Sistema de Software (No funcionales)}

\subsection{Confiabilidad}

\begin{description}
    \item[RNF-01] El sistema debe producir resultados consistentes ante entradas iguales.
    \begin{itemize}
        \item Prioridad: Alta
        \item Módulo: Todos
        \item Criterio: Repetición del proceso genera mismos resultados
    \end{itemize}

    \item[RNF-02] Se debe registrar el comportamiento del sistema mediante logs detallados.
    \begin{itemize}
        \item Prioridad: Alta
        \item Módulo: Todos
        \item Criterio: Archivos de log por módulo
    \end{itemize}

    \item[RNF-03] En caso de interrupciones, los módulos deben permitir ser reiniciados.
    \begin{itemize}
        \item Prioridad: Alta
        \item Módulo: Arquitectura General
        \item Criterio: Pipeline puede reiniciarse parcialmente
    \end{itemize}
\end{description}

\subsection{Disponibilidad}

\begin{description}
    \item[RNF-04] El sistema estará disponible para ejecución local en cualquier momento.
    \begin{itemize}
        \item Prioridad: Alta
        \item Módulo: Infraestructura
        \item Criterio: Pipeline completo corre offline
    \end{itemize}

    \item[RNF-05] Scripts y notebooks deben estar organizados en GitHub.
    \begin{itemize}
        \item Prioridad: Alta
        \item Módulo: Infraestructura
        \item Criterio: Repositorio contiene notebooks funcionales
    \end{itemize}
\end{description}

\subsection{Seguridad}

\begin{description}
    \item[RNF-06] Se evitará recolectar información personal.
    \begin{itemize}
        \item Prioridad: Alta
        \item Módulo: Scraping
        \item Criterio: Ningún campo sensible almacenado
    \end{itemize}

    \item[RNF-07] Los spiders implementarán throttling.
    \begin{itemize}
        \item Prioridad: Alta
        \item Módulo: Scraping
        \item Criterio: Tiempo entre requests configurable
    \end{itemize}

    \item[RNF-08] El código incluirá controles básicos de errores.
    \begin{itemize}
        \item Prioridad: Alta
        \item Módulo: Todos
        \item Criterio: Pipeline continúa sin detenerse
    \end{itemize}
\end{description}

\subsection{Mantenibilidad}

\begin{description}
    \item[RNF-09] El sistema debe estar documentado a nivel de código.
    \begin{itemize}
        \item Prioridad: Alta
        \item Módulo: Todos
        \item Criterio: Documentación presente en repositorio
    \end{itemize}

    \item[RNF-10] Cada módulo será independiente y versionado.
    \begin{itemize}
        \item Prioridad: Alta
        \item Módulo: Arquitectura Modular
        \item Criterio: Se puede actualizar un módulo sin afectar demás
    \end{itemize}
\end{description}

\subsection{Portabilidad}

\begin{description}
    \item[RNF-11] El sistema debe poder ejecutarse en Linux, macOS o Windows con WSL.
    \begin{itemize}
        \item Prioridad: Alta
        \item Módulo: Infraestructura
        \item Criterio: Se ejecuta en tres sistemas operativos
    \end{itemize}

    \item[RNF-12] Se debe proporcionar un archivo de entorno reproducible.
    \begin{itemize}
        \item Prioridad: Alta
        \item Módulo: Infraestructura
        \item Criterio: Archivo requirements.txt ejecutado con éxito
    \end{itemize}
\end{description}

\section{Requerimientos de la Base de Datos}

\begin{description}
    \item[BD-01] El sistema debe utilizar PostgreSQL como sistema de gestión de base de datos.
    \begin{itemize}
        \item Prioridad: Alta
        \item Módulo: Almacenamiento
        \item Criterio: PostgreSQL 13+ con soporte para arrays REAL[]
        \item Estado: IMPLEMENTADO
    \end{itemize}

    \item[BD-02] La base de datos debe almacenar vacantes con deduplicación SHA256.
    \begin{itemize}
        \item Prioridad: Alta
        \item Módulo: Almacenamiento
        \item Criterio: Hash único por job\_description para prevenir duplicados
        \item Estado: IMPLEMENTADO (0.5\% dedup rate)
    \end{itemize}

    \item[BD-03] La base de datos debe almacenar skills extraídas con método y confianza.
    \begin{itemize}
        \item Prioridad: Alta
        \item Módulo: Extracción
        \item Criterio: Tabla job\_extracted\_skills con campos: job\_id, skill, method, confidence
        \item Estado: IMPLEMENTADO
    \end{itemize}

    \item[BD-04] La base de datos debe almacenar embeddings de 768 dimensiones.
    \begin{itemize}
        \item Prioridad: Alta
        \item Módulo: Embeddings
        \item Criterio: Tabla skill\_embeddings con columna embedding REAL[768]
        \item Estado: IMPLEMENTADO (14,133 embeddings almacenados)
    \end{itemize}

    \item[BD-05] La base de datos debe mantener trazabilidad de matching ESCO.
    \begin{itemize}
        \item Prioridad: Alta
        \item Módulo: Matching
        \item Criterio: Tabla job\_esco\_matches con layer, confidence score, skill\_id
        \item Estado: IMPLEMENTADO
    \end{itemize}
\end{description}

\section{Trazabilidad de Requerimientos}

La trazabilidad entre los objetivos del proyecto y los requerimientos específicos se presenta en la siguiente tabla:

\begin{table}[H]
\centering
\begin{tabular}{|p{3cm}|p{10cm}|}
\hline
Objetivo & Requerimientos Relacionados \\
\hline
Extracción automatizada de vacantes & RF-01, RDP-01, RNF-06, RNF-07, REI-06 \\
\hline
Procesamiento y limpieza textual & RF-02, RF-03, RDP-02, BD-01, BD-02 \\
\hline
Extracción de habilidades & RF-04, RF-04.1, RF-04.2, RF-04.3, RDP-05, RDP-06, BD-03, BD-05 \\
\hline
Generación de embeddings & RF-04.4, RF-04.5, RDP-03, RDP-04, BD-04 \\
\hline
Clustering de perfiles & RF-05, RDP-03 \\
\hline
Visualización de resultados & RF-06, REI-03, RDZ-04 \\
\hline
Arquitectura modular & RDZ-01, RDZ-03, RDZ-05, RNF-10 \\
\hline
Mantenibilidad y reproducibilidad & RNF-09, RNF-12, REI-01, REI-02 \\
\hline
\end{tabular}
\caption{Trazabilidad entre objetivos y requerimientos}
\end{table}

\section{Validación y Verificación}

Para cada requerimiento especificado se define un método de validación:

\begin{itemize}
    \item Validación por Inspección (VI): Revisión manual de código, logs y resultados
    \item Validación por Prueba (VP): Ejecución de scripts de test automatizados
    \item Validación por Demostración (VD): Ejecución completa del pipeline con datos reales
    \item Validación por Análisis (VA): Evaluación de métricas, tiempos y performance
\end{itemize}

\begin{table}[H]
\centering
\begin{tabular}{|p{2.5cm}|p{2cm}|p{8cm}|}
\hline
Requerimiento & Método & Criterio de Aceptación \\
\hline
RF-01 & VD & Scraping exitoso desde 3+ portales con > 300 jobs por país \\
\hline
RF-02 & VP & Deduplicación funcional, 0 duplicados en test de 1000 jobs \\
\hline
RF-03 & VI & Texto limpio sin HTML, tokens válidos, 552 palabras promedio \\
\hline
RF-04 & VP & Extracción exitosa con 2+ métodos (Regex + NER), success rate > 95\% \\
\hline
RF-04.1 & VP & Taxonomía cargada: 14,174 skills (ESCO + O*NET + Manual) \\
\hline
RF-04.2 & VA & Match rate > 10\%, Layer 1 + Layer 2 activos, threshold fuzzy 0.92 \\
\hline
RF-04.3 & VI & Skills emergentes rastreadas con frecuencia y contexto \\
\hline
RF-04.4 & VA & Embeddings generados en < 5 min (GPU), L2-normalized, 768D \\
\hline
RF-04.5 & VA & FAISS construido en < 1 min, queries > 100/seg \\
\hline
RF-05 & VD & Clustering ejecutado sin errores, clusters coherentes \\
\hline
RF-06 & VI & Visualizaciones exportadas en PDF/PNG/HTML \\
\hline
RDP-01 a RDP-06 & VA & Métricas de performance validadas contra umbrales \\
\hline
RNF-01 a RNF-12 & VI/VD & Tests de reproducibilidad, logs, portabilidad \\
\hline
BD-01 a BD-05 & VP & Queries SQL exitosas, integridad referencial \\
\hline
\end{tabular}
\caption{Métodos de validación por requerimiento}
\end{table}

\section{Estado de Implementación - Resumen}

\begin{table}[H]
\centering
\begin{tabular}{|p{4cm}|p{3cm}|p{6cm}|}
\hline
Módulo/Fase & \textbf{Estado} & Observaciones \\
\hline
Fase 0 - Setup Inicial & COMPLETADO & Taxonomía, embeddings, FAISS index \\
\hline
Fase 1 - Scraping & COMPLETADO & 23,352 jobs, 3 portales \\
\hline
Fase 2 - Pipeline A & COMPLETADO & Regex + NER + Mapping activo \\
\hline
Fase 2 - Pipeline B & COMPLETADO & Extracción con LLMs \\
\hline
Módulo 6 - Clustering & COMPLETADO & HDBSCAN + UMAP \\
\hline
Módulo 7 - Visualización & COMPLETADO & Plotly + reportes analíticos \\
\hline
\end{tabular}
\caption{Estado de implementación por módulo}
\end{table}


% ============================================================================
% REFERENCIAS
% ============================================================================
\chapter*{REFERENCIAS}
\addcontentsline{toc}{chapter}{REFERENCIAS}
\printbibliography[heading=none]

% ============================================================================
% APÉNDICES
% ============================================================================
\chapter*{APÉNDICES}
\addcontentsline{toc}{chapter}{APÉNDICES}

En esta sección del documento se presentan los apéndices relacionados con el SRS, incluyendo diagramas adicionales, especificaciones técnicas detalladas y documentación complementaria.

\section*{A. Diagramas del Sistema}

\subsection*{A.1 Diagrama de Casos de Uso}

\begin{figure}[H]
    \centering
    \includegraphics[width=0.75\textwidth,height=0.85\textheight,keepaspectratio]{figures/DiagramaCasosdeUso.png}
    \caption{Diagrama de Casos de Uso del Sistema}
    \label{fig:casos-uso}
\end{figure}

\newpage

\subsection*{A.2 Esquema de Base de Datos}

\begin{figure}[H]
    \centering
    \includegraphics[width=0.90\textwidth,height=0.75\textheight,keepaspectratio]{figures/DiagramaER.png}
    \caption{Diagrama Entidad-Relación de la Base de Datos PostgreSQL}
    \label{fig:diagrama-er}
\end{figure}

\newpage

\subsection*{A.3 Arquitectura del Sistema}

\begin{figure}[H]
    \centering
    \includegraphics[width=0.85\textwidth,height=0.75\textheight,keepaspectratio]{figures/pipeline_arquitectura.png}
    \caption{Arquitectura del Pipeline de Procesamiento}
    \label{fig:arquitectura}
\end{figure}

\end{document}
