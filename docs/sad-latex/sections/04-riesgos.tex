\section{Riesgos}

Los riesgos identificados se agrupan en tres categorías principales: riesgos de producto, riesgos de proceso y riesgos de proyecto. Cada uno puede afectar la calidad, validez científica y éxito del observatorio.

\subsection{Riesgos de Producto}

Estos riesgos se relacionan con la calidad, precisión, rendimiento y fiabilidad del sistema final.

\subsubsection{Riesgos de Precisión}

\begin{itemize}
    \item \textbf{Falsos positivos en extracción NER}: Extracción de frases genéricas o disclaimers legales como skills técnicas (ej. ``national origin'', ``aspirar a la excelencia''). Ya identificado en pruebas con match rate de 10.6\% y 87.4\% emergent skills, requiere mejora de filtros NER.

    \item \textbf{Degradación de modelos pre-entrenados}: spaCy es\_core\_news\_lg y E5 multilingual fueron entrenados en lenguaje general, no especializado en tech jobs de LatAm. Posible bajo rendimiento en Spanglish y jerga técnica local.

    \item \textbf{Baja cobertura de ESCO}: Taxonomía ESCO v1.1.0 data de 2016-2017, no cubre frameworks modernos (Next.js, SolidJS, Remix). Match rate de 12.6\% es esperado pero puede limitar análisis comparativo.
\end{itemize}

\subsubsection{Riesgos de Rendimiento}

\begin{itemize}
    \item \textbf{Latencia acumulativa en Pipeline B}: Procesamiento con LLM puede tomar 5-10 segundos por oferta. Para 600K ofertas = 833 horas de cómputo (34 días continuos). Puede hacer inviable procesamiento completo del corpus objetivo.

    \item \textbf{Cuellos de botella en I/O de BD}: Inserts/updates frecuentes en PostgreSQL durante extracción pueden saturar I/O del disco. Mitigado con batch processing pero requiere monitoreo.

    \item \textbf{Memoria insuficiente para UMAP}: Reducción dimensional de 14K+ embeddings de 768D requiere ~10GB RAM. Servidores limitados pueden fallar en esta etapa.
\end{itemize}

\subsubsection{Riesgos de Fiabilidad}

\begin{itemize}
    \item \textbf{Pérdida de datos por fallos de hardware}: Scraping de meses puede perderse por fallo de disco sin backups. Sistema académico sin infraestructura de alta disponibilidad.

    \item \textbf{Corrupción de embeddings}: Generación de embeddings interrumpida puede dejar skill\_embeddings table en estado inconsistente. Difícil de detectar sin validación exhaustiva.

    \item \textbf{Dependencia de servicios externos}: Scrapers dependen de portales web que pueden cambiar HTML structure, implementar rate limiting más agresivo, o bloquear IPs.
\end{itemize}

\subsection{Riesgos de Proceso}

Estos riesgos están asociados al desarrollo, experimentación y mantenimiento del sistema.

\subsubsection{Riesgos de Experimentación Científica}

\begin{itemize}
    \item \textbf{Sesgo de selección en Gold Standard}: Anotación manual de 300 ofertas puede tener sesgos (ej. sobre-representación de Python jobs, sub-representación de .NET). Invalida evaluación comparativa de Pipelines A vs B.

    \item \textbf{Inter-annotator disagreement}: Dos anotadores pueden discrepar en qué constituye una ``skill''. Cohen's Kappa $<$0.80 invalida Gold Standard.

    \item \textbf{Overfitting a ESCO}: Sistema optimizado para maximizar match rate con ESCO puede perder skills emergentes valiosas. Sesgo hacia skills tradicionales europeas vs. innovaciones LatAm.
\end{itemize}

\subsubsection{Riesgos de Mantenibilidad}

\begin{itemize}
    \item \textbf{Complejidad de debugging de pipeline de 8 etapas}: Error en Etapa 7 (clustering) puede ser causado por problema en Etapa 5 (embeddings) o Etapa 3 (extracción). Trazabilidad completa mitiga pero no elimina complejidad.

    \item \textbf{Falta de documentación de decisiones experimentales}: Cambios en parámetros (ej. UMAP n\_neighbors 10$\rightarrow$15) sin documentar impactan reproducibilidad.

    \item \textbf{Dependencia de expertos en dominio}: Validación de resultados de clustering requiere expertos en mercado laboral tech LatAm. Pérdida de acceso a expertos puede paralizar validación cualitativa.
\end{itemize}

\subsubsection{Riesgos de Implementación}

\begin{itemize}
    \item \textbf{Curva de aprendizaje de tecnologías especializadas}: FAISS, UMAP, HDBSCAN son tecnologías avanzadas con documentación limitada en español. Configuración incorrecta puede generar resultados inválidos.

    \item \textbf{Limitaciones de tiempo del equipo}: Proyecto académico con 2 desarrolladores part-time. Implementación de Pipeline B (LLM) puede consumir tiempo asignado a análisis de resultados.
\end{itemize}

\subsection{Riesgos de Proyecto}

Estos riesgos corresponden a factores externos o limitaciones generales que pueden afectar el cumplimiento de objetivos académicos.

\subsubsection{Recursos Computacionales Limitados}

\begin{itemize}
    \item \textbf{GPU insuficiente para LLM}: Gemma 3 4B y Llama 3 3B requieren 3-6 GB VRAM (con cuantización Q4). Laptops académicos con GPUs integradas pueden ser insuficientes.

    \item \textbf{Almacenamiento limitado}: 600K ofertas con descripción completa + embeddings + clústeres puede requerir $>$50GB. Servidores universitarios con cuotas de almacenamiento pueden limitar corpus procesable.

    \item \textbf{Tiempo de cómputo para experimentos}: Cada iteración de ajuste de parámetros requiere re-ejecutar clustering completo (minutos/horas). Exploraciones extensivas de hiperparámetros pueden ser inviables.
\end{itemize}

\subsubsection{Acceso a Datos}

\begin{itemize}
    \item \textbf{Bloqueo de IPs por portales}: Scraping agresivo puede resultar en bloqueo permanente de IPs universitarias. Requiere proxies rotacionales (costo) o scraping throttled (meses de recolección).

    \item \textbf{Cambios legales en protección de datos}: Regulaciones futuras (ej. GDPR-like en LatAm) pueden prohibir scraping de ofertas laborales. Impacta viabilidad de recolección continua.

    \item \textbf{Desaparición de portales minoritarios}: Portales pequeños pueden cerrar operaciones. Impacta cobertura geográfica del análisis.
\end{itemize}

\subsubsection{Limitaciones de Alcance Académico}

\begin{itemize}
    \item \textbf{Imposibilidad de validar con usuarios reales}: Sistema académico no tiene acceso a reclutadores o candidatos para validar utilidad práctica de insights generados.

    \item \textbf{Horizonte temporal limitado}: Tesis debe completarse en 6-12 meses. Análisis de tendencias temporales idealmente requiere múltiples años de datos.

    \item \textbf{Restricciones de publicación académica}: Implementación de componentes innovadores puede ser necesaria para publicación en conferencias top-tier, pero excede alcance de tesis de pregrado.
\end{itemize}

\subsection{Matriz de Riesgos}

La Tabla \ref{tab:matriz-riesgos} resume los riesgos principales con su probabilidad, impacto y estrategia de mitigación.

\begin{table}[H]
\centering
\caption{Matriz de Riesgos del Proyecto}
\label{tab:matriz-riesgos}
\begin{tabular}{|p{4cm}|c|c|p{5cm}|}
\hline
\textbf{Riesgo} & \textbf{Prob.} & \textbf{Impacto} & \textbf{Mitigación} \\
\hline
Falsos positivos NER & Alta & Alto & Mejora de filtros post-extracción, validación manual de muestra \\
\hline
Latencia Pipeline B & Media & Alto & Procesamiento solo de subconjunto representativo (300 ofertas) \\
\hline
Pérdida de datos & Baja & Crítico & Backups automáticos diarios de PostgreSQL \\
\hline
Sesgo Gold Standard & Media & Alto & Revisión por múltiples anotadores, Cohen's Kappa $>$0.80 \\
\hline
GPU insuficiente & Media & Medio & Cuantización Q4, uso de Google Colab, modelos $<$4B parámetros \\
\hline
Bloqueo de IPs & Alta & Medio & Rate limiting conservador (1-2 req/s), user-agent rotation \\
\hline
\end{tabular}
\end{table}
