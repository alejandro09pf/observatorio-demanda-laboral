\chapter{CONTEXTO DEL PROBLEMA}

\section{Oportunidad y problema}

\subsection{Contexto del problema}

El mercado laboral en América Latina se encontró, durante la última década, en una compleja encrucijada definida por la confluencia de dos fuerzas a menudo contrapuestas: una acelerada transformación digital y la persistencia de desafíos estructurales, como una elevada informalidad laboral y brechas de capital humano \cite{echeverria2022}. La pandemia de COVID-19 actuó como un catalizador sin precedentes, intensificando la adopción de tecnologías y, con ello, la demanda de competencias digitales, al tiempo que exponía la vulnerabilidad de los mercados de trabajo de la región \cite{azuara2022}. Este dinamismo generó el riesgo de que la automatización y la digitalización, de no ser gestionadas estratégicamente, pudiesen exacerbar las desigualdades existentes, conduciendo a una mayor polarización y segmentación social \cite{echeverria2022}.

Para analizar este fenómeno regional de manera tangible y robusta, este proyecto seleccionó como casos de estudio a tres de las economías más grandes y digitalmente activas de habla hispana: Colombia, México y Argentina. La elección de estos países respondió a tres criterios estratégicos. Primero, su alto volumen de publicaciones de ofertas laborales en portales digitales aseguró la viabilidad de una recolección masiva de datos (web scraping), fundamental para el entrenamiento de modelos de lenguaje robustos \cite{aguilera2018, martinez2024, rubio2024}. Segundo, la existencia de estudios previos en cada país, aunque metodológicamente limitados, confirmó la pertinencia del problema y proporcionó una línea de base para la comparación \cite{cardenas2015}. Y tercero, su diversidad en términos de realidades económicas, territoriales y de madurez digital permitió validar que la solución desarrollada fuese portable y adaptable a los distintos contextos que caracterizan a América Latina.

\subsection{El caso de Colombia: cambio estructural en la demanda de habilidades}

El caso de Colombia sirvió como una ilustración profunda de esta dinámica. El diagnóstico nacional previo al proyecto ya indicaba que el principal cuello de botella para la inclusión digital no era la falta de infraestructura, sino la brecha de capital humano. Específicamente, el ``Índice de Brecha Digital'' (IBD) del Ministerio de Tecnologías de la Información y las Comunicaciones reveló que la dimensión de ``Habilidades Digitales'' constituía el mayor componente individual de la brecha en el país.

Esta evidencia fue posteriormente corroborada y cuantificada por el análisis empírico de la demanda laboral, el cual demostró que la pandemia generó un cambio estructural y persistente en el mercado. Se encontró que, en los 18 meses posteriores al inicio de la crisis sanitaria, las vacantes tecnológicas aumentaron en un 50\% en comparación con las no tecnológicas \cite{rubio2024}.

Este cambio no fue solo cuantitativo, sino también cualitativo: se observó una marcada caída en la demanda de herramientas ofimáticas tradicionales como Excel (cuya mención en ofertas cayó del 35.8\% en 2018 al 17.4\% en 2023) y un surgimiento exponencial de tecnologías especializadas asociadas al desarrollo web y la gestión de datos, como bases de datos NoSQL (12.3\%), el framework Django (5.5\%) y la librería React (5.3\%) para el año 2023 \cite{rubio2024}.

\subsection{Realidades en México y Argentina}

De manera similar, en México se ha documentado un desajuste significativo entre la oferta educativa y la demanda laboral en el sector tecnológico. El análisis de miles de ofertas laborales reveló que las empresas demandan perfiles con habilidades especializadas en desarrollo de software, análisis de datos y ciberseguridad, mientras que la oferta de graduados se concentra en áreas tradicionales de sistemas \cite{martinez2024}.

En Argentina, el sector de Tecnologías de la Información ha experimentado un crecimiento sostenido, convirtiéndose en uno de los principales generadores de empleo calificado. Sin embargo, la identificación de las tecnologías y roles más demandados ha sido un desafío constante para instituciones educativas y agencias de empleo \cite{aguilera2018}.

\subsection{Formulación del problema}

A pesar de que el contexto del problema ---la creciente e insatisfecha demanda de habilidades tecnológicas--- estaba claramente identificado, los métodos existentes en la región para analizarlo presentaban limitaciones metodológicas significativas que impedían una comprensión profunda y ágil del fenómeno.

Los estudios de referencia en los países seleccionados, si bien valiosos para establecer tendencias macro, se basaron en enfoques de análisis léxico y reglas manuales. En Colombia, el análisis se centró en un sistema de clasificación basado en la Clasificación Internacional Uniforme de Ocupaciones (CIUO), utilizando algoritmos de emparejamiento de texto con tokenización y métricas de similitud basadas en n-gramas \cite{rubio2024}. De forma análoga, en Argentina, los estudios se concentraron en técnicas de minería de texto con análisis de frecuencias y bigramas para identificar patrones en las ofertas del sector TI \cite{aguilera2018}. En México, el enfoque combinó datos de encuestas con scraping de portales, apoyándose en el análisis de frecuencia de términos y la creación de tipologías manuales para segmentar las habilidades \cite{martinez2024}.

La limitación fundamental compartida por estos enfoques es su dependencia de la correspondencia léxica explícita, lo que los hace incapaces de capturar la riqueza semántica del lenguaje. Estos métodos no podían:

\begin{itemize}
    \item Detectar habilidades implícitas (aquellas que se infieren del contexto de un cargo pero no se mencionan directamente)
    \item Gestionar la ambigüedad del lenguaje informal o el uso de anglicismos técnicos (``Spanglish'')
    \item Identificar clústeres de competencias emergentes que aún no forman parte de taxonomías estandarizadas
\end{itemize}

La alta variabilidad en la redacción de las ofertas laborales, la falta de estructuras normalizadas y la rápida aparición de nuevas tecnologías hacían que estos sistemas fueran metodológicamente frágiles y requirieran un constante mantenimiento manual \cite{echeverria2022}.

\subsection{Vacío identificado a nivel institucional}

Más allá de las limitaciones académicas individuales, esta carencia de infraestructura analítica ha sido reconocida a nivel institucional. El Banco Interamericano de Desarrollo (BID) ha señalado la falta de pipelines de análisis modernos y automatizados en la región, destacando que la mayoría de los sistemas existentes, si bien articulan el scraping, todavía se basan en reglas fijas o mapeos manuales y no han incorporado técnicas de embeddings ni de NLP avanzado \cite{echeverria2022}.

En consecuencia, el problema específico que este proyecto abordó fue la ausencia de una herramienta automatizada y de extremo a extremo que, adaptada a las particularidades lingüísticas y estructurales del español latinoamericano, permitiera superar las limitaciones de los análisis léxicos tradicionales. Se identificó la necesidad de un sistema capaz de extraer, estructurar y analizar la evolución de las habilidades tecnológicas de manera semántica, escalable y con un mayor grado de autonomía, integrando para ello técnicas avanzadas de Procesamiento de Lenguaje Natural (NLP), enriquecimiento contextual con Large Language Models (LLMs) y algoritmos de agrupamiento no supervisado.

\section{Propuesta de solución}

Para dar respuesta al problema formulado, se diseñó e implementó un observatorio de demanda laboral tecnológica basado en un pipeline modular y automatizado, un proyecto enmarcado en las áreas de Ingeniería de Sistemas y Ciencia de Datos. El sistema fue concebido como una solución de extremo a extremo que integró las etapas de recolección, procesamiento, análisis semántico y segmentación de ofertas de empleo publicadas en Colombia, México y Argentina. El objetivo fue crear una arquitectura robusta, replicable y adaptada a las complejidades del contexto latinoamericano, superando las limitaciones de los enfoques puramente léxicos o manuales.

La solución se materializó a través de un sistema compuesto por módulos secuenciales y cohesivos. El primer módulo consistió en un motor de adquisición de datos que, mediante técnicas de web scraping, extrajo de forma sistemática y ética decenas de miles de ofertas laborales de portales de empleo clave en la región. El núcleo del sistema fue su arquitectura de extracción dual, compuesta por dos pipelines paralelos:

\textbf{Pipeline A (Tradicional):} Implementó un método de extracción basado en Reconocimiento de Entidades Nombradas (NER) utilizando un EntityRuler de spaCy, poblado con la taxonomía completa de ESCO, combinado con expresiones regulares para capturar un baseline de habilidades explícitas de alta precisión.

\textbf{Pipeline B (Basado en LLMs):} Empleó Large Language Models (LLMs) como Llama 3 para realizar una extracción semántica, capaz de identificar no solo habilidades explícitas sino también de inferir competencias implícitas a partir del contexto de la vacante.

Posteriormente, un módulo de mapeo de dos capas normalizó las habilidades extraídas por ambos pipelines contra la taxonomía ESCO. La primera capa realizó una coincidencia léxica (exacta y difusa), mientras que la segunda ejecutó una búsqueda de similitud semántica de alto rendimiento, utilizando embeddings multilingües (E5) y un índice FAISS pre-calculado. Finalmente, un módulo de análisis no supervisado aplicó una secuencia metodológica de embeddings, reducción de dimensionalidad con UMAP y agrupamiento con HDBSCAN para identificar clústeres de habilidades y perfiles emergentes.

\section{Justificación de la solución}

La solución implementada se justificó como una alternativa superior y mejor adaptada para el análisis de la demanda de habilidades en América Latina, ya que abordó directamente las debilidades metodológicas identificadas en los estudios previos. A diferencia de los enfoques basados exclusivamente en reglas léxicas \cite{aguilera2018, rubio2024}, la arquitectura de dos pipelines paralelos permitió una validación empírica cruzada: combinó la auditabilidad y alta precisión para habilidades conocidas del Pipeline A con la potencia inferencial y la capacidad de descubrir habilidades implícitas del Pipeline B.

Este diseño comparativo proveyó un marco para evaluar objetivamente el rendimiento de los LLMs, en lugar de depender únicamente de su capacidad ``black-box''.

Técnicamente, el sistema representó un avance significativo en escalabilidad y eficiencia. La implementación de un índice FAISS para la búsqueda semántica de similitud permitió procesar grandes volúmenes de datos a una velocidad órdenes de magnitud superior a las búsquedas en bases de datos vectoriales convencionales, haciendo factible el análisis de todo el corpus recolectado.

Adicionalmente, el sistema fue diseñado explícitamente para la realidad del español latinoamericano. Este enfoque abordó directamente una limitación crítica de trabajos de vanguardia en LLMs, los cuales se han desarrollado y validado casi exclusivamente sobre datasets en inglés, ignorando las particularidades lingüísticas (como el ``Spanglish'') del dominio tecnológico en la región.

Finalmente, el valor agregado del proyecto residió en su síntesis estratégica de metodologías de vanguardia. El sistema no se limitó a una sola técnica, sino que articuló la cobertura del scraping regional, la potencia de los LLMs ajustados para generar salidas estructuradas, y la capacidad estructuradora del clustering semántico. Al hacerlo, se desarrolló un observatorio más completo, robusto y metodológicamente transparente que las alternativas existentes, estableciendo una base sólida y replicable para el monitoreo dinámico de la demanda laboral en la región.

\section{Descripción del proyecto}

El proyecto se concibió como un observatorio automatizado para capturar, normalizar y analizar avisos de empleo en Latinoamérica. Se integraron múltiples portales (CO, MX y AR), se diseñó una base de datos relacional con soporte vectorial, y se implementó un pipeline de extracción de habilidades (NER/regex/LLM) alineadas a ESCO, con generación de indicadores, visualizaciones y reportes. Operativamente, se planificó escalar hasta 600.000 avisos para la defensa, garantizando calidad, trazabilidad y reproducibilidad.

\subsection{Objetivo general}

Desarrollar un sistema que permita procesar y segmentar la demanda de habilidades tecnológicas en Colombia, México y Argentina, mediante técnicas de procesamiento de lenguaje natural.

\subsection{Objetivos específicos}

\begin{itemize}
    \item Construir un estado del arte exhaustivo para comparar trabajos existentes en el ámbito de observatorios laborales automatizados y técnicas de procesamiento de lenguaje natural en español.

    \item Diseñar una arquitectura modular, escalable y reutilizable para el observatorio laboral automatizado, fundamentada en las mejores prácticas identificadas en el estado del arte.

    \item Implementar e integrar técnicas de inteligencia artificial para la identificación, normalización y agrupación semántica de habilidades tecnológicas en ofertas laborales en español.

    \item Validar el desempeño y la robustez de la arquitectura y los modelos propuestos mediante métricas cuantitativas y estudios empíricos.
\end{itemize}

\subsection{Entregables, estándares y justificación}

\begin{longtable}{|p{5cm}|p{5cm}|p{5cm}|}
\hline
\textbf{Entregable} & \textbf{Estándares asociados} & \textbf{Justificación} \\
\hline
\endfirsthead

\hline
\textbf{Entregable} & \textbf{Estándares asociados} & \textbf{Justificación} \\
\hline
\endhead

\hline
\endfoot

\hline
\endlastfoot

Repositorio de código (spiders, orquestador, pipelines) & PEP 8/257/484; Conv. Commits; SemVer & Mantenibilidad, legibilidad y control de versiones. \\
\hline

Esquema BD y migraciones (PostgreSQL + pgvector) & Normalización (3NF); SQL best practices & Integridad, trazabilidad y soporte a consultas vectoriales. \\
\hline

Spiders y configuración de scraping & Polite crawling (delays/retries); manejo anti-bots & Captura estable a escala y resiliencia ante cambios UI. \\
\hline

Orquestador CLI + scheduler & CLI UX (Typer); jobs idempotentes & Operación reproducible, programable y auditable. \\
\hline

Módulo de extracción/normalización de habilidades & ISO/IEC/IEEE 29148 (requisitos); ESCO & Consistencia semántica y comparabilidad entre países. \\
\hline

Embeddings y análisis (E5, UMAP, HDBSCAN) & Procedimientos reproducibles; semillas fijas & Descubrimiento de patrones y replicabilidad experimental. \\
\hline

Datasets consolidados (CSV/JSON) + diccionario de datos & Esquemas declarativos; control de versiones & Consumo externo y verificación de calidad. \\
\hline

Documentación técnica y de proyecto (SRS, SPMP, VFP, manuales) & IEEE 1058 (plan de proyecto); 29148 (requisitos) & Alineación con buenas prácticas y transferencia de conocimiento. \\
\hline

Reportes y visualizaciones (PDF/PNG/CSV) & Principios de visualización; metadatos & Comunicación clara de hallazgos a públicos no técnicos. \\
\hline

Plan de operación y mantenimiento (Docker/monitoring) & Buenas prácticas Docker/Logging & Despliegue consistente y observabilidad del sistema. \\
\hline

\end{longtable}
