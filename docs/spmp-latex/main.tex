% ============================================================================
% PLAN DE ADMINISTRACIÓN DE PROYECTO (SPMP)
% Observatorio de Demanda Laboral en Tecnología en Latinoamérica
% ============================================================================

\documentclass[11pt,oneside,letterpaper]{report}

% ============================================================================
% PAQUETES
% ============================================================================
\usepackage[utf8]{inputenc}
\usepackage[spanish,es-tabla]{babel}
\usepackage[letterpaper,top=3cm,bottom=3cm,left=3cm,right=3cm]{geometry}
\usepackage{times}
\usepackage{graphicx}
\usepackage{amsmath,amssymb}
\usepackage{setspace}
\usepackage{fancyhdr}
\usepackage{titlesec}
\usepackage{tocloft}
\usepackage[hidelinks]{hyperref}
\usepackage{listings}
\usepackage{xcolor}
\usepackage{float}
\usepackage{longtable}
\usepackage{multirow}
\usepackage{array}
\usepackage{booktabs}
\usepackage[backend=biber,style=ieee,citestyle=numeric-comp,sorting=none]{biblatex}

% TikZ para diagramas
\usepackage{tikz}
\usetikzlibrary{shapes.geometric, arrows.meta, positioning, shadows, fit, shapes.multipart}

% ============================================================================
% CONFIGURACIÓN DE BIBLIOGRAFÍA
% ============================================================================
\addbibresource{bibliografia.bib}

% ============================================================================
% CONFIGURACIÓN DE CÓDIGO
% ============================================================================
\lstset{
    basicstyle=\ttfamily\small,
    breaklines=true,
    frame=single,
    numbers=left,
    numberstyle=\tiny\color{gray},
    keywordstyle=\color{blue},
    commentstyle=\color{green!60!black},
    stringstyle=\color{orange},
    showstringspaces=false
}

% ============================================================================
% CONFIGURACIÓN DE INTERLINEADO
% ============================================================================
\onehalfspacing

% ============================================================================
% CONFIGURACIÓN DE ENCABEZADOS Y PIE DE PÁGINA
% ============================================================================
\pagestyle{fancy}
\fancyhf{}
\fancyhead[L]{Pontificia Universidad Javeriana}
\fancyhead[R]{Plan de Administración de Proyecto}
\fancyfoot[R]{Página \thepage}
\renewcommand{\headrulewidth}{0.4pt}
\renewcommand{\footrulewidth}{0.4pt}

\fancypagestyle{plain}{%
  \fancyhf{}%
  \fancyhead[L]{Pontificia Universidad Javeriana}
  \fancyhead[R]{Plan de Administración de Proyecto}
  \fancyfoot[R]{Página \thepage}
  \renewcommand{\headrulewidth}{0.4pt}
  \renewcommand{\footrulewidth}{0.4pt}
}

% ============================================================================
% CONFIGURACIÓN DE TÍTULOS
% ============================================================================
\titleformat{\chapter}[display]
{\normalfont\Large\bfseries\centering}
{}{0pt}{\Large}

\titlespacing*{\chapter}{0pt}{20pt}{20pt}

\titleformat{\section}
{\normalfont\large\bfseries}{\thesection}{1em}{}

\titleformat{\subsection}
{\normalfont\normalsize\bfseries}{\thesubsection}{1em}{}

\titleformat{\subsubsection}
{\normalfont\normalsize\bfseries}{\thesubsubsection}{1em}{}

% ============================================================================
% INFORMACIÓN DEL DOCUMENTO
% ============================================================================
\newcommand{\proyectoTitulo}{Observatorio de Demanda Laboral en Tecnología en Latinoamérica}
\newcommand{\estudianteUno}{Nicolas Camacho Alarcón}
\newcommand{\cedulaUno}{cc. 1000942178}
\newcommand{\celularUno}{3175714599}
\newcommand{\correoUno}{camachoa.nicolas@javeriana.edu.co}
\newcommand{\estudianteDos}{Alejandro Pinzón Fajardo}
\newcommand{\cedulaDos}{cc. 1052411260}
\newcommand{\celularDos}{3115369454}
\newcommand{\correoDos}{alejandro\_pinzon@javeriana.edu.co}
\newcommand{\director}{Ing. Luis Gabriel Moreno Sandoval}
\newcommand{\cedulaDirector}{cc.}
\newcommand{\celularDirector}{3016278993}
\newcommand{\correoDirector}{morenoluis@javeriana.edu.co}
\newcommand{\trabajoDirector}{Pontificia Universidad Javeriana; Profesor Temporal Departamento de Sistemas}
\newcommand{\anio}{2025}
\newcommand{\mes}{Noviembre}
\newcommand{\fecha}{Noviembre de 2025}

% ============================================================================
% DOCUMENTO
% ============================================================================
\begin{document}

% ============================================================================
% PORTADA
% ============================================================================
% ============================================================================
% PORTADA - SRS
% Especificación de Requerimientos de Software
% ============================================================================

\begin{titlepage}
\centering

\vspace*{1cm}

{\Large\bfseries \proyectoTitulo\par}
\vspace{0.5cm}
{[Grupo 8]\par}

\vspace{2cm}

{\Large\bfseries ESPECIFICACIÓN DE REQUERIMIENTOS DE SOFTWARE\par}

\vspace{1cm}

{[\fecha]\par}
\vspace{0.5cm}


\vspace{2cm}

\includegraphics[width=0.4\textwidth]{logo-javeriana.png}

\vfill

{\large Autores:\par}
\vspace{0.5cm}
{\large \autorUno\par}
{\large \autorDos\par}

\vspace{1cm}

{\large Pontificia Universidad Javeriana\par}
{\large Facultad de Ingeniería\par}
{\large Bogotá, Colombia\par}
{\large \mes{} de \anio\par}

\end{titlepage}


% ============================================================================
% PREFACIO
% ============================================================================
\chapter*{Prefacio}
\addcontentsline{toc}{chapter}{Prefacio}

Este documento tiene como objetivo presentar el plan de gestión del proyecto titulado ``Observatorio Automatizado de Demanda Laboral en Tecnología para Latinoamérica'', con el fin de que el lector determine si le interesa profundizar en su contenido. En él se describen los objetivos, alcance, entregables, cronograma, riesgos y herramientas del proyecto, con un enfoque académico y técnico. Está dirigido principalmente a los estudiantes integrantes del equipo, al director del trabajo, a los docentes evaluadores y a cualquier interesado en comprender la planificación de un proyecto de software de mediano alcance con componentes de inteligencia artificial, procesamiento de lenguaje natural y scraping de datos en contexto latinoamericano.

% ============================================================================
% HISTORIAL DE CAMBIOS
% ============================================================================
\chapter*{Historial de Cambios}
\addcontentsline{toc}{chapter}{Historial de Cambios}

\begin{table}[H]
\centering
\small
\begin{tabular}{|c|p{2cm}|p{3.5cm}|p{6cm}|p{2.5cm}|}
\hline
\textbf{Versión} & \textbf{Fecha} & \textbf{Autor(es)} & \textbf{Descripción del cambio} & \textbf{Sección afectada} \\
\hline
0.1 & 2025-06-05 & Nicolás Camacho, Alejandro Pinzón & Creación inicial del documento con estructura base & Todas \\
\hline
0.2 & 2025-07-01 & Alejandro Pinzón & Actualización del cronograma y ajuste de estimaciones de esfuerzo & Cap. 4 - Administración \\
\hline
0.3 & 2025-07-20 & Nicolás Camacho & Expansión del análisis de alternativas tecnológicas para scraping y NLP & Cap. 3 - Contexto \\
\hline
0.4 & 2025-08-05 & Alejandro Pinzón & Incorporación de glosario técnico completo con 29 términos y referencias bibliográficas & Glosario y Bibliografía \\
\hline
0.5 & 2025-08-25 & Nicolás Camacho, Alejandro Pinzón & Detalle de criterios de aceptación cuantitativos por fase & Cap. 3 - Plan de Aceptación \\
\hline
0.6 & 2025-09-15 & Alejandro Pinzón & Inclusión de diagramas BPMN para procesos de administración, calidad y riesgos & Cap. 5, 6 \\
\hline
1.0 & 2025-11-15 & Nicolás Camacho, Alejandro Pinzón & Versión final aprobada para entrega formal & Todas \\
\hline
\end{tabular}
\caption{Historial de versiones del documento SPMP}
\end{table}

\textbf{Nota:} Este historial refleja las versiones principales del documento durante su desarrollo. Cambios menores de formato, corrección de errores tipográficos y ajustes de redacción no se registran individualmente. El control de versiones detallado del código y documentación técnica se gestiona mediante Git y GitHub según lo descrito en el Capítulo 6.

% ============================================================================
% TABLA DE CONTENIDOS
% ============================================================================
\renewcommand{\contentsname}{CONTENIDO}
\tableofcontents
\newpage

% ============================================================================
% LISTA DE TABLAS
% ============================================================================
\renewcommand{\listtablename}{LISTA DE TABLAS}
\listoftables
\newpage

% ============================================================================
% LISTA DE FIGURAS
% ============================================================================
\renewcommand{\listfigurename}{LISTA DE FIGURAS}
\listoffigures
\newpage

% ============================================================================
% CAPÍTULOS
% ============================================================================
\chapter{Vista General del Proyecto}

\section{Visión del Producto}

El presente proyecto busca desarrollar un sistema semiautomatizado, ejecutable de forma periódica y local, que permita procesar y segmentar la demanda de habilidades tecnológicas en Colombia, México y Argentina mediante un pipeline de scraping, procesamiento semántico y visualización macro. El sistema permitirá generar insumos estructurados para el posterior análisis de tendencias laborales en el sector tecnológico latinoamericano, fortaleciendo la toma de decisiones por parte de instituciones educativas, investigadores y organismos interesados en cerrar brechas entre la formación y el mercado laboral.

Se espera que, una vez completado, el sistema siente las bases para futuras ampliaciones en cobertura territorial, frecuencia de actualización, refinamiento técnico y despliegue institucional, permitiendo construir un observatorio laboral replicable, contextualizado al español latinoamericano y alineado a estándares internacionales como ESCO o CIUO.

\section{Propósito, Alcance y Objetivos}

\subsection{Propósito}

Este proyecto se realiza con el fin de sentar las bases técnicas y metodológicas para un observatorio de demanda laboral tecnológica en Latinoamérica, integrando en un solo sistema fases de extracción, procesamiento y visualización de habilidades demandadas en vacantes reales en línea, con especial enfoque en el idioma español y el contexto regional.

\subsection{Alcance}

El proyecto contempla la implementación de un pipeline local compuesto por las siguientes etapas: (1) scraping de portales de empleo (Computrabajo, elempleo.com, Bumeran), (2) extracción de habilidades mediante NER y regex, (3) enriquecimiento semántico con LLMs, (4) representación vectorial con embeddings multilingües, (5) clustering con UMAP y HDBSCAN, y (6) generación de visualizaciones macro estáticas.

Quedan fuera del alcance: el desarrollo de dashboards interactivos, la construcción de portales web, la automatización continua del pipeline y la integración con bases de datos externas o APIs privadas.

\subsection{Objetivo general}

Desarrollar un sistema que permita procesar y segmentar la demanda de habilidades tecnológicas en Colombia, México y Argentina, mediante técnicas de procesamiento de lenguaje natural.

\subsection{Objetivos Específicos}

\begin{itemize}
    \item Construir un estado del arte exhaustivo para comparar trabajos existentes en el ámbito de observatorios laborales automatizados y técnicas de procesamiento de lenguaje natural en español.

    \item Diseñar una arquitectura modular, escalable y reutilizable para el observatorio laboral automatizado, fundamentada en las mejores prácticas identificadas en el estado del arte.

    \item Implementar e integrar técnicas de inteligencia artificial para la identificación, normalización y agrupación semántica de habilidades tecnológicas en ofertas laborales en español.

    \item Validar el desempeño y la robustez de la arquitectura y los modelos propuestos mediante métricas cuantitativas y estudios empíricos.
\end{itemize}

\section{Supuestos y Restricciones}

\subsection{Supuestos}

\begin{itemize}
    \item Se mantendrá acceso continuo a portales de empleo y/o APIs para extracción de vacantes.
    \item Existirá un corpus suficiente y variado de vacantes en español.
    \item Se contará con infraestructura local adecuada para ejecutar las tareas técnicas.
    \item El equipo mantendrá una coordinación fluida durante las semanas de ejecución.
    \item Se contará con modelos a utilizar que tengan el funcionamiento adecuado en español.
\end{itemize}

\subsection{Restricciones}

\begin{itemize}
    \item Existencia de carga académica paralela por parte del equipo.
    \item Capacidad limitada de procesamiento local (sin uso de GPUs o servidores externos).
    \item Tiempo acotado para ajuste fino de prompts, clustering y visualizaciones.
\end{itemize}

\section{Entregables}

\begin{table}[H]
\centering
\begin{tabular}{|p{4cm}|p{4.5cm}|p{3cm}|p{2.5cm}|}
\hline
\textbf{Entregable} & \textbf{Descripción} & \textbf{Destinatario} & \textbf{Fecha estimada} \\
\hline
Repositorio funcional del sistema & Código completo de scraping, NER, LLMs, clustering y visualización & Docentes evaluadores & Semana 14 \\
\hline
Dataset limpio de vacantes y habilidades & Base de datos estructurada con datos procesados & Equipo y docentes & Semana 13 \\
\hline
Diccionario de habilidades y embeddings & Archivo con habilidades extraídas, enriquecidas y vectorizadas & Equipo técnico & Semana 10 \\
\hline
Visualizaciones macro & Gráficos estáticos de resultados para validación cualitativa & Asesor y evaluadores & Semana 12 \\
\hline
Documento explicativo del sistema & Guía metodológica, arquitectura y funcionamiento & Universidad / Archivo final & Semana 15 \\
\hline
Pruebas funcionales y logs & Evidencia de validación de módulos y resultados intermedios & Asesor y docentes & Semana 14 \\
\hline
\end{tabular}
\caption{Entregables del Proyecto}
\end{table}

\section{Resumen de Calendarización y Presupuesto}

\begin{table}[H]
\centering
\begin{tabular}{|l|l|c|}
\hline
\textbf{Fase} & \textbf{Actividad} & \textbf{Semanas} \\
\hline
F1 & Diseño y arquitectura del sistema & 1--2 \\
\hline
F2 & Scraping y carga a base de datos & 3--5 \\
\hline
F3 & Extracción de habilidades (NER + regex) & 6--7 \\
\hline
F4 & Enriquecimiento con LLMs & 8--9 \\
\hline
F5 & Embeddings + clustering & 10--11 \\
\hline
F6 & Visualización macro y evaluación & 12 \\
\hline
F7 & Pruebas, documentación y entrega final & 13--16 \\
\hline
\end{tabular}
\caption{Fases del Proyecto}
\end{table}

\begin{table}[H]
\centering
\begin{tabular}{|l|l|l|}
\hline
\textbf{Recurso} & \textbf{Descripción} & \textbf{Costo} \\
\hline
Infraestructura local & Uso de computadoras personales & Sin costo adicional \\
\hline
Modelos preentrenados & HuggingFace, spaCy, BETO, etc. & Gratuito (open source) \\
\hline
API de LLMs & GPT-3.5 para pruebas (en caso de acceso) & Versión gratuita / académica \\
\hline
Licencias y software & VSCode, GitHub, Dash, PostgreSQL & Gratuito \\
\hline
Herramientas de documentación & Google Docs, Overleaf, GitHub Wiki & Gratuito \\
\hline
\end{tabular}
\caption{Ítems Consolidados}
\end{table}

\section{Evolución del Plan}

El plan seguirá una lógica iterativa inspirada en CRISP-DM y Scrum no estricto. Cualquier cambio al plan será discutido en reuniones semanales y validado por consenso. Las actualizaciones serán registradas en Google Docs y GitHub, y se comunicarán al equipo con antelación. Las decisiones sobre cambios mayores (como rediseño de etapas o reemplazo de técnicas) deberán estar documentadas en el acta de revisión de fase correspondiente.

\chapter{Glosario}

\textbf{1. Portales de empleo}

Son plataformas web donde empresas publican vacantes laborales y profesionales buscan oportunidades. En este proyecto se consideran fuentes como LinkedIn, Computrabajo, Bumeran, ZonaJobs e Indeed, que constituyen insumos primarios para los procesos de scraping y análisis \cite{aguilera2018,cardenas2015}.

\textbf{2. Web Scraping}

Técnica de recolección automatizada de datos desde páginas web, utilizando librerías como BeautifulSoup, Selenium o Playwright. Permite extraer de forma estructurada información relevante de las ofertas publicadas \cite{orozco2019}.

\textbf{3. Oferta laboral}

Se refiere al anuncio publicado por una organización donde se describe el perfil buscado, incluyendo título del cargo, funciones, requisitos y habilidades deseadas \cite{rubio2024}.

\textbf{4. Base de datos relacional (PostgreSQL)}

Sistema que organiza los datos recolectados en tablas interconectadas, facilitando su consulta, limpieza y posterior análisis mediante estructuras SQL \cite{martinez2024}.

\textbf{5. Normalización de datos}

Proceso de limpieza, estandarización y unificación de formatos para reducir ambigüedad, errores y duplicados, y mejorar la coherencia del análisis posterior \cite{campos2024}.

\section*{Procesamiento de texto y extracción de habilidades}

\textbf{6. Expresiones regulares (Regex)}

Lenguaje sintáctico utilizado para identificar y extraer patrones textuales específicos (como frases que contengan habilidades o requisitos) en grandes volúmenes de texto \cite{lukauskas2023}.

\textbf{7. Named Entity Recognition (NER)}

Técnica de procesamiento de lenguaje natural (NLP) que identifica y clasifica entidades en un texto, como nombres de habilidades, empresas o tecnologías \cite{nguyen2024}.

\textbf{8. Tokenización}

Consiste en dividir un texto en unidades mínimas llamadas ``tokens'' (palabras, signos u oraciones), facilitando el análisis lingüístico automatizado \cite{nguyen2024}.

\textbf{9. Lematización}

Proceso que transforma las palabras a su forma canónica o raíz gramatical, permitiendo uniformar variaciones morfológicas del lenguaje \cite{echeverria2022}.

\textbf{10. Stopwords}

Términos frecuentes sin valor informativo (como ``de'', ``por'', ``la''), comúnmente eliminados en tareas de procesamiento textual \cite{nguyen2024}.

\textbf{11. Co-ocurrencia}

Medida estadística que indica la frecuencia con que dos o más términos aparecen juntos en un texto, útil para detectar relaciones semánticas \cite{campos2024}.

\textbf{12. Bigramas y trigramas}

Secuencias de dos o tres palabras consecutivas utilizadas para capturar patrones de lenguaje más complejos que las palabras individuales \cite{aguilera2018}.

\section*{Modelado con LLMs y enriquecimiento semántico}

\textbf{13. LLM (Large Language Models)}

Modelos de lenguaje de gran escala (como GPT o T5) entrenados sobre corpus masivos, capaces de generar texto, extraer conocimiento implícito y realizar razonamiento contextualizado \cite{nguyen2024,razumovskaia2024}.

\textbf{14. Prompt Engineering}

Diseño estratégico de instrucciones o ejemplos para guiar la salida de un LLM, crucial en tareas de extracción de habilidades o clasificación de ocupaciones \cite{razumovskaia2024}.

\textbf{15. Few-shot learning}

Habilidad de los LLMs para realizar tareas complejas con pocos ejemplos, lo cual resulta clave cuando se carece de datasets etiquetados masivamente en español \cite{nguyen2024}.

\textbf{16. Chain-of-Thought Reasoning (CoT)}

Técnica que induce a los modelos a razonar paso a paso, mejorando precisión en tareas como clasificación y desambiguación semántica \cite{razumovskaia2024}.

\textbf{17. Infer-Retrieve-Rank (IRR)}

Enfoque que primero infiere una entidad, luego recupera candidatos posibles, y finalmente los rankea con base en relevancia, utilizado para seleccionar habilidades o clasificar ocupaciones \cite{lopez2025}.

\textbf{18. Habilidades explícitas vs implícitas}

Las primeras están textualmente expresadas (``manejo de Python''), mientras que las segundas deben inferirse por contexto (``implementación de modelos supervisados'') \cite{nguyen2024}.

\section*{Representación vectorial y análisis semántico}

\textbf{19. Embeddings semánticos}

Representaciones numéricas de textos que capturan similitudes semánticas, permitiendo análisis cuantitativos y clustering. Ejemplos incluyen word2vec, BERT y E5 \cite{kavas2025,vasquez2024}.

\textbf{20. Embeddings multilingües}

Vectores entrenados para representar texto en múltiples idiomas en un mismo espacio semántico. Son esenciales para manejar contenido mixto español-inglés en ofertas laborales \cite{echeverria2022,razumovskaia2024}.

\textbf{21. Modelos de lenguaje en español}

Incluyen variantes como BETO, MarIA, T5-español, que han sido entrenadas en corpus hispanos y se adaptan mejor a tareas de extracción en este idioma \cite{nguyen2024}.

\textbf{22. Espacio vectorial}

Marco matemático donde entidades como palabras, frases o documentos son representadas como vectores en un espacio multidimensional \cite{kavas2025}.

\textbf{23. Reducción de dimensionalidad (UMAP)}

Técnica que transforma espacios de alta dimensionalidad en representaciones más simples, conservando la estructura semántica subyacente para facilitar análisis y visualización \cite{lukauskas2023}.

\section*{Segmentación y visualización}

\textbf{24. Clustering (HDBSCAN)}

Algoritmo no supervisado que detecta grupos naturales de observaciones (como habilidades o perfiles laborales) según su similitud semántica, sin requerir número de clusters predefinido \cite{lukauskas2023}.

\textbf{25. Evaluación por coherencia semántica}

Métrica que mide qué tan bien están agrupadas las instancias similares dentro de un modelo, clave para validar la efectividad del clustering \cite{vasquez2024}.

\textbf{26. Silhouette Score}

Indicador que evalúa la calidad de los clusters considerando qué tan cohesionados y separados están entre sí \cite{lukauskas2023}.

\textbf{27. Visualización de datos}

Proceso de representar información compleja en formatos gráficos o interactivos que permiten interpretar resultados, comunicar hallazgos y apoyar decisiones \cite{rubio2024}.

\textbf{28. Python}

Lenguaje de programación ampliamente utilizado en ciencia de datos y NLP, por su sintaxis sencilla y librerías especializadas como scikit-learn, spaCy, transformers y pandas \cite{nguyen2024}.

\textbf{29. Taxonomía de habilidades (ESCO, CIUO-08, O*NET)}

Sistemas jerárquicos y normalizados de clasificación de habilidades y ocupaciones, fundamentales para anclar el análisis a estándares internacionales y mejorar interoperabilidad de los resultados \cite{cardenas2015,echeverria2022}.

\chapter{Contexto del proyecto}

\section{Modelo de Ciclo de Vida}

El proyecto ``Observatorio Automatizado de Demanda Laboral en Tecnología para Latinoamérica'' adopta un enfoque metodológico mixto que combina elementos del modelo CRISP-DM con prácticas ágiles inspiradas en Scrum, adaptadas a un entorno académico. Esta combinación busca asegurar tanto una estructura clara y validada como una flexibilidad en la ejecución iterativa del desarrollo.

El modelo CRISP-DM (Cross-Industry Standard Process for Data Mining) proporciona una base sólida para proyectos de análisis de datos, al organizar el trabajo en fases encadenadas y recurrentes. Su estructura es especialmente adecuada para proyectos que incluyen scraping, procesamiento textual, análisis semántico y generación de visualizaciones, como es el caso de este sistema.

El ciclo de vida del proyecto se compone de seis fases principales encadenadas y recurrentes. La primera fase corresponde a la comprensión del dominio y diseño del sistema, incluyendo revisión crítica del estado del arte, definición del pipeline modular, y planificación por etapas metodológicas. La segunda fase implementa la extracción de datos mediante scraping con spiders personalizados por portal y país, almacenando resultados de forma estructurada en base de datos relacional.

La tercera fase ejecuta procesamiento semántico y enriquecimiento, aplicando NER, expresiones regulares y validación con LLMs para extracción explícita e implícita de habilidades técnicas. La cuarta fase realiza vectorización y clustering, obteniendo embeddings semánticos, reducción de dimensionalidad con UMAP y agrupación de perfiles mediante HDBSCAN. La quinta fase genera visualización y validación, produciendo gráficos estáticos para evaluación cualitativa y cuantitativa por expertos del dominio. Finalmente, la sexta fase consolida documentación y empaquetado final, integrando guía metodológica, código versionado, resultados obtenidos y recomendaciones para futuras iteraciones.

Cada una de estas fases se articula mediante entregables intermedios verificables, como bases de datos limpias, corpus anotados, scripts funcionales y reportes interpretables.

\begin{figure}[H]
\centering
\includegraphics[width=0.95\textwidth]{figures/CiclodeVidadelProyecto.png}
\caption{Ciclo de Vida del Proyecto basado en CRISP-DM con prácticas ágiles}
\label{fig:ciclo-vida}
\end{figure}

\subsection{Análisis de Alternativas y Justificación}

En el contexto del desarrollo del sistema ``Observatorio Automatizado de Demanda Laboral en Tecnología para Latinoamérica'', se evaluaron distintas alternativas de modelos de ciclo de vida y enfoques metodológicos. A continuación, se analizan tres enfoques representativos: cascada tradicional, metodología ágil (Scrum completo), y CRISP-DM adaptado con prácticas ágiles.

\subsubsection{Modelo de Cascada Tradicional}

El modelo de cascada propone una secuencia rígida de etapas: análisis de requerimientos, diseño, implementación, pruebas y despliegue. Su lógica lineal facilita la planificación y la documentación desde el inicio, siendo útil en proyectos con requerimientos estables y completamente definidos.

\textbf{Ventajas:}
\begin{itemize}
    \item Claridad en los entregables por fase.
    \item Control estricto del avance y dependencias.
    \item Buena trazabilidad documental.
\end{itemize}

\textbf{Limitaciones:}
\begin{itemize}
    \item Supone requerimientos estables, lo cual no aplica a este proyecto, donde los resultados intermedios pueden modificar fases posteriores.
    \item No permite incorporar descubrimientos progresivos ni resultados exploratorios.
    \item La validación se da muy tarde, sin retroalimentación temprana.
\end{itemize}

\textbf{Conclusión:} Debido a la naturaleza exploratoria, técnica y adaptable del proyecto, el modelo de cascada resulta inadecuado.

\subsubsection{Scrum completo (ágil puro)}

Scrum es un marco ágil iterativo que organiza el trabajo en sprints, con roles definidos y eventos regulares. Favorece la adaptación continua y la entrega incremental de valor.

\textbf{Ventajas:}
\begin{itemize}
    \item Flexibilidad ante cambios y descubrimientos técnicos.
    \item Retroalimentación constante.
    \item Priorización de entregables más relevantes en cada ciclo.
\end{itemize}

\textbf{Limitaciones:}
\begin{itemize}
    \item Supone la existencia de un cliente activo y disponible para validar cada sprint, lo cual no aplica a un proyecto académico sin cliente externo.
    \item En su forma pura, Scrum puede fragmentar demasiado procesos que requieren una visión de pipeline.
    \item Requiere una madurez organizacional y disciplina de roles que excede el alcance del equipo.
\end{itemize}

\textbf{Conclusión:} Si bien Scrum ofrece beneficios para la iteración y mejora continua, su estructura estricta no se ajusta del todo al carácter investigativo y académico del presente proyecto.

\subsubsection{CRISP-DM adaptado + prácticas ágiles (modelo elegido)}

El modelo CRISP-DM fue diseñado para proyectos de minería de datos y análisis avanzado. Sus fases se ajustan de forma natural a un proyecto basado en scraping, NLP, embeddings y clustering. La combinación con prácticas ligeras de Scrum permite mantener orden sin renunciar a la flexibilidad.

\textbf{Ventajas:}
\begin{itemize}
    \item Alineación directa con las etapas técnicas del proyecto.
    \item Permite iteración interna por módulo o fase.
    \item No requiere cliente externo constante.
    \item Favorece la documentación, trazabilidad y replicabilidad.
    \item Facilita la planeación modular, con entregables verificables en cada fase.
\end{itemize}

\textbf{Limitaciones:}
\begin{itemize}
    \item No cubre explícitamente aspectos de comunicación o roles.
    \item Requiere una adaptación cuidadosa al contexto académico.
\end{itemize}

\textbf{Justificación final:}

Se adopta un modelo híbrido fundamentado en CRISP-DM como columna vertebral del flujo de trabajo, complementado con prácticas ágiles inspiradas en Scrum para planificación, validación y control de avances.

\section{Análisis de Alternativas Tecnológicas}

En complemento al análisis metodológico, se presenta a continuación un estudio comparativo de las principales alternativas tecnológicas evaluadas para cada componente crítico del sistema, con justificación de las decisiones finales adoptadas.

\subsection{Herramientas de Web Scraping}

Se evaluaron tres alternativas de herramientas para scraping de portales de empleo, cada una con características diferenciadas según el tipo de contenido web a procesar. La Tabla \ref{tab:comparacion-scraping} presenta la comparación sistemática de estas herramientas considerando descripción, ventajas principales y limitaciones técnicas.

\begin{table}[H]
\centering
\caption{Comparación de Herramientas de Web Scraping}
\label{tab:comparacion-scraping}
\small
\begin{tabular}{|p{2.5cm}|p{4.5cm}|p{4cm}|p{3.5cm}|}
\hline
\textbf{Herramienta} & \textbf{Descripción} & \textbf{Ventajas} & \textbf{Limitaciones} \\
\hline
Scrapy & Framework asíncrono en Python para extracción a gran escala con arquitectura modular basada en spiders & Alto rendimiento vía ejecución asíncrona (Twisted), gestión automática de concurrencia/reintentos, exportación nativa JSON/CSV/SQL, comunidad activa & No ejecuta JavaScript nativamente, curva aprendizaje moderada \\
\hline
Selenium & Herramienta de automatización de navegadores para contenido dinámico JavaScript & Renderiza JavaScript completamente, simulación de interacciones humanas (clicks/scroll), compatible multi-navegador & Significativamente más lento que Scrapy, alto consumo CPU/memoria, gestión explícita de drivers \\
\hline
Playwright & Alternativa moderna a Selenium (Microsoft) con API simplificada & Más rápido y estable que Selenium, API limpia y moderna, soporte nativo capturas/red & Comunidad más pequeña, menos ejemplos específicos scraping laboral \\
\hline
\end{tabular}
\end{table}

La decisión final adoptó Scrapy como herramienta principal por su rendimiento superior y robustez para scraping masivo de páginas estáticas o semi-dinámicas, características esenciales dado el volumen esperado de ofertas a procesar. Selenium se empleará como respaldo estratégico para portales que requieran ejecución de JavaScript para renderizar contenido dinámico, aceptando el trade-off de mayor latencia a cambio de cobertura completa. Playwright queda documentado como alternativa secundaria viable en caso de problemas técnicos con Selenium, ofreciendo mejor performance que este último pero menor madurez ecosistémica.

\subsection{Sistemas de Gestión de Bases de Datos}

Se evaluaron tres alternativas de sistemas de gestión de bases de datos con paradigmas diferentes: relacional tradicional, NoSQL orientado a documentos, y relacional embebido. La Tabla \ref{tab:comparacion-db} sintetiza las características, ventajas y limitaciones de cada opción.

\begin{table}[H]
\centering
\caption{Comparación de Sistemas de Gestión de Bases de Datos}
\label{tab:comparacion-db}
\small
\begin{tabular}{|p{2.5cm}|p{4.5cm}|p{4cm}|p{3.5cm}|}
\hline
\textbf{Sistema} & \textbf{Descripción} & \textbf{Ventajas} & \textbf{Limitaciones} \\
\hline
PostgreSQL & SGBD relacional open-source con soporte avanzado para consultas complejas e integridad referencial & Soporte JSON nativo, ACID completo, extensiones búsqueda texto (pg\_trgm), escalabilidad probada, integración pandas/SQLAlchemy & Requiere instalación/configuración local, mayor complejidad administrativa \\
\hline
MongoDB & Base NoSQL orientada a documentos con formato JSON/BSON flexible & Esquema flexible para datos semi-estructurados, almacenamiento anidaciones complejas, alto rendimiento escritura masiva & Sin integridad referencial estricta, consultas relacionales difíciles, menos adecuado para análisis estructurado \\
\hline
SQLite & BD relacional ligera embebida sin necesidad de servidor & Configuración mínima (archivo único), ideal prototipado rápido/datasets pequeños, compatible SQL estándar & Rendimiento limitado con grandes volúmenes, sin concurrencia escritura eficiente, carece funciones avanzadas PostgreSQL \\
\hline
\end{tabular}
\end{table}

La decisión final seleccionó PostgreSQL como base de datos principal por cuatro razones técnicas fundamentales. Primero, su capacidad para manejar esquemas estructurados con integridad referencial garantiza consistencia de datos crítica para análisis posteriores. Segundo, su soporte avanzado para consultas analíticas complejas con índices GIN y B-tree permite agregaciones eficientes sobre el corpus completo. Tercero, su extensibilidad para búsqueda textual mediante pg\_trgm facilita matching fuzzy de habilidades. Cuarto, su integración sólida con el ecosistema Python de ciencia de datos mediante SQLAlchemy y pandas reduce fricción en etapas de procesamiento y análisis.

\subsection{Bibliotecas de Procesamiento de Lenguaje Natural (NLP)}

Se evaluaron tres alternativas de bibliotecas para procesamiento de lenguaje natural en español, cada una con diferentes trade-offs entre velocidad, precisión y facilidad de uso. La Tabla \ref{tab:comparacion-nlp} presenta la comparación sistemática.

\begin{table}[H]
\centering
\caption{Comparación de Bibliotecas de Procesamiento de Lenguaje Natural}
\label{tab:comparacion-nlp}
\small
\begin{tabular}{|p{2.8cm}|p{4.2cm}|p{4cm}|p{3.5cm}|}
\hline
\textbf{Biblioteca} & \textbf{Descripción} & \textbf{Ventajas} & \textbf{Limitaciones} \\
\hline
spaCy & Biblioteca industrial NLP en Python optimizada para eficiencia con soporte español & Rápida y eficiente producción, modelos preentrenados calidad (es\_core\_news), soporte tokenización/lematización/POS/NER, integración HuggingFace & Modelos NER genéricos no especializados dominio laboral, requiere ajuste fino para habilidades técnicas \\
\hline
Stanford NLP / Stanza & Suite herramientas NLP Stanford con implementación Python & Modelos lingüísticos fundamento académico sólido, análisis sintáctico profundo (dependencias), modelos multilingües corpus universales & Más lento que spaCy, configuración compleja, menor integración embeddings modernos \\
\hline
Transformers HuggingFace & Modelos Transformer (BERT, RoBERTa) preentrenados español (BETO, RoBERTuito) & Representaciones contextuales alta calidad, BETO específico español, adaptable fine-tuning dominios & Requiere recursos computacionales significativos, latencia alta tiempo real, fine-tuning necesita datasets grandes \\
\hline
\end{tabular}
\end{table}

La decisión final adoptó spaCy como biblioteca principal de NLP por tres razones fundamentales. Primero, su balance entre velocidad y calidad en tareas básicas de NER y tokenización permite procesar el corpus completo en tiempo razonable. Segundo, su facilidad de uso y documentación robusta reduce la curva de aprendizaje del equipo. Tercero, su integración nativa con modelos Transformer de HuggingFace permite combinar eficiencia en tareas básicas con capacidades avanzadas cuando sea necesario. La estrategia complementará spaCy con expresiones regulares personalizadas para habilidades técnicas específicas no cubiertas por modelos genéricos, y evaluará el uso de modelos Transformer como BETO para tareas de enriquecimiento semántico en fases posteriores si los recursos computacionales lo permiten.

\subsection{Modelos de Embeddings Semánticos}

Se evaluaron cuatro alternativas de modelos para generación de representaciones vectoriales semánticas de habilidades, considerando tanto modelos especializados en similitud como embeddings generales. La Tabla \ref{tab:comparacion-embeddings} sintetiza la comparación.

\begin{table}[H]
\centering
\caption{Comparación de Modelos de Embeddings Semánticos}
\label{tab:comparacion-embeddings}
\footnotesize
\begin{tabular}{|p{2.3cm}|p{3.8cm}|p{4cm}|p{3.4cm}|}
\hline
\textbf{Modelo} & \textbf{Descripción} & \textbf{Ventajas} & \textbf{Limitaciones} \\
\hline
BETO & BERT preentrenado corpus masivo español para comprensión contextual & Entrenado específicamente español, representaciones contextuales alta calidad, compatible biblioteca Transformers & No optimizado similitud semántica densos, requiere fine-tuning sentence similarity, alto costo computacional \\
\hline
Multilingual-E5 / LaBSE & Modelos multilingües diseñados embeddings densos oraciones con similitud cross-lingual & Optimizados similitud semántica (cosine), soporte múltiples idiomas mismo espacio, LaBSE 100+ idiomas, E5 rendimiento superior benchmarks & Modelos grandes (memoria), menor especialización dominio laboral \\
\hline
SBERT & BERT extendido para embeddings oraciones vía siamese networks & Rápido generación vs BERT estándar, modelos multilingües disponibles (paraphrase-multilingual), adopción amplia similitud & Rendimiento variable idioma/dominio, algunos modelos sin español entrenamiento principal \\
\hline
fastText & Embeddings basados subpalabras livianos entrenables localmente & Rápido y ligero, maneja OOV vía subpalabras, entrenable corpus específico proyecto & No captura contexto semántico profundo, embeddings estáticos no contextuales, menor rendimiento similitud compleja \\
\hline
\end{tabular}
\end{table}

La decisión final adoptó Multilingual-E5 como modelo principal de embeddings por tres características críticas. Primero, su optimización específica para similitud semántica mediante entrenamiento contrastivo lo hace ideal para tareas de matching y clustering de habilidades. Segundo, su soporte multilingüe nativo permite procesar seamlessly el código mixto español-inglés característico del dominio tecnológico latinoamericano. Tercero, su rendimiento superior en benchmarks recientes de sentence similarity garantiza robustez empírica validada. Como alternativa secundaria se documentó LaBSE para escenarios que requieran mayor cobertura multilingüe, y fastText como respaldo computacionalmente liviano para experimentos rápidos o entornos con recursos limitados.

\section{Lenguajes y Herramientas}

El desarrollo del Observatorio de Demanda Laboral en Tecnología para Latinoamérica requiere una combinación de herramientas de software, lenguajes de programación, marcos de trabajo y bibliotecas especializadas que conforman el stack tecnológico integral del sistema.

\subsection{Stack Tecnológico Principal}

Python constituye el lenguaje de programación principal del proyecto debido a su simplicidad sintáctica, versatilidad para prototipado rápido y robusto ecosistema de bibliotecas especializadas en ciencia de datos, scraping web y procesamiento de lenguaje natural. La madurez de su comunidad open-source garantiza soporte continuo y abundante documentación técnica para los componentes críticos del pipeline.

La arquitectura de extracción de datos emplea Scrapy como framework principal para scraping asíncrono de alta eficiencia, complementado estratégicamente con Selenium para portales que requieren ejecución de JavaScript y renderizado dinámico de contenido, tal como se detalló en la comparación de herramientas de web scraping presentada anteriormente. El almacenamiento persistente se fundamenta en PostgreSQL como sistema de gestión de bases de datos relacionales, seleccionado por su robustez transaccional, soporte avanzado para consultas complejas mediante índices GIN y JSONB, y capacidades de integridad referencial críticas para garantizar consistencia de datos históricos.

El procesamiento de lenguaje natural integra spaCy con el modelo es\_core\_news\_lg como biblioteca principal, aprovechando su eficiencia computacional y modularidad para tokenización, lematización y reconocimiento de entidades nombradas en español. Esta capa se complementa con expresiones regulares personalizadas diseñadas específicamente para detectar patrones técnicos del dominio laboral latinoamericano que no están cubiertos por modelos genéricos. Como respaldo metodológico se consideran NLTK y Stanza para casos que requieran análisis sintáctico de dependencias o mayor granularidad lingüística, aunque su uso queda condicionado a evaluaciones específicas de desempeño versus complejidad.

El enriquecimiento semántico se implementa mediante Hugging Face Transformers con Gemma 3 4B Instruct como modelo de lenguaje principal, seleccionado tras evaluación comparativa de cuatro alternativas open-source considerando precisión en español, capacidad de razonamiento semántico y viabilidad computacional en hardware limitado. Este modelo permite inferir habilidades implícitas mediante prompting estructurado y normalizar variantes dialectales técnicas. El diseño se complementa con técnicas de Prompt Engineering consistentes en el diseño iterativo de instrucciones especializadas para tareas como normalización de habilidades a taxonomías estándar y extracción contextual de competencias técnicas no explícitas.

La representación vectorial semántica emplea la biblioteca SentenceTransformers para generar embeddings densos mediante modelos multilingües como Multilingual-E5, LaBSE y SBERT, tal como se presentó en la comparación de modelos de embeddings. La selección de Multilingual-E5 como modelo principal se fundamenta en su optimización específica para similitud semántica cross-lingual y superior desempeño en benchmarks de recuperación de información en español. Como alternativa de respaldo se mantiene fastText para casos que requieran embeddings ligeros entrenables localmente sobre vocabulario específico del dominio.

\subsection{Herramientas de Análisis y Visualización}

El agrupamiento y reducción de dimensionalidad combina UMAP para proyección no lineal de espacios de embeddings de alta dimensión a representaciones bidimensionales o tridimensionales interpretables, junto con HDBSCAN como algoritmo robusto de clustering basado en densidad que no requiere predefinir el número de grupos y maneja efectivamente ruido y outliers. Como métodos tradicionales de comparación se mantienen disponibles k-means para clustering particional y DBSCAN para validación cruzada de resultados, permitiendo evaluar la estabilidad de las agrupaciones obtenidas mediante múltiples técnicas.

La visualización de resultados emplea principalmente Plotly y Dash como bibliotecas especializadas para generación de gráficos interactivos y dashboards analíticos, aprovechando su capacidad de exportación a formatos estáticos de alta resolución para inclusión en documentos académicos. Como herramientas complementarias se utilizan Matplotlib y Seaborn para generar visualizaciones exploratorias rápidas durante el desarrollo y análisis preliminar de datos, facilitando iteraciones ágiles en el diseño de representaciones gráficas finales.

\subsection{Infraestructura de Desarrollo y Documentación}

El control de versiones y colaboración del código se gestiona mediante Git como sistema de control de versiones distribuido, integrado con GitHub como plataforma para hospedaje de repositorios, revisión de código mediante pull requests, seguimiento de issues y automatización de pruebas mediante GitHub Actions. Esta infraestructura garantiza trazabilidad completa de cambios, facilita trabajo paralelo en ramas independientes y documenta decisiones técnicas mediante commits descriptivos y discusiones en pull requests.

La documentación del proyecto se estructura en tres niveles complementarios. Google Docs se emplea para documentación colaborativa en tiempo real durante fases de diseño y planificación, permitiendo comentarios síncronos y versionado automático de decisiones arquitectónicas. Overleaf constituye el entorno principal para redacción final de documentos académicos formales en LaTeX, garantizando calidad tipográfica profesional y gestión eficiente de referencias bibliográficas mediante BibTeX. Finalmente, Markdown se estandariza como formato para documentación técnica embebida en el repositorio GitHub, incluyendo archivos README, guías de instalación, documentación de APIs internas y comentarios explicativos de notebooks Jupyter, facilitando navegación contextual de la documentación junto al código fuente.

\section{Plan de Aceptación del Producto}

El Plan de Aceptación del Producto define los criterios mediante los cuales cada entregable del Observatorio de Demanda Laboral en Tecnología para Latinoamérica será evaluado por el equipo académico para considerar su aceptación formal. Cada componente del sistema debe cumplir estándares específicos de calidad técnica, documentación y validación empírica antes de ser integrado al pipeline completo.

\subsection{Diseño técnico y arquitectura}

El entregable de diseño técnico y arquitectura será aceptado cuando se cumplan tres criterios fundamentales de completitud y rigor metodológico. Primero, debe existir un diagrama modular del pipeline completo documentado en notación BPMN o equivalente que represente claramente las siete etapas del sistema, los flujos de datos entre componentes y las tecnologías asignadas a cada módulo. Segundo, se requiere una justificación detallada y fundamentada de la elección de tecnologías específicas para cada componente, incluyendo análisis comparativo de alternativas evaluadas, trade-offs considerados y alineación con restricciones de hardware y tiempo del proyecto. Tercero, el documento metodológico inicial que formaliza el ciclo de vida CRISP-DM adaptado y la estrategia de evaluación dual debe estar validado formalmente por el asesor académico mediante firma o aprobación escrita.

La validación de este entregable se ejecutará mediante tres técnicas complementarias que garantizan rigor académico y coherencia técnica. La revisión por pares entre los integrantes del equipo asegurará consistencia interna de la documentación y detección temprana de inconsistencias arquitectónicas. La validación en reunión formal de asesoría permitirá al director del proyecto evaluar la viabilidad técnica y la alineación con objetivos académicos del trabajo de grado. Finalmente, el documento será compartido y versionado tanto en Google Docs para colaboración síncrona como en GitHub para trazabilidad histórica de decisiones arquitectónicas mediante commits descriptivos.

\begin{figure}[H]
\centering
\includegraphics[width=0.95\textwidth]{figures/DiagramaBPMN.png}
\caption{Diagrama BPMN del flujo general del proceso del observatorio}
\label{fig:bpmn-general}
\end{figure}

\subsection{Sistema de extracción (scraping) y carga a base de datos}

El módulo de extracción y carga de datos será considerado aceptable cuando demuestre robustez funcional en tres dimensiones operativas. Primero, deben existir spiders de Scrapy completamente funcionales para al menos dos portales de empleo por país objetivo, totalizando seis implementaciones independientes capaces de extraer título, descripción, empresa, ubicación y fecha de publicación de ofertas laborales. Segundo, el sistema debe demostrar almacenamiento exitoso y consistente de vacantes en PostgreSQL con un corpus mínimo de 500 registros reales extraídos de portales en producción, validando así la viabilidad de la estrategia de scraping a escala. Tercero, el código fuente debe estar documentado con docstrings descriptivos, estructurado modularmente mediante clases reutilizables y equipado con manejo robusto de excepciones y mecanismos de respaldo ante fallos de red o cambios en estructura HTML de portales.

La validación técnica de este componente empleará tres estrategias de verificación empírica. La validación de funcionamiento en tiempo real será ejecutada por el asesor académico mediante demostración en vivo del proceso de scraping, observando logs de ejecución y confirmando inserción de registros en base de datos durante sesión de asesoría. La revisión del script y estructura de base de datos analizará la calidad del código Python, la normalización del esquema relacional PostgreSQL, y la adecuación de índices para consultas posteriores. Finalmente, se ejecutará comparación sistemática de outputs contra muestras manuales de portales y verificación automática de duplicados mediante consultas SQL de agregación, garantizando integridad y deduplicación efectiva del corpus.

\subsection{Extracción de habilidades explícitas (NER y regex)}

El componente de extracción de habilidades explícitas mediante Pipeline A alcanzará aceptación cuando satisfaga tres requisitos de funcionalidad y evaluabilidad. Primero, debe estar completamente integrado al menos un modelo funcional de reconocimiento de entidades nombradas en español, específicamente spaCy con es\_core\_news\_lg extendido mediante EntityRuler con patrones ESCO, demostrando capacidad de identificar tecnologías, lenguajes de programación y frameworks en texto libre. Segundo, las expresiones regulares deben estar adaptadas específicamente al dominio laboral tecnológico latinoamericano, capturando variantes dialectales y patrones sintácticos característicos del español técnico, con resultados de extracción almacenados en formato estructurado evaluable mediante métricas cuantitativas. Tercero, el sistema debe generar anotaciones automatizadas de habilidades extraídas disponibles en formato JSON o CSV para validación posterior, incluyendo contexto de aparición y confianza de detección.

La evaluación de calidad se fundamentará en tres procedimientos de validación complementarios. La validación manual de muestras aleatorias será ejecutada sobre un subconjunto estadísticamente representativo de 50 ofertas laborales, verificando manualmente la completitud y precisión de habilidades detectadas versus lectura humana del texto completo. La revisión de logs y outputs del módulo de extracción analizará trazas de ejecución para identificar patrones de falsos positivos, falsos negativos y degradación de rendimiento en tipos específicos de ofertas. Finalmente, se ejecutará comparación sistemática con glosarios laborales estándar como ESCO y O*NET, calculando tasas de cobertura de la taxonomía y documentando habilidades emergentes no presentes en catálogos oficiales.

\subsection{Enriquecimiento semántico con LLMs}

El módulo de enriquecimiento semántico mediante Pipeline B con Gemma 3 4B Instruct será aceptado cuando cumpla tres estándares de diseño, desempeño y trazabilidad. Primero, deben existir prompts completamente definidos y documentados para al menos dos tareas críticas del sistema, específicamente normalización de habilidades a taxonomía ESCO e inferencia de competencias técnicas implícitas, incluyendo instrucciones del sistema, ejemplos few-shot y formato de salida estructurado JSON. Segundo, los resultados del modelo deben demostrar tasa de precisión superior al 70 por ciento sobre una muestra controlada de 100 ofertas anotadas manualmente, medida mediante métricas de F1-score en extracción y exactitud en normalización taxonómica. Tercero, el sistema debe garantizar trazabilidad completa del razonamiento del modelo mediante logging de prompts enviados, respuestas generadas y justificaciones textuales producidas por el LLM, permitiendo auditoría posterior de decisiones automatizadas.

La validación de este componente experimental seguirá tres estrategias de evaluación rigurosa. La evaluación cualitativa será ejecutada en reunión conjunta con el asesor académico, presentando casos representativos de inferencia semántica exitosa, limitaciones detectadas y análisis de errores sistemáticos del modelo. La comparación contra listas de habilidades base establecerá línea de referencia mediante contraste con outputs de Pipeline A tradicional y anotaciones manuales del gold standard, cuantificando mejora incremental aportada por razonamiento LLM. Finalmente, se generará reporte técnico exhaustivo documentando arquitectura de prompts, ejemplos representativos de inputs y outputs, explicaciones de casos extremos y recomendaciones de refinamiento iterativo del diseño de instrucciones.

\subsection{Embeddings y clustering}

El pipeline de representación vectorial y agrupamiento semántico alcanzará criterios de aceptación cuando demuestre tres capacidades técnicas fundamentales. Primero, todas las habilidades extraídas deben estar representadas mediante embeddings densos de 768 dimensiones en formato vectorial numpy o tensors PyTorch almacenados persistentemente, generados mediante Multilingual-E5 con normalización L2 para garantizar comparabilidad mediante similitud coseno. Segundo, la aplicación de clustering debe ejecutarse exitosamente mediante HDBSCAN sobre proyección UMAP bidimensional o tridimensional, produciendo al menos tres clústeres semánticamente significativos interpretables por expertos del dominio, con análisis cualitativo detallado de coherencia temática intra-cluster y separabilidad inter-cluster. Tercero, debe existir visualización preliminar interactiva o estática de alta resolución generada mediante UMAP que represente espacialmente las agrupaciones de habilidades, incluyendo etiquetado de clústeres principales y destacado de habilidades representativas.

La validación de calidad del clustering combinará análisis cuantitativo y cualitativo mediante tres técnicas complementarias. La validación de cohesión semántica entre elementos de un mismo clúster será ejecutada manualmente por los investigadores, verificando que tecnologías agrupadas compartan dominio de aplicación, nivel de abstracción o contexto de uso típico, documentando casos de agrupación exitosa y outliers mal clasificados. La revisión técnica incluirá generación de gráficos de proyección UMAP coloreados por cluster y cálculo de métricas intrínsecas como Silhouette Score, DBCV y estadísticos de distribución de tamaños de clústeres, estableciendo umbrales mínimos de calidad de agrupamiento. Finalmente, se producirá informe de análisis de clústeres interpretado colaborativamente por el equipo, identificando taxonomías emergentes de habilidades, tendencias de co-ocurrencia tecnológica y patrones geográficos o temporales en la demanda laboral.

\subsection{Visualización macro}

El módulo de visualización macro cumplirá estándares de aceptación cuando genere productos gráficos interpretables, exportables y validados técnicamente mediante tres criterios de completitud analítica. Primero, debe producir al menos tres tipos de visualizaciones complementarias que representen dimensiones críticas del observatorio: distribución de frecuencia de habilidades más demandadas mediante gráficos de barras o wordclouds ponderados, distribución geográfica de perfiles tecnológicos mediante mapas de calor por país o visualizaciones comparativas multi-región, y proyección espacial de clústeres semánticos mediante scatter plots UMAP coloreados por agrupación con leyendas descriptivas. Segundo, todas las visualizaciones deben ser exportables a formatos de alta resolución PNG o PDF con dimensiones adecuadas para inclusión en documentos académicos, acompañadas de reportes textuales interpretando tendencias identificadas, outliers relevantes y limitaciones de los análisis. Tercero, las interfaces de generación de gráficos deben exhibir usabilidad mínima para revisión técnica, permitiendo ajuste de parámetros básicos como rangos de visualización, filtros temporales o geográficos, y personalización de esquemas de color.

La validación de calidad gráfica y analítica seguirá tres procedimientos de evaluación iterativa. La presentación formal ante el asesor académico incluirá demostración en vivo de generación de visualizaciones con feedback inmediato sobre claridad interpretativa, adecuación de escalas y elección de representaciones gráficas, documentando sugerencias de refinamiento en actas de reunión. La validación del código en entorno local verificará reproducibilidad de gráficos mediante ejecución independiente por ambos integrantes del equipo, confirmando ausencia de dependencias de rutas absolutas y documentación adecuada de bibliotecas requeridas. Finalmente, se ejecutará revisión sistemática de consistencia visual y semántica de las gráficas, verificando que colores, leyendas y títulos sean autoexplicativos, que escalas numéricas sean apropiadas sin distorsiones perceptuales, y que interpretaciones textuales estén fundamentadas empíricamente en datos presentados.

\subsection{Documentación técnica y guía metodológica}

El entregable de documentación técnica y guía metodológica constituye componente crítico para asegurar reproducibilidad científica y transferencia de conocimiento, siendo aceptado cuando satisfaga tres estándares de completitud, claridad y trazabilidad. Primero, debe existir redacción clara, completa y estructurada de todas las etapas metodológicas del pipeline CRISP-DM adaptado, incluyendo justificación de decisiones arquitectónicas, descripción detallada de algoritmos implementados, especificación de hiperparámetros seleccionados y documentación de experimentos fallidos con lecciones aprendidas que informen futuras iteraciones. Segundo, el documento debe contener instrucciones de replicación paso a paso suficientemente detalladas para que un investigador externo con conocimientos equivalentes pueda reproducir el sistema completo, especificando versiones exactas de dependencias Python, comandos de instalación, parámetros de configuración de base de datos y procedimientos de descarga de modelos preentrenados. Tercero, la documentación debe incluir artefactos técnicos completos como logs representativos de ejecuciones exitosas y fallidas, textos completos de prompts LLM utilizados con anotaciones explicativas, scripts auxiliares de preprocesamiento y análisis, y resultados clave tabulados incluyendo métricas de desempeño, ejemplos de outputs y visualizaciones finales.

La validación de calidad documental seguirá tres procedimientos rigurosos de evaluación académica. La revisión integral por parte del asesor académico examinará coherencia argumentativa, profundidad técnica, adecuación de referencias bibliográficas a trabajos relacionados y cumplimiento de estándares de formato institucional LaTeX de la universidad. La comparación con estándares académicos internacionales de replicabilidad verificará que el documento contenga todos los elementos mínimos especificados en guías como CRISP-DM, incluyendo diccionario de datos, diagramas de arquitectura, pseudocódigo de algoritmos críticos y análisis de limitaciones y trabajo futuro. Finalmente, se ejecutará validación cruzada exhaustiva entre documento metodológico y repositorio GitHub, confirmando que todos los módulos mencionados existan en el código, que versiones de software documentadas coincidan con requirements.txt, y que ejemplos de outputs presentados sean auténticamente generados por el sistema y no artificialmente construidos.

\section{Organización del Proyecto y Comunicación}

\subsection{Interfaces Externas}

En el desarrollo del proyecto se identifican varias entidades externas que, aunque no forman parte directa del equipo de desarrollo, cumplen funciones esenciales en el acompañamiento académico, la evaluación, la provisión de insumos y la validación conceptual del sistema.

\begin{table}[H]
\centering
\small
\begin{tabular}{|p{3cm}|p{4cm}|p{5cm}|p{2.5cm}|}
\hline
\textbf{Entidad externa} & \textbf{Rol en el proyecto} & \textbf{Tipo de interacción} & \textbf{Frecuencia} \\
\hline
Director del Proyecto (Ing. Luis Gabriel Moreno Sandoval) & Supervisión académica, validación técnica, orientación metodológica & Reuniones de seguimiento, revisión de entregables, aprobación de decisiones clave & Quincenal \\
\hline
Docentes evaluadores & Evaluación formal del trabajo de grado, validación de calidad académica & Presentaciones formales, defensa pública, retroalimentación escrita & 2-3 sesiones durante el proyecto \\
\hline
Portales de empleo (LinkedIn, Computrabajo, Bumeran, Indeed) & Fuentes de datos primarias (ofertas laborales) & Acceso web mediante scraping, consulta de APIs públicas (si disponibles) & Diaria durante fase de scraping \\
\hline
Comunidades técnicas (Stack Overflow, GitHub Issues, foros de spaCy/HuggingFace) & Soporte técnico, resolución de dudas, acceso a ejemplos y soluciones & Consulta de documentación, publicación de issues, revisión de ejemplos & Según necesidad \\
\hline
Proveedores de modelos preentrenados (HuggingFace, spaCy, OpenAI) & Acceso a modelos de NLP, embeddings y LLMs & Descarga de modelos, uso de APIs gratuitas o académicas & Según fase técnica \\
\hline
Pontificia Universidad Javeriana (Departamento de Sistemas) & Provisión de recursos institucionales, validación académica, aprobación formal del trabajo & Entrega de documentos formales, uso de recursos bibliotecarios, acceso institucional & Permanente \\
\hline
Colegas y compañeros de carrera & Revisión cruzada de código, retroalimentación informal, validación de usabilidad & Sesiones de código compartido, discusiones técnicas informales & Ocasional \\
\hline
\end{tabular}
\caption{Tabla de Interfaces Externas del Proyecto}
\label{tab:interfaces-externas}
\end{table}

La gestión de las interfaces externas sigue tres estrategias operativas diferenciadas según el tipo de entidad. La comunicación con el director del proyecto se realizará mediante reuniones quincenales programadas con agenda previa compartida, comunicación por correo electrónico para consultas urgentes que requieran respuesta en menos de 48 horas, y revisión colaborativa de entregables mediante Google Drive compartido para documentos en progreso y GitHub para código versionado con pull requests descriptivos. El acceso a portales de empleo se gestionará respetando estrictamente términos de servicio, archivos robots.txt y políticas anti-scraping de cada sitio, implementando delays entre peticiones de 2-5 segundos, rotación automática de user agents para simular navegación diversa y limitación de tasa de peticiones a máximo 1-2 requests por segundo por portal. El uso de modelos y herramientas de código abierto se documentará exhaustivamente, citando licencias específicas y repositorios oficiales de cada biblioteca, respetando términos de uso académico cuando aplique y contribuyendo reportes de bugs o mejoras a comunidades open-source cuando sea técnicamente viable.

\subsection{Organigrama y Descripción de Roles}

El equipo de desarrollo del proyecto está conformado por dos estudiantes de la carrera de Ingeniería de Sistemas de la Pontificia Universidad Javeriana, cada uno con responsabilidades claramente definidas según sus fortalezas técnicas y organizativas, garantizando cobertura completa de las fases del pipeline mediante distribución balanceada de carga de trabajo.

Nicolás Camacho Alarcón asume el rol de Líder Técnico y Arquitecto del Sistema, siendo responsable del diseño de la arquitectura general del observatorio, coordinación técnica entre módulos interdependientes, toma de decisiones clave sobre selección de modelos de NLP, herramientas de scraping y estructura del pipeline de procesamiento. Adicionalmente supervisa la integración funcional de componentes desarrollados independientemente y garantiza coherencia técnica del sistema completo mediante revisiones de código y validación de estándares de calidad establecidos.

Alejandro Pinzón Fajardo cumple el rol de Desarrollador de Módulos y Responsable de Documentación, estando encargado del desarrollo técnico detallado de componentes específicos del sistema incluyendo implementación de spiders de scraping, módulos de procesamiento de texto, generación de embeddings y construcción de visualizaciones macro. También lidera la redacción de documentos formales académicos siguiendo estándares institucionales LaTeX, la planificación sistemática de pruebas funcionales por módulo y la ejecución de validaciones empíricas sobre muestras controladas del corpus.

\begin{figure}[H]
\centering
\begin{tikzpicture}[
    node distance=1.5cm and 2cm,
    every node/.style={font=\small},
    director/.style={
        rectangle,
        draw=black,
        thick,
        fill=blue!20,
        text width=5.5cm,
        align=center,
        minimum height=1.2cm,
        rounded corners=3pt,
        drop shadow
    },
    equipo/.style={
        rectangle,
        draw=black,
        thick,
        fill=green!15,
        text width=5cm,
        align=center,
        minimum height=2.8cm,
        rounded corners=3pt,
        drop shadow
    },
    line/.style={
        draw,
        thick,
        -Stealth
    }
]

% Nodo del Director
\node[director] (director) {
    \textbf{Director del Proyecto}\\[2pt]
    \small Ing. Luis Gabriel Moreno Sandoval\\[1pt]
    \footnotesize Pontificia Universidad Javeriana
};

% Nodo del Equipo de Desarrollo
\node[equipo, below=of director] (equipo) {
    \textbf{Equipo de Desarrollo}\\[6pt]
    \textbf{Nicolás Camacho Alarcón}\\
    \footnotesize Líder Técnico y Arquitecto del Sistema\\[4pt]
    \textbf{Alejandro Pinzón Fajardo}\\
    \footnotesize Desarrollo de Módulos y Documentación
};

% Conexión
\draw[line] (director) -- (equipo);

\end{tikzpicture}
\caption{Organigrama del equipo de desarrollo del proyecto}
\label{fig:organigrama}
\end{figure}

\chapter{Administración del Proyecto}

\section{Métodos y Herramientas de Estimación}

El presente proyecto ha sido estimado utilizando una combinación de métodos empíricos y de juicio experto, apoyados en la experiencia previa de los integrantes del equipo, la naturaleza del trabajo requerido, y la complejidad técnica de cada fase. Dado que no se cuenta con herramientas formales de estimación como PERT o COCOMO, se optó por un enfoque ágil y práctico, ajustado a las condiciones reales del equipo.

\subsection{Método de Estimación del Proyecto}

El presente proyecto se apoya en una estimación realista de esfuerzos basada en tres principios clave: el análisis de tareas específicas por fase metodológica, la disponibilidad horaria semanal del equipo, y la experiencia práctica en proyectos previos similares. Dado que se trata de un trabajo de grado con una duración prevista de entre 14 y 16 semanas y un equipo de tres personas, se estimó un rango total de \textbf{440 a 500 horas de trabajo distribuidas entre los tres integrantes}.

\subsubsection{Método de estimación utilizado}

Para esta estimación se empleó una \textbf{técnica de descomposición (WBS)}, donde cada fase del proyecto fue dividida en tareas concretas, a las que se asignaron horas aproximadas según su complejidad, herramientas requeridas y experiencia previa del equipo. Esta estimación fue luego contrastada con la disponibilidad semanal esperada de cada integrante, ajustada por compromisos académicos paralelos.

El enfoque adoptado es cualitativo, iterativo y conservador. En lugar de aplicar fórmulas matemáticas complejas (como COCOMO), se optó por la \textbf{estimación fundamentada en experticia del equipo}, validada contra planes previos y distribuida de forma proporcional.

\subsubsection{Estimación del esfuerzo por integrante}

\begin{table}[H]
\centering
\begin{tabular}{|l|l|c|}
\hline
\textbf{Integrante} & \textbf{Rol principal} & \textbf{Horas estimadas} \\
\hline
Nicolás Camacho & Líder técnico, arquitectura & 170 \\
\hline
Alejandro Pinzón & Documentación y pruebas & 130 \\
\hline
Alejandro Pinzón & Desarrollo de módulos & 140 \\
\hline
\textbf{Total estimado} & & \textbf{440 horas} \\
\hline
\end{tabular}
\caption{Tiempo de Esfuerzo por Integrante}
\end{table}

\subsubsection{Estimación por fase metodológica}

\begin{table}[H]
\centering
\begin{tabular}{|l|c|}
\hline
\textbf{Fase} & \textbf{Horas estimadas} \\
\hline
Diseño inicial del sistema & 45 \\
\hline
Scraping y carga a base de datos & 70 \\
\hline
NER y normalización de habilidades & 60 \\
\hline
Enriquecimiento con LLMs & 60 \\
\hline
Embeddings y clustering & 50 \\
\hline
Visualización macro evaluativa & 40 \\
\hline
Documentación, validación y ajustes finales & 70 \\
\hline
\textbf{Total} & \textbf{395 horas} \\
\hline
\end{tabular}
\caption{Tiempo Estimado por Fase}
\end{table}

\textit{Nota:} El total por fases metodológicas no considera horas de reuniones, organización, imprevistos o revisión con el asesor, por lo que se reserva un \textbf{margen adicional de 45–55 horas} para estas actividades transversales, que completan las 440–500 horas globales.

\subsubsection{Herramientas empleadas para la estimación}

\begin{itemize}
    \item \textbf{Google Sheets} para tabular las actividades y distribuir tiempos por fase e integrante.
    \item \textbf{Planificación semanal estimada} en base a 8–12 horas de dedicación por persona por semana.
    \item \textbf{Retroalimentación del director de proyecto}, validando la viabilidad y distribución del esfuerzo.
\end{itemize}

\section{Inicio del proyecto}

\subsection{Entrenamiento del Personal}

Dado que el proyecto ``Observatorio de Demanda Laboral en Tecnología en Latinoamérica'' involucra herramientas avanzadas como modelos de lenguaje, scraping dinámico, procesamiento semántico en español, embeddings multilingües y técnicas de clustering no supervisado, se ha establecido un plan de entrenamiento ligero pero enfocado, que permita nivelar al equipo en los aspectos técnicos esenciales sin comprometer el cronograma establecido.

El entrenamiento será liderado por Nicolás Camacho, quien cuenta con mayor experiencia técnica en las tecnologías clave del proyecto. Su rol incluirá el diseño de pequeñas cápsulas de formación y la transferencia directa de conocimiento al resto del equipo mediante los siguientes mecanismos:

\begin{itemize}
    \item \textbf{Explicaciones prácticas durante las reuniones semanales:} Nicolás explicará en vivo el funcionamiento del código que esté desarrollando en cada fase, incluyendo justificación técnica y recomendaciones de buenas prácticas.

    \item \textbf{Ejemplos rápidos y contextualizados:} Se utilizarán notebooks o scripts breves con datos simples para ilustrar conceptos clave como scraping con Selenium, prompts con LLMs, embeddings, UMAP y clustering.

    \item \textbf{Lecturas y recursos recomendados:} Se compartirá documentación oficial, artículos clave y videos cortos en español o inglés, según el tema y nivel de dificultad.

    \item \textbf{Revisión cruzada de avances:} Los integrantes aplicarán lo aprendido directamente en sus tareas asignadas, recibiendo retroalimentación de Nicolás de forma continua.
\end{itemize}

A continuación, se detalla la planificación del entrenamiento interno:

\begin{table}[H]
\centering
\small
\begin{tabular}{|p{3.5cm}|p{2cm}|p{2.5cm}|p{3cm}|p{2.5cm}|}
\hline
\textbf{Habilidad o tema técnico} & \textbf{Integrantes a entrenar} & \textbf{Responsable del entrenamiento} & \textbf{Método principal} & \textbf{Tiempo estimado} \\
\hline
Scraping web con Scrapy, Selenium y Playwright & Alejandro & Nicolás & Ejemplos en vivo + código comentado & Semanas 2–3 \\
\hline
Procesamiento de lenguaje en español (spaCy, regex, taxonomías laborales) & Alejandro & Nicolás & Lecturas + demos & Semanas 3–5 \\
\hline
Uso de LLMs y diseño de prompts & Alejandro & Nicolás & Ejercicios guiados + discusión en grupo & Semanas 5–6 \\
\hline
Embeddings multilingües y reducción dimensional & Alejandro & Nicolás & Notebooks breves + aplicación directa & Semanas 6–7 \\
\hline
Clustering con HDBSCAN y UMAP & Alejandro & Nicolás & Ejemplo sintético + retroalimentación & Semana 8 \\
\hline
Visualización macro con Dash / Plotly & Alejandro & Nicolás & Revisión cruzada de visualizaciones & Semanas 9–10 \\
\hline
Documentación técnica y guía metodológica & Todo el equipo & Alejandro & Plantillas + validación por Nicolás & Permanente \\
\hline
\end{tabular}
\caption{Plan de Entrenamiento Interno}
\end{table}

Este entrenamiento se desarrollará de manera continua, práctica y orientada a resolver las necesidades reales del proyecto. No se contempla una fase de formación previa extensa, sino que el aprendizaje estará integrado al flujo de trabajo, adaptado al tiempo disponible y priorizando las tareas más urgentes.

Nicolás también brindará apoyo puntual cuando surjan dudas específicas, y fomentará una cultura de colaboración y mejora técnica, donde se valoran las preguntas, la exploración autónoma y la mejora iterativa. La documentación técnica, liderada por Daniel, incluirá además las explicaciones simplificadas de los procesos clave, con el fin de reforzar el aprendizaje colectivo y dejar registro de las decisiones técnicas para futuras iteraciones o réplicas del proyecto.

\subsection{Infraestructura}

Para el adecuado desarrollo del proyecto ``Observatorio de Demanda Laboral en Tecnología en Latinoamérica'', se requiere una infraestructura técnica que permita la implementación modular del sistema, desde la recolección de datos hasta el análisis semántico y la documentación de resultados. Esta infraestructura combina herramientas de software de código abierto, recursos computacionales locales y servicios colaborativos en la nube, con una planeación clara sobre su adquisición, configuración y mantenimiento.

Los detalles completos de herramientas de software, especificaciones de equipos, y actividades de obtención, despliegue y mantenimiento se encuentran documentados en las tablas 9, 10, 11, 12 y 13 del presente documento.

\section{Planes de Trabajo del Proyecto}

\subsection{Descomposición de Actividades}

La estructura de descomposición de actividades del proyecto (WBS - Work Breakdown Structure) se organiza por fases metodológicas, cada una con sus respectivas subactividades. La siguiente tabla representa la estructura general adoptada para la ejecución, según se detalla en la Tabla 14.

\subsection{Calendarización}

La calendarización define las fechas estimadas de inicio y finalización de cada fase principal, así como su secuencia de ejecución, distribuida en una duración total de 16 semanas, tal como se muestra en la Tabla 15 y la Figura 4 (Carta Gantt).

\begin{figure}[H]
\centering
\includegraphics[width=\textwidth]{figures/CartaGantt.png}
\caption{Carta Gantt del cronograma del proyecto (16 semanas)}
\label{fig:carta-gantt}
\end{figure}

\subsection{Asignación de Recursos}

Para cada fase principal del proyecto, se han identificado los recursos necesarios en términos de recursos humanos, software, hardware y documentación, con el fin de asegurar el cumplimiento eficiente de los objetivos. Los detalles completos se encuentran en la Tabla 16.

\subsection{Asignación de Presupuesto y Justificación}

Este proyecto no contempla flujo de dinero real ni remuneración alguna para los integrantes. Sin embargo, se presenta una estimación simbólica del costo técnico potencial, que refleja lo que implicaría económicamente replicar el esfuerzo si se ejecutara en un entorno profesional o institucional. Esta estimación es útil para evaluar necesidades técnicas, justificar decisiones y proyectar posibles inversiones en caso de escalamiento.

El presupuesto estimado se presenta en la Tabla 17, con un total aproximado de hasta \$8.600.000 COP en valor técnico para réplica profesional del proyecto.

\textit{Esta asignación tiene fines estimativos y no representa una solicitud ni gestión presupuestal formal. Ningún recurso económico será solicitado ni entregado a los integrantes.}

\chapter{Monitoreo y Control del Proyecto}

\section{Administración de Requerimientos}

La administración de requerimientos en este proyecto sigue un enfoque pragmático y orientado a la adaptabilidad técnica, en el que se priorizan los requerimientos funcionales directamente vinculados con las fases metodológicas definidas y se permite cierto grado de flexibilidad en la implementación cuando surgen limitantes técnicas, cambios en herramientas disponibles o imprevistos identificados durante las retrospectivas semanales.

Dada la naturaleza académica del proyecto y su metodología híbrida basada en CRISP-DM y Scrum no estricto, los requerimientos serán gestionados de forma iterativa, sin pretender alcanzar niveles de rigidez formal propios de proyectos empresariales bajo contratos estrictos. En cambio, se promoverá la documentación continua de decisiones técnicas, la trazabilidad modular de los cambios, y la validación interna de que cada fase cumple con los objetivos esperados antes de avanzar a la siguiente.

El proceso general de gestión de requerimientos se presenta en el siguiente diagrama BPMN:

\begin{figure}[H]
\centering
\includegraphics[width=0.95\textwidth]{figures/BPMNAdministraciondeRequerimientos.png}
\caption{Proceso de Administración de Requerimientos (BPMN)}
\label{fig:bpmn-requerimientos}
\end{figure}

\subsection{Proceso de Gestión de Cambios}

Los cambios a los requerimientos podrán ser propuestos por cualquier miembro del equipo de desarrollo, el asesor académico del proyecto, o identificados mediante descubrimiento técnico emergente durante fases de ejecución cuando limitaciones no previstas o oportunidades de mejora se hacen evidentes. Todo cambio propuesto seguirá flujo estructurado de cinco etapas secuenciales que garantizan análisis riguroso de implicaciones y trazabilidad completa de decisiones.

La primera etapa corresponde a identificación formal del cambio, donde el integrante que detecta la necesidad documenta exhaustivamente el motivo técnico o metodológico que lo origina, el impacto esperado sobre funcionalidad del sistema o calidad de resultados, y las alternativas posibles de implementación con análisis comparativo preliminar de trade-offs. La segunda etapa ejecuta evaluación sistemática de impacto mediante discusión estructurada en reunión semanal del equipo, analizando si el cambio afecta cronograma global o de fases específicas, redistribución de carga de trabajo entre integrantes, introducción de nuevas dependencias técnicas entre módulos, o modificación de entregables comprometidos con el asesor.

La tercera etapa consiste en decisión formal por consenso del equipo técnico, donde el cambio se aprueba o rechaza mediante acuerdo explícito de ambos integrantes fundamentado en análisis de viabilidad técnica y alineación con objetivos del proyecto. En caso de desacuerdo persistente entre integrantes sobre viabilidad o prioridad del cambio, el asesor del proyecto Ing. Luis Gabriel Moreno Sandoval tendrá voz decisoria final para desbloquear la decisión. La cuarta etapa actualiza documentación del proyecto si el cambio es aprobado, registrando formalmente la modificación en acta semanal de reunión con justificación explícita, actualizando documento técnico del proyecto en sección de decisiones arquitectónicas, y ajustando planificación WBS afectada incluyendo redistribución de horas estimadas si aplica.

La quinta etapa final ejecuta comunicación formal y ajuste operativo del trabajo en curso, donde los integrantes afectados por el cambio aprobado ajustan implementación actual conforme a nueva especificación, actualizan código y documentación técnica correspondiente, y notifican formalmente el estado de implementación del cambio en la siguiente sesión de seguimiento quincenal con el asesor académico.

\subsection{Trazabilidad de Requerimientos}

Los requerimientos funcionales del observatorio están vinculados directamente a las siete fases del proyecto CRISP-DM adaptado y sus entregables técnicos asociados, garantizando que cada funcionalidad especificada pueda ser rastreada desde su origen conceptual hasta su validación empírica final. La trazabilidad se mantiene operativa mediante cuatro mecanismos complementarios de documentación y versionado que cubren diferentes niveles de granularidad del sistema.

El primer mecanismo consiste en tabla de requerimientos funcionales y fases incluida en el documento de especificación de requerimientos, donde cada requerimiento funcional está explícitamente asociado a una o más fases metodológicas del WBS, facilitando identificación precisa del momento de su implementación técnica, validación funcional y prueba de aceptación. El segundo mecanismo emplea el repositorio de código versionado en GitHub, donde cada módulo del sistema tiene carpeta identificada con estructura jerárquica clara y commits descriptivos que referencian mediante tags o mensajes la fase técnica correspondiente, el requerimiento funcional implementado y el autor responsable de la implementación.

El tercer mecanismo mantiene actas de seguimiento semanal almacenadas en Google Drive compartido, constituyendo registro cronológico de avances donde se indica explícitamente qué requerimientos funcionales fueron implementados durante la semana, cuáles fueron probados exitosamente mediante validación funcional, y cuáles sufrieron modificaciones de especificación o priorización por decisiones del equipo o feedback del asesor. El cuarto mecanismo documenta en sección específica del documento técnico final una matriz de trazabilidad bidireccional que vincula cada requerimiento funcional con su fase de diseño arquitectónico inicial, etapa de implementación técnica, procedimientos de prueba ejecutados y resultados de validación empírica con métricas cuantitativas de aceptación.

\subsection{Aprobación y Validación de Cambios}

Todos los cambios propuestos a requerimientos funcionales o no funcionales del sistema deben ser validados formalmente por tres instancias complementarias de revisión que garantizan viabilidad técnica, coherencia académica y alineación con objetivos institucionales del trabajo de grado.

La primera instancia corresponde al equipo de desarrollo conformado por Nicolás Camacho y Alejandro Pinzón, quienes ejecutan validación técnica de viabilidad del cambio propuesto, analizando si es implementable con stack tecnológico actual, si requiere refactorización significativa de módulos ya desarrollados, si introduce deuda técnica aceptable o crítica, y si su implementación cabe dentro del tiempo restante del cronograma considerando carga de trabajo existente. La segunda instancia involucra al asesor del proyecto Ing. Luis Gabriel Moreno Sandoval, quien ejecuta validación académica del cambio propuesto, verificando coherencia con objetivos específicos del trabajo de grado, alineación con metodología CRISP-DM establecida, pertinencia para aportes esperados del proyecto y adecuación a estándares de calidad de trabajos de investigación en Ingeniería de Sistemas. La tercera instancia requiere aprobación formal del director del proyecto si el cambio implica modificación significativa del alcance definido inicialmente, ajuste de fechas de entregables comprometidos institucionalmente, o reasignación sustancial de recursos humanos o técnicos que afecte viabilidad global del cronograma.

La aprobación formal de cambios validados se materializa mediante firma digital del asesor en documento PDF del acta correspondiente o mediante confirmación escrita explícita en correo electrónico institucional archivado en carpeta compartida del proyecto, garantizando trazabilidad auditable de todas las decisiones de gestión de cambios ejecutadas durante el desarrollo.

\section{Monitoreo y Control de Progreso}

El monitoreo del proyecto se realizará de forma continua mediante reuniones semanales, indicadores de avance y mecanismos de reporte estructurado que permitan detectar desviaciones de forma temprana y aplicar correcciones con suficiente anticipación.

\subsection{Métricas de Seguimiento}

Se emplearán las siguientes métricas para evaluar el avance del proyecto:

\begin{table}[H]
\centering
\small
\begin{tabular}{|p{3.5cm}|p{4.5cm}|p{3cm}|p{3cm}|}
\hline
\textbf{Métrica} & \textbf{Descripción} & \textbf{Frecuencia de medición} & \textbf{Responsable} \\
\hline
Porcentaje de avance por fase & Proporción de actividades completadas en cada fase metodológica & Semanal & Líder técnico \\
\hline
Horas trabajadas acumuladas & Total de horas dedicadas por cada integrante & Semanal & Cada integrante \\
\hline
Número de requerimientos implementados & Cantidad de funcionalidades técnicas completadas y validadas & Quincenal & Líder técnico \\
\hline
Tasa de cumplimiento del cronograma & Relación entre fechas planificadas y fechas reales de finalización de fases & Al finalizar cada fase & Coordinador de proyecto \\
\hline
Cantidad de defectos encontrados en validación & Errores técnicos detectados en pruebas funcionales & Al finalizar cada fase & Encargado de pruebas \\
\hline
\end{tabular}
\caption{Métricas de Seguimiento del Proyecto}
\end{table}

\subsection{Actividades de Reporte}

El reporte estructurado del avance del proyecto se realizará mediante cuatro mecanismos complementarios de comunicación que operan en diferentes frecuencias y niveles de detalle técnico, garantizando transparencia continua hacia el asesor académico y coordinación efectiva interna del equipo.

El primer mecanismo consiste en reuniones semanales de seguimiento ejecutadas cada lunes con duración de 1 hora, requiriendo presencia sincrónica de todo el equipo mediante videoconferencia o sesión presencial. La agenda estructurada incluye discusión del avance técnico de la semana anterior medido contra métricas del WBS, presentación de bloqueos técnicos encontrados con descripción de síntomas y hipótesis de causas, revisión cuantitativa de métricas de avance por fase y por integrante, y planificación detallada de tareas de la siguiente semana con asignación explícita de responsables y dependencias entre actividades.

El segundo mecanismo genera actas de reunión documentadas al finalizar cada sesión semanal, incluyendo registro de asistencia con nombres y roles, temas técnicos y administrativos tratados, decisiones tomadas con justificación y votación si aplica, compromisos adquiridos por cada integrante con fechas límite específicas, y fecha confirmada de próxima reunión. Las actas se almacenan en carpeta compartida de Google Drive con nomenclatura estandarizada que incluye fecha ISO 8601 para facilitar búsqueda cronológica y auditoría posterior.

El tercer mecanismo establece reportes técnicos quincenales al asesor del proyecto, enviados cada dos semanas mediante correo electrónico institucional con documento PDF adjunto de 1 a 2 páginas estructurado en secciones estándar. El contenido incluye estado actual de cada fase metodológica con porcentaje de completitud estimado, problemas técnicos detectados durante el período con severidad clasificada, soluciones técnicas aplicadas o en evaluación con análisis de alternativas, y planes de trabajo para las próximas dos semanas con énfasis en hitos críticos del cronograma.

El cuarto mecanismo ejecuta reunión de revisión de fase al finalizar cada una de las siete fases metodológicas del CRISP-DM adaptado, consistiendo en sesión extendida con el asesor de hasta 2 horas de duración donde se presenta formalmente el entregable técnico correspondiente mediante demostración funcional en vivo, se valida calidad técnica y académica del resultado obtenido mediante inspección de código y análisis de métricas de evaluación, y se ajusta colaborativamente el plan de trabajo para fases subsiguientes si desviaciones detectadas o lecciones aprendidas así lo requieren.

\subsection{Acciones Correctivas}

Cuando se detecte desviación significativa en el cronograma planificado, degradación de recursos técnicos o humanos disponibles, o deterioro de calidad técnica de entregables contra criterios de aceptación establecidos, el equipo podrá aplicar cuatro tipos de acciones correctivas escaladas según severidad del problema identificado.

La primera acción correctiva consiste en reprogramación táctica de actividades del WBS, aplicable cuando una fase específica se retrasa más de una semana respecto a fechas comprometidas. El equipo evalúa reducir alcance técnico de otras fases menos críticas mediante priorización de requerimientos funcionales, redistribuir tareas entre integrantes según especialización y carga actual de trabajo, o ajustar cronograma global mediante extensión de fechas con aprobación formal del asesor académico y análisis de impacto en fecha de defensa final.

La segunda acción correctiva implementa refuerzo técnico colaborativo ante bloqueos técnicos que superan capacidad individual de resolución, donde Nicolás Camacho en su rol de líder técnico brindará acompañamiento adicional intensivo a los integrantes afectados, incluyendo sesiones de pair programming sincrónico de 2 a 4 horas para debugging conjunto de código problemático, revisión arquitectónica guiada de diseño de módulos con alto acoplamiento o baja cohesión, y transferencia acelerada de conocimiento especializado en componentes técnicos complejos del stack.

La tercera acción correctiva ejecuta simplificación técnica controlada cuando una técnica o algoritmo planificado resulta inviable por limitaciones computacionales de hardware disponible, insuficiencia de datos de entrenamiento o anotación, o complejidad de implementación excesiva que amenaza cronograma crítico. El equipo reemplazará la técnica original por alternativa más simple pero técnicamente válida y académicamente defendible, como migrar de fine-tuning de modelos transformer a prompting directo con LLMs preentrenados, sustituir clustering jerárquico por particionamiento k-means, o simplificar pipeline de limpieza de datos reduciendo etapas de normalización.

La cuarta acción correctiva aplica extensión controlada del cronograma como último recurso cuando calidad de entregables técnicos está severamente comprometida y amenaza viabilidad de defensa académica. El equipo podrá solicitar formalmente extensión de hasta 2 semanas adicionales, previa validación con el director del proyecto mediante reunión extraordinaria que presente evidencia de problemas críticos encontrados, ajuste formal del plan de trabajo con redistribución de actividades en tiempo extendido, y compromiso escrito de ambos integrantes de dedicación intensiva en período adicional.

Todas las acciones correctivas ejecutadas deberán quedar exhaustivamente documentadas en acta de reunión correspondiente, especificando problema técnico o administrativo detectado con métricas cuantitativas de desviación, decisión correctiva tomada con justificación de selección entre alternativas evaluadas, responsable designado de implementar la acción con accountability explícita, y plazo de ejecución con fecha límite verificable en siguiente sesión de seguimiento.

\section{Cierre del Proyecto}

El cierre del proyecto contempla un conjunto de actividades formales que garantizan la entrega completa de los resultados, la documentación adecuada de aprendizajes, la transferencia de conocimiento y la validación final por parte de los evaluadores designados.

\subsection{Entrega del Producto}

La entrega final del proyecto al asesor académico y a la universidad incluirá seis componentes técnicos y documentales complementarios que garantizan completitud funcional del sistema, reproducibilidad científica de resultados y transferencia efectiva de conocimiento generado durante el desarrollo.

El primer componente consiste en repositorio de código fuente funcional alojado en GitHub con estructura modular jerárquica que refleja arquitectura del sistema, código Python completamente comentado mediante docstrings en formato Google o NumPy, archivo README markdown con instrucciones detalladas de instalación y ejecución del sistema, archivo requirements.txt con dependencias especificando versiones exactas de bibliotecas utilizadas, y carpeta de ejemplos con scripts de demostración para ejecución del pipeline completo sobre datos sintéticos o muestra real reducida.

El segundo componente entrega dataset procesado exportado desde PostgreSQL mediante dump SQL comprimido conteniendo esquema completo de base de datos y datos de producción, incluyendo tablas de vacantes scrapeadas con metadata temporal y geográfica, habilidades extraídas por ambos pipelines con scores de confianza, vectores de embeddings de 768 dimensiones en formato binario eficiente, asignaciones de clustering con parámetros HDBSCAN utilizados, y diccionario de datos explicativo en formato markdown documentando esquema relacional, tipos de datos, restricciones de integridad y semántica de cada columna.

El tercer componente produce documento técnico final del trabajo de grado en formato PDF profesional de mínimo 60 páginas siguiendo plantilla institucional de la Pontificia Universidad Javeriana o formato IEEE Computer Society, conteniendo introducción contextualizando problema de observación de demanda laboral, metodología CRISP-DM adaptada con justificación de decisiones arquitectónicas, descripción detallada de arquitectura del sistema con diagramas BPMN y de componentes, explicación técnica exhaustiva de cada una de las siete fases con pseudocódigo de algoritmos críticos, presentación de resultados obtenidos con métricas cuantitativas y análisis cualitativo de outputs, validación empírica mediante comparación con gold standard y benchmarks de estado del arte, conclusiones sintetizando aportes del proyecto y lecciones aprendidas, y recomendaciones fundamentadas para trabajo futuro y escalamiento del observatorio.

El cuarto componente compila visualizaciones generadas y notebooks analíticos en carpeta organizada jerárquicamente, incluyendo gráficos de alta resolución en formatos PNG y PDF vectorial representando histogramas de distribución de habilidades más frecuentes, proyecciones UMAP de espacios de embeddings coloreadas por clústeres identificados, mapas de calor geográficos comparando perfiles tecnológicos por país, nubes de palabras ponderadas por frecuencia de términos técnicos, y notebooks Jupyter completamente ejecutables con análisis exploratorios de datos, experimentos de parametrización de clustering, y explicaciones técnicas didácticas paso a paso de decisiones metodológicas.

El quinto componente provee manual de usuario técnico conciso de 5 a 10 páginas dirigido a desarrolladores o investigadores que deseen replicar el sistema en infraestructura diferente, especificando requisitos mínimos de hardware incluyendo CPU, RAM, almacenamiento y GPU opcional, instrucciones paso a paso de instalación de dependencias mediante pip o conda en entorno virtual Python aislado, configuración de variables de entorno para credenciales de base de datos y paths de modelos preentrenados descargados, y secuencia de comandos para ejecutar cada fase del pipeline con parámetros por defecto validados y troubleshooting de errores comunes encontrados durante desarrollo.

El sexto componente documenta evidencias empíricas de validación del sistema mediante carpeta de artefactos técnicos, conteniendo logs de ejecución representativos de scraping exitoso y manejo de errores de red, métricas cuantitativas de evaluación tabuladas incluyendo precisión y recall del NER, F1-score Post-ESCO de ambos pipelines, coherencia interna de clústeres mediante Silhouette Score y DBCV, ejemplos concretos de habilidades enriquecidas semánticamente por Pipeline B con justificaciones textuales del LLM, y capturas de pantalla de alta resolución de visualizaciones funcionales ejecutadas sobre corpus completo de 30,660 ofertas laborales.

\subsection{Actividades de Cierre}

Al finalizar el desarrollo técnico del proyecto e integración de todos los componentes del observatorio, se ejecutarán seis actividades formales de cierre que garantizan completitud de entregables, documentación de lecciones aprendidas y preparación adecuada para defensa académica ante jurados evaluadores.

La primera actividad consiste en revisión final exhaustiva de entregables mediante checklist estructurada, donde el equipo verifica sistemáticamente que todos los componentes técnicos y documentales estén completos según especificación del plan de proyecto, funcionales mediante ejecución end-to-end del pipeline sobre datos reales sin errores críticos, y correctamente documentados con comentarios de código, README actualizado y manual de usuario validado mediante prueba de instalación en máquina limpia.

La segunda actividad ejecuta sesión de retrospectiva final del equipo con duración de 2 horas, donde ambos integrantes reflexionan colaborativamente sobre aprendizajes técnicos adquiridos en cada fase del CRISP-DM, dificultades técnicas y organizacionales enfrentadas con análisis de causas raíz, decisiones arquitectónicas acertadas que facilitaron desarrollo ágil y modular, y mejoras metodológicas aplicables a futuros proyectos académicos o profesionales similares. Los insights se documentan formalmente en acta de lecciones aprendidas archivada en repositorio del proyecto para consulta posterior.

La tercera actividad presenta sistema completo al asesor académico mediante sesión formal de demostración técnica de 2 horas, donde se expone funcionamiento end-to-end del observatorio ejecutando pipeline completo en vivo sobre muestra representativa de datos, se explican decisiones técnicas clave como selección de Gemma 3 4B sobre alternativas evaluadas y diseño de arquitectura híbrida de pipelines, y se responden preguntas técnicas y metodológicas del asesor sobre implementación de componentes, validación de resultados y alineación con objetivos académicos del trabajo de grado.

La cuarta actividad prepara defensa pública del trabajo de grado mediante elaboración colaborativa de presentación profesional de 20 a 30 minutos en formato PowerPoint o Beamer LaTeX, enfocada estratégicamente en comunicación clara de resultados técnicos obtenidos con métricas cuantitativas destacadas, procedimientos de validación empírica ejecutados sobre gold standard anotado, aportes metodológicos del proyecto como adaptación de CRISP-DM a contexto latinoamericano, y demostración visual impactante del sistema funcionando mediante screenshots y videos cortos de scraping, clustering y visualizaciones generadas.

La quinta actividad archiva formalmente el proyecto completo mediante subida de todos los entregables finales a repositorio institucional de la Pontificia Universidad Javeriana si protocolo aplica, y entrega de copia completa a la universidad en formatos solicitados por programa académico incluyendo documento PDF del trabajo de grado firmado digitalmente, código fuente comprimido en archivo ZIP con estructura de carpetas preservada, datasets procesados exportados como dumps SQL, y licencia de uso académico firmada autorizando uso interno con fines educativos y de investigación.

La sexta actividad opcional ejecuta liberación pública del código fuente del observatorio bajo licencia open source permisiva como MIT o Apache 2.0, condicionada a decisión consensuada del equipo considerando beneficio para comunidad técnica y académica, y autorización formal del asesor verificando que no existan restricciones institucionales o de propiedad intelectual que impidan publicación abierta. La liberación incluye documentación clara de instalación, datasets sintéticos de ejemplo para testing, y sección de contribuciones invitando a comunidad a reportar issues y proponer mejoras mediante pull requests de GitHub.

\subsection{Criterios de Aceptación Final}

El proyecto del Observatorio de Demanda Laboral en Tecnología para Latinoamérica será formalmente considerado completo y aceptable para defensa académica ante jurados cuando satisfaga cinco criterios de completitud técnica, funcionalidad sistémica, documentación exhaustiva, validación académica y cumplimiento cronológico.

El primer criterio requiere que todos los requerimientos funcionales prioritarios del sistema identificados en documento de especificación hayan sido implementados técnicamente en código Python funcional y validados mediante pruebas de aceptación ejecutadas sobre datos reales, confirmando que módulos de scraping, extracción dual mediante Pipeline A y B, generación de embeddings, clustering HDBSCAN y visualizaciones macro operan correctamente según especificaciones técnicas establecidas.

El segundo criterio establece que el sistema debe ser capaz de ejecutar pipeline completo de siete fases desde scraping inicial de portales hasta generación de visualizaciones finales sin errores críticos que detengan ejecución, aceptando únicamente warnings no fatales o errores manejados gracefully mediante try-except con logging apropiado, y completando procesamiento de corpus completo de 30,660 ofertas laborales en tiempo razonable sin memory leaks o deadlocks.

El tercer criterio exige que documentación técnica del proyecto esté completa cubriendo todas las fases metodológicas con descripción detallada de implementación, coherente manteniendo consistencia terminológica y de notación entre secciones, y suficientemente clara para permitir replicación exitosa del sistema por terceros con conocimientos técnicos de nivel medio en Python, NLP y bases de datos relacionales, validado mediante prueba empírica de instalación por persona externa al equipo.

El cuarto criterio condiciona aceptación final a aprobación formal del asesor del proyecto Ing. Luis Gabriel Moreno Sandoval sobre calidad técnica del sistema implementado evaluada mediante inspección de código y demostración funcional, y calidad académica del documento de trabajo de grado evaluada mediante revisión de profundidad metodológica, rigurosidad de análisis de resultados y alineación con estándares de investigación en Ingeniería de Sistemas de la Pontificia Universidad Javeriana.

El quinto criterio verifica que se hayan cumplido entregas programadas de entregables según cronograma ajustado del proyecto considerando cambios aprobados formalmente durante desarrollo, o alternativamente que extensiones de fechas aplicadas hayan sido justificadas formalmente mediante actas de reunión documentando causas de retrasos, aprobación escrita del asesor aceptando nueva calendarización, y evidencia de mitigación de riesgos identificados para evitar futuros incumplimientos.

Una vez satisfechos estos cinco criterios de aceptación mediante verificación documentada y aprobación formal del asesor académico, el proyecto podrá ser sometido a evaluación final por parte de los jurados designados por el Departamento de Sistemas en sesión de defensa pública programada según calendario académico institucional.

\chapter{Procesos de Soporte}

\section{Gestión de la Configuración}

La gestión de la configuración del proyecto tiene como propósito mantener la integridad, trazabilidad y control de versiones de todos los artefactos generados durante el desarrollo, incluyendo código fuente, datasets, documentación técnica, modelos entrenados, scripts de procesamiento y archivos de configuración. Dado el carácter académico del proyecto y su naturaleza modular, se adoptará un enfoque pragmático basado en Git, GitHub y herramientas colaborativas de documentación, sin pretender alcanzar niveles de formalidad propios de entornos empresariales regulados.

\subsection{Elementos de Configuración}

Los elementos que estarán bajo control de configuración incluyen:

\begin{table}[H]
\centering
\small
\begin{tabular}{|p{3.5cm}|p{5cm}|p{3cm}|p{2.5cm}|}
\hline
\textbf{Elemento de configuración} & \textbf{Descripción} & \textbf{Herramienta de gestión} & \textbf{Responsable} \\
\hline
Código fuente del sistema & Scripts de scraping, NER, LLMs, clustering y visualización en Python & GitHub & Nicolás \\
\hline
Configuraciones de entorno & Archivos .env, config.json, requirements.txt, docker-compose.yml & GitHub & Nicolás \\
\hline
Bases de datos y esquemas & Dump SQL de PostgreSQL con estructura de tablas y datos procesados & GitHub + Google Drive & Nicolás \\
\hline
Documentación técnica del proyecto & Documento SPMP, SRS, memoria técnica, manuales & Google Docs + Overleaf & Alejandro \\
\hline
Notebooks de análisis & Jupyter notebooks con pruebas exploratorias, validaciones y visualizaciones & GitHub & Nicolás \\
\hline
Datasets intermedios & Archivos CSV, JSON o pickle con datos procesados por fase & Google Drive & Nicolás \\
\hline
Modelos y embeddings & Archivos de modelos descargados o ajustados, vectores precomputados & Google Drive & Nicolás \\
\hline
Actas de reunión y seguimiento & Registros de reuniones semanales y decisiones de equipo & Google Docs & Alejandro \\
\hline
\end{tabular}
\caption{Elementos de Configuración del Proyecto}
\end{table}

\subsection{Proceso de Control de Versiones}

El control de versiones del código y de los artefactos técnicos se realizará mediante Git y GitHub, siguiendo las siguientes prácticas:

\begin{itemize}
    \item \textbf{Rama principal (main):} Contiene la versión estable del sistema, probada y validada al cierre de cada fase metodológica. Los commits a esta rama requieren revisión previa.

    \item \textbf{Ramas de desarrollo por fase (dev-fase-X):} Cada fase metodológica tendrá una rama temporal donde se desarrollarán las funcionalidades correspondientes. Al validarse técnicamente, se fusionará a main mediante pull request.

    \item \textbf{Commits descriptivos:} Todo commit debe tener un mensaje claro que indique qué cambio se realizó, por qué y en qué fase del proyecto se hizo. Formato sugerido: ``[FASE-X] Descripción breve del cambio''.

    \item \textbf{Versionado semántico para entregas:} Cada entregable mayor será etiquetado con un tag de versión (v0.1, v0.2, v1.0, etc.) para facilitar la trazabilidad histórica.

    \item \textbf{Sincronización diaria:} Los integrantes deberán hacer push de sus avances al menos una vez al día, para evitar conflictos de integración y facilitar la colaboración.
\end{itemize}

\subsection{Gestión de Cambios en la Configuración}

Cualquier cambio en la configuración del sistema (por ejemplo, modificación de estructura de base de datos, cambio de librerías clave, ajuste de arquitectura de scraping) deberá seguir este procedimiento:

\begin{enumerate}
    \item El integrante que propone el cambio documenta la razón técnica, el impacto esperado y las alternativas evaluadas.

    \item Se discute en reunión semanal si el cambio es necesario, viable y justificado.

    \item Si se aprueba, se implementa en una rama específica, se prueba localmente, y se documenta en el README o en comentarios del código.

    \item Se abre un pull request para revisión por parte de Nicolás antes de fusionar a la rama principal.

    \item Se actualiza el archivo de configuración correspondiente, se registra el cambio en el acta semanal y se comunica formalmente al equipo.
\end{enumerate}

\begin{figure}[H]
\centering
\includegraphics[width=0.95\textwidth]{figures/BPMNControldeCambios.png}
\caption{Proceso de Gestión de Cambios en la Configuración (BPMN)}
\label{fig:bpmn-cambios}
\end{figure}

\subsection{Backup y Recuperación}

Para garantizar la disponibilidad de los artefactos del proyecto ante pérdidas accidentales, se implementarán las siguientes medidas:

\begin{itemize}
    \item \textbf{GitHub como repositorio central:} Todo el código y documentación técnica se alojará en GitHub, que ofrece redundancia automática y permite recuperar versiones previas en caso de errores.

    \item \textbf{Google Drive compartido:} Los datasets grandes, modelos preentrenados, documentos en edición activa y archivos binarios se almacenarán en una carpeta compartida de Google Drive con acceso restringido al equipo.

    \item \textbf{Backups semanales locales:} Cada integrante mantendrá una copia local actualizada del repositorio completo, sincronizada semanalmente mediante git pull.

    \item \textbf{Exportaciones SQL periódicas:} La base de datos PostgreSQL será exportada (dump) al finalizar cada fase técnica, con fecha y versión identificada, y almacenada en GitHub y Google Drive.
\end{itemize}

En caso de pérdida de datos, se recurrirá al historial de GitHub o a los backups de Google Drive, según corresponda. Si un integrante pierde su copia local, podrá clonar nuevamente el repositorio completo y descargar los archivos grandes desde Drive.

\section{Aseguramiento de Calidad}

El aseguramiento de calidad tiene como objetivo garantizar que los entregables del proyecto cumplan con los estándares técnicos esperados, funcionen correctamente en distintos escenarios de uso, y estén adecuadamente documentados para facilitar su comprensión, validación y réplica por terceros. Dado el contexto académico, se priorizará la validación funcional, la coherencia metodológica y la transparencia técnica por encima de métricas formales de calidad de software empresarial.

\subsection{Estándares de Calidad Aplicados}

Se aplicarán los siguientes estándares de calidad técnica en el desarrollo del proyecto:

\begin{itemize}
    \item \textbf{PEP 8 (Python):} Todo el código Python seguirá las convenciones de estilo de PEP 8, incluyendo nombres de variables descriptivos, separación clara de funciones, indentación de 4 espacios y límite de 100 caracteres por línea cuando sea posible.

    \item \textbf{Modularidad y reutilización:} Cada fase del pipeline será implementada como un módulo independiente con entradas y salidas claramente definidas, facilitando la depuración, el testing y la reutilización futura.

    \item \textbf{Documentación interna del código:} Funciones complejas deberán incluir docstrings explicativas en español, indicando propósito, parámetros de entrada, salida esperada y ejemplo de uso.

    \item \textbf{Manejo de errores:} Se implementarán validaciones básicas de entrada, manejo de excepciones con mensajes claros, y logging de eventos críticos para facilitar el diagnóstico de problemas.

    \item \textbf{Reproducibilidad técnica:} Todos los procesos estarán documentados con suficiente detalle como para que un tercero con conocimientos técnicos medios pueda replicar el pipeline completo.
\end{itemize}

\subsection{Actividades de Verificación y Validación}

Las actividades de verificación (¿construimos el sistema correctamente?) y validación (¿construimos el sistema correcto?) se llevarán a cabo de forma iterativa en cada fase del proyecto:

\begin{table}[H]
\centering
\small
\begin{tabular}{|p{3cm}|p{4.5cm}|p{4cm}|p{2.5cm}|}
\hline
\textbf{Fase} & \textbf{Actividad de verificación} & \textbf{Actividad de validación} & \textbf{Responsable} \\
\hline
Scraping & Revisar que las vacantes extraídas tengan estructura completa y datos válidos & Comparar muestra manual con resultados del scraper para verificar precisión & Nicolás \\
\hline
NER y regex & Ejecutar el script sobre un conjunto de prueba y verificar que extrae habilidades sin errores de sintaxis & Revisar manualmente 50 vacantes y evaluar si las habilidades extraídas son correctas y relevantes & Alejandro \\
\hline
LLMs & Probar distintos prompts y verificar que el formato de salida es consistente & Evaluar cualitativamente si las habilidades enriquecidas tienen sentido técnico y semántico & Nicolás \\
\hline
Embeddings y clustering & Verificar que las dimensiones de los vectores son correctas y que HDBSCAN no arroja errores & Inspeccionar visualmente los clusters generados y verificar que agrupan habilidades coherentes & Nicolás \\
\hline
Visualización & Comprobar que los gráficos se generan sin errores y se exportan correctamente & Revisar con el asesor si las visualizaciones comunican información relevante de forma clara & Alejandro \\
\hline
\end{tabular}
\caption{Actividades de Verificación y Validación por Fase}
\end{table}

\begin{figure}[H]
\centering
\includegraphics[width=0.95\textwidth]{figures/BPMNControlCalidad.png}
\caption{Proceso de Control de Calidad del Proyecto (BPMN)}
\label{fig:bpmn-calidad}
\end{figure}

\subsection{Revisión Técnica de Entregables}

Al finalizar cada fase metodológica, se realizará una revisión técnica formal del entregable correspondiente, siguiendo este procedimiento:

\begin{enumerate}
    \item El responsable de la fase presenta el entregable al equipo completo en reunión semanal, explicando la implementación técnica, las decisiones tomadas y los resultados obtenidos.

    \item Los demás integrantes revisan el código, prueban el módulo localmente, y formulan preguntas técnicas o identifican posibles mejoras.

    \item Se redacta un checklist de verificación con criterios mínimos de aceptación (por ejemplo: ``¿El código ejecuta sin errores?'', ``¿Los datos de salida tienen la estructura esperada?'', ``¿Está documentado el proceso?'').

    \item Si el entregable cumple todos los criterios, se aprueba y se fusiona a la rama principal. Si no, se registra una lista de correcciones pendientes y se establece un plazo para aplicarlas.

    \item El asesor del proyecto revisa el entregable en la reunión quincenal de seguimiento y valida que esté alineado con los objetivos técnicos y académicos del proyecto.
\end{enumerate}

\subsection{Métricas de Calidad}

Se emplearán las siguientes métricas cualitativas y cuantitativas para evaluar la calidad del sistema:

\begin{itemize}
    \item \textbf{Cobertura de requisitos funcionales:} Porcentaje de requisitos implementados respecto al total definido en el documento SRS.

    \item \textbf{Tasa de errores en validación manual:} Proporción de habilidades extraídas o enriquecidas que son incorrectas o irrelevantes, evaluada sobre una muestra de 100 registros.

    \item \textbf{Coherencia de clusters:} Evaluación cualitativa de si los grupos de habilidades generados tienen sentido semántico y técnico, mediante inspección visual y discusión con el asesor.

    \item \textbf{Complejidad ciclomática moderada:} Funciones con complejidad razonable, evitando bloques de código excesivamente largos o anidados.

    \item \textbf{Legibilidad del código:} Evaluación subjetiva de si el código es comprensible para un desarrollador externo, verificada mediante revisión cruzada entre integrantes.
\end{itemize}

\section{Gestión de Riesgos}

La gestión de riesgos del proyecto tiene como objetivo identificar de forma anticipada las amenazas potenciales que puedan afectar el cronograma, la calidad técnica, los recursos disponibles o el cumplimiento de los objetivos, y definir estrategias de mitigación y contingencia para minimizar su impacto en caso de materializarse.

\subsection{Identificación de Riesgos}

A continuación, se presentan los principales riesgos identificados para el proyecto, clasificados por categoría:

\begin{table}[H]
\centering
\small
\begin{tabular}{|c|p{5cm}|c|c|p{4cm}|}
\hline
\textbf{ID} & \textbf{Descripción del riesgo} & \textbf{Prob.} & \textbf{Impacto} & \textbf{Categoría} \\
\hline
R01 & Cambios en la estructura HTML de portales de empleo que rompan el scraper & Media & Alto & Técnico \\
\hline
R02 & Bloqueo o limitación de acceso por parte de los portales web (anti-scraping) & Media & Alto & Técnico \\
\hline
R03 & Rendimiento insuficiente del modelo NER en español para el dominio laboral & Media & Medio & Técnico \\
\hline
R04 & Falta de datos suficientes o de calidad en portales de algunos países & Baja & Alto & Técnico \\
\hline
R05 & Complejidad técnica excesiva en implementación de clustering semántico & Media & Medio & Técnico \\
\hline
R06 & Sobrecarga académica de los integrantes afectando disponibilidad semanal & Alta & Medio & Organizacional \\
\hline
R07 & Retrasos acumulados que comprometan el cronograma general del proyecto & Media & Alto & Organizacional \\
\hline
R08 & Falta de coordinación o conflictos internos en el equipo & Baja & Medio & Organizacional \\
\hline
R09 & Fallas técnicas en equipos personales (hardware, conectividad, software) & Baja & Medio & Infraestructura \\
\hline
R10 & Pérdida de datos por falta de backups adecuados & Baja & Alto & Infraestructura \\
\hline
\end{tabular}
\caption{Identificación y Clasificación de Riesgos}
\end{table}

\begin{figure}[H]
\centering
\includegraphics[width=0.95\textwidth]{figures/BPMNIdentificaciondeRiesgos.png}
\caption{Proceso de Identificación y Gestión de Riesgos (BPMN)}
\label{fig:bpmn-riesgos}
\end{figure}

\subsection{Análisis de Riesgos}

Cada riesgo ha sido evaluado en términos de probabilidad de ocurrencia (Baja, Media, Alta) e impacto en el proyecto (Bajo, Medio, Alto), permitiendo priorizarlos según su nivel de criticidad.

Los riesgos de mayor criticidad (probabilidad media-alta e impacto alto) son:

\begin{itemize}
    \item \textbf{R01 y R02:} Problemas con el scraping, que afectarían directamente la disponibilidad de datos y podrían paralizar las fases posteriores.
    \item \textbf{R06 y R07:} Problemas organizacionales relacionados con disponibilidad de tiempo y cumplimiento del cronograma, que podrían generar retrasos en cascada.
\end{itemize}

Estos riesgos recibirán especial atención en las estrategias de mitigación y monitoreo continuo.

\subsection{Estrategias de Mitigación}

Para cada riesgo prioritario, se definen las siguientes estrategias de mitigación:

\begin{table}[H]
\centering
\small
\begin{tabular}{|c|p{5.5cm}|p{5.5cm}|}
\hline
\textbf{ID} & \textbf{Estrategia de mitigación (preventiva)} & \textbf{Plan de contingencia (reactiva)} \\
\hline
R01 & Diseñar scrapers modulares y flexibles. Revisar semanalmente la estructura de los portales durante la fase de scraping. & Si un portal cambia, ajustar el scraper en un plazo máximo de 3 días. Si no es viable, reemplazar por otro portal del mismo país. \\
\hline
R02 & Implementar delays aleatorios entre peticiones, rotar user agents, respetar robots.txt y usar Selenium/Playwright cuando sea necesario. & Si un portal bloquea el acceso, cambiar a scraping manual asistido o buscar datasets públicos alternativos (APIs, Kaggle). \\
\hline
R03 & Probar varios modelos NER en español (spaCy, BETO, modelos de HuggingFace) en etapa temprana. Validar con muestras pequeñas antes de procesar todo el corpus. & Si el rendimiento es bajo, complementar con expresiones regulares ad-hoc, diccionarios de habilidades predefinidos, o ajuste manual de resultados. \\
\hline
R04 & Validar acceso a portales y disponibilidad de vacantes antes de iniciar scraping masivo. Tener al menos 2 portales por país como respaldo. & Si un país tiene datos insuficientes, reducir su alcance o reemplazarlo por otro país latinoamericano con mayor disponibilidad. \\
\hline
R05 & Comenzar con técnicas simples (K-Means, DBSCAN) antes de pasar a HDBSCAN. Realizar pruebas con datasets sintéticos pequeños. & Si HDBSCAN no converge o genera clusters poco útiles, reducir dimensionalidad con PCA en lugar de UMAP, o aplicar clustering jerárquico tradicional. \\
\hline
R06 & Planificar cronograma considerando semanas de exámenes y entregas paralelas. Distribuir carga de trabajo de forma flexible. & Si un integrante tiene sobrecarga puntual, redistribuir tareas urgentes entre el resto del equipo temporalmente. \\
\hline
R07 & Hacer seguimiento semanal estricto del cronograma. Detectar retrasos tempranamente y aplicar correcciones inmediatas. & Si el retraso es mayor a 2 semanas, reducir el alcance de fases menos críticas o solicitar extensión formal del plazo final. \\
\hline
R09 & Mantener backups semanales en GitHub y Google Drive. Documentar configuraciones de entorno en README. & Si un equipo falla, el integrante afectado podrá clonar el repositorio completo en otro equipo y continuar trabajando con mínima pérdida de tiempo. \\
\hline
R10 & Implementar sistema de backups automáticos semanales en GitHub y Google Drive. Verificar integridad de backups mensualmente. & Si hay pérdida de datos, restaurar desde el último backup disponible. Si no hay backup, repetir el trabajo perdido priorizando lo más crítico. \\
\hline
\end{tabular}
\caption{Estrategias de Mitigación y Planes de Contingencia}
\end{table}

\subsection{Monitoreo de Riesgos}

El seguimiento de riesgos se realizará de forma continua durante todo el proyecto, mediante las siguientes actividades:

\begin{itemize}
    \item \textbf{Revisión semanal de riesgos:} En cada reunión de seguimiento, se dedicará un espacio breve (10 minutos) para revisar si algún riesgo identificado se ha materializado, si han surgido nuevos riesgos, y si las estrategias de mitigación están funcionando.

    \item \textbf{Indicadores de alerta temprana:} Se monitorearán señales que indiquen la proximidad de riesgos, tales como:
    \begin{itemize}
        \item Errores frecuentes en el scraper (indicador de R01)
        \item Aumento en tiempo de respuesta de portales o CAPTCHAs (indicador de R02)
        \item Bajo rendimiento en métricas de NER (indicador de R03)
        \item Retrasos superiores a 3 días en tareas asignadas (indicador de R06 y R07)
    \end{itemize}

    \item \textbf{Registro de incidentes:} Cualquier materialización de riesgo será documentada en el acta semanal, indicando el riesgo activado, la acción correctiva aplicada, el responsable y el resultado obtenido.

    \item \textbf{Actualización de la matriz de riesgos:} Si durante el proyecto se identifica un nuevo riesgo relevante, se agregará a la tabla de riesgos con su respectiva evaluación y estrategia de mitigación.
\end{itemize}

\subsection{Responsabilidades en Gestión de Riesgos}

\begin{itemize}
    \item \textbf{Nicolás Camacho (Líder técnico):} Responsable de identificar y monitorear riesgos técnicos (R01–R05), proponer soluciones técnicas y coordinar la aplicación de estrategias de mitigación.

    \item \textbf{Alejandro Pinzón (Coordinador de proyecto):} Responsable de monitorear riesgos organizacionales (R06–R08), gestionar el cronograma, detectar desviaciones y proponer acciones correctivas.

    \item \textbf{Todo el equipo:} Responsable de reportar cualquier riesgo detectado, participar en la evaluación de impacto, y colaborar en la implementación de planes de contingencia cuando sea necesario.
\end{itemize}

La gestión de riesgos será un proceso continuo y colaborativo, integrado a las actividades semanales del proyecto, con el objetivo de minimizar incertidumbre y maximizar la probabilidad de éxito en la entrega final del observatorio laboral.


% ============================================================================
% REFERENCIAS
% ============================================================================
\chapter*{BIBLIOGRAFÍA}
\addcontentsline{toc}{chapter}{BIBLIOGRAFÍA}
\printbibliography[heading=none]

\end{document}
