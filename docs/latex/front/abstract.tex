\thispagestyle{fancy}
\fancyfoot[R]{Page vii}

\begin{center}
    {\large\bfseries ABSTRACT}
\end{center}

\vspace{1cm}

El desajuste entre las habilidades demandadas por el mercado y la oferta formativa en Latinoamérica dificultaba decisiones de política, academia y empresa. Este proyecto abordó el problema construyendo un observatorio automatizado que recolectó avisos de empleo multi-portal y multi-país, escalable hacia $\sim$600.000 registros. Se integraron spiders (Scrapy/Selenium con anti-detección), una base PostgreSQL con pgvector y un pipeline de extracción/normalización de habilidades (NER/regex con apoyo LLM) alineadas a ESCO. El sistema generó indicadores, consultas y visualizaciones reproducibles, entregando evidencia comparable por país, sector y tiempo para orientar currículos, formación y estrategias de talento.
