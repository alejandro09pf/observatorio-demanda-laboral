\section{Restricciones}

Estas son limitaciones específicas que afectan la capacidad del sistema para cumplir con ciertos requisitos o estándares, y deben ser consideradas durante el desarrollo y evaluación.

\subsection{Restricciones Computacionales}

\subsubsection{C-3. Hardware Disponible y Recursos Computacionales}

\begin{itemize}
    \item Servidores universitarios con CPU Intel Xeon (16 cores) / AMD Ryzen (8 cores)
    \item RAM: 16-32 GB (compartida con otros procesos)
    \item GPU: NVIDIA GTX 1660 / RTX 3060 con al menos 8GB VRAM requeridos para modelos de 7B parámetros, o Apple MPS
    \item Almacenamiento: 100-200 GB cuota en servidores universitarios
\end{itemize}

Implicaciones:
\begin{itemize}
    \item LLMs ejecutables localmente requieren GPU con al menos 8GB VRAM para modelos de 7B parámetros con cuantización Q4
    \item Procesamiento batch preferido sobre tiempo real
    \item FAISS en modo CPU (suficiente para 14K vectores)
    \item Imposibilidad de usar modelos grandes (GPT-4, Llama 70B) sin infraestructura adicional
\end{itemize}

\subsection{Restricciones de Tiempo}

\subsubsection{Cronograma Académico}

\begin{itemize}
    \item Tesis de pregrado: 6-12 meses (2 semestres)
    \item Tiempo efectivo de desarrollo: ~4-6 meses (clases + otras asignaturas)
    \item Deadline inflexible para defensa de grado
\end{itemize}

Implicaciones:
\begin{itemize}
    \item Priorización de Pipeline A (tradicional) sobre Pipeline B (experimental)
    \item Exploraciones de hiperparámetros limitadas (no exhaustivas)
    \item Validación cualitativa sobre subconjunto representativo
    \item Implementación incremental con entregas funcionales iterativas
\end{itemize}

\subsection{Restricciones Técnicas (continuación)}

\subsubsection{C-4. Latencia de Procesamiento LLM}

El procesamiento de ofertas con LLMs ejecutados localmente introduce latencia significativa (40-45 segundos/oferta) comparado con métodos tradicionales. Alternativamente, las llamadas a APIs de LLMs comerciales (OpenAI, Anthropic) reducirían latencia pero introducirían costos variables y dependencias externas. El diseño debe balancear calidad de resultados con viabilidad técnica y tiempos de procesamiento.

\subsection{Restricciones de Datos}

\subsubsection{C-1. Límites de Scraping y C-2. Dinamismo del DOM}

\begin{itemize}
    \item \textbf{C-1}: Los portales de empleo implementan medidas anti-bot (CAPTCHAs, rate limiting, bloqueos por IP) que restringen la velocidad y volumen de recolección. El sistema debe respetar estas limitaciones mediante delays adaptativos, rotación de user-agents y estrategias de backoff exponencial.
    \item \textbf{C-2}: La estructura HTML de los portales cambia frecuentemente sin previo aviso, lo que genera fragilidad en los selectores CSS/XPath. El sistema debe incluir monitoreo de fallos y mecanismos de alerta para intervención manual cuando los spiders dejan de funcionar.
    \item Rate limiting: 1-2 requests/segundo por portal
    \item Bloqueo de IPs ante comportamiento sospechoso
    \item Contenido JavaScript requiere Selenium (más lento)
\end{itemize}

\subsubsection{C-5. Heterogeneidad de Formatos y C-6. Incompletitud de Información}

\begin{itemize}
    \item \textbf{C-5}: No existe un estándar para la publicación de ofertas laborales. Los portales utilizan campos, nomenclaturas y niveles de detalle diferentes, lo que dificulta la normalización automática.
    \item \textbf{C-6}: Muchas ofertas omiten información relevante (salario, requisitos detallados, tecnologías específicas), limitando la profundidad del análisis para ciertos campos.
\end{itemize}

\subsubsection{C-7. Ruido Lingüístico}

Las ofertas contienen errores ortográficos, abreviaciones no estándar, mezcla de idiomas y uso informal del lenguaje, lo que reduce la efectividad de técnicas de NLP basadas en corpus formales.

\begin{itemize}
    \item Ofertas en español (España + LatAm) y Spanglish técnico
    \item Modelos NLP optimizados para español de España
    \item Escasa literatura sobre NLP para español técnico latinoamericano
\end{itemize}

\subsubsection{C-8. Volatilidad Temporal}

Las ofertas se eliminan o modifican frecuentemente (típicamente tienen vigencia de 30-60 días), lo que requiere estrategias de recolección periódica y versionado de datos.

\begin{itemize}
    \item Dataset actual: marzo-diciembre 2024 (9 meses)
    \item Análisis de tendencias de largo plazo limitado
    \item Imposibilidad de comparar con años anteriores
\end{itemize}

\subsection{Cumplimiento Normativo}

\subsubsection{Protección de Datos}

El sistema debe cumplir con regulaciones legales y normativas vigentes en los tres países objetivo:

\begin{itemize}
    \item Colombia: Ley 1581 de 2012 - Tratamiento de datos personales
    \item México: Ley Federal de Protección de Datos Personales
    \item Argentina: Ley 25.326 - Protección de Datos Personales
\end{itemize}

Medidas implementadas:
\begin{itemize}
    \item Anonimización de datos: No se almacenan emails, teléfonos, nombres de candidatos
    \item Datos scrapeados son públicos (ofertas laborales visibles sin login)
    \item Uso exclusivo con fines académicos e investigación
    \item No comercialización de datos recolectados
    \item Eliminación de información sensible (salarios detallados)
\end{itemize}

\subsection{Restricciones de Taxonomías}

\subsubsection{ESCO v1.1.0}

\begin{itemize}
    \item Versión desactualizada (2016-2017)
    \item Enfoque europeo (menor cobertura de tech LatAm)
    \item No incluye frameworks modernos (Next.js, Remix, SolidJS)
    \item Actualizaciones oficiales lentas (años)
\end{itemize}

Mitigación:
\begin{itemize}
    \item Expansión manual con 152 O*NET + 83 curated skills
    \item Skills emergentes catalogadas para futura integración
    \item Análisis cualitativo de skills no matched
\end{itemize}

\subsection{Restricciones Metodológicas}

\subsubsection{C-9. Ausencia de Ground Truth}

No existe un dataset etiquetado de referencia para habilidades en ofertas laborales en español latinoamericano, lo que dificulta la evaluación cuantitativa rigurosa de los modelos de extracción.

\begin{itemize}
    \item Presupuesto limitado para anotadores profesionales
    \item Anotación manual limitada a 300 ofertas (1.3\% del corpus)
    \item Anotadores: estudiantes de ingeniería (no expertos en RRHH)
    \item Sesgo potencial hacia perfiles técnicos conocidos
\end{itemize}

\subsubsection{C-10. Sesgo de Fuente}

Los portales de empleo no representan el universo completo del mercado laboral. Excluyen ofertas publicadas en sitios corporativos directos, redes sociales, o canales informales, introduciendo sesgo de formalidad y tamaño de empresa.

\subsubsection{Validación de Clustering}

\begin{itemize}
    \item No existen ground truth labels para clústeres de skills
    \item Evaluación cualitativa subjetiva
    \item Métricas intrínsecas (silhouette, DBCV) solo aproximadas
\end{itemize}

\subsection{Restricciones de Publicación Académica}

\subsubsection{Requisitos Universitarios}

\begin{itemize}
    \item Documento de tesis debe seguir formato institucional (LaTeX PUJ)
    \item Extensión limitada: 80-120 páginas
    \item No se puede publicar código con licencias restrictivas
    \item Resultados deben ser originales (no publicados previamente)
\end{itemize}

Implicaciones:
\begin{itemize}
    \item Documentación técnica detallada en repositorio GitHub
    \item Código abierto con licencia MIT
    \item Publicación en conferencias académicas después de defensa de grado
\end{itemize}

\subsection{Resumen de Restricciones}

La Tabla \ref{tab:resumen-restricciones} presenta un resumen consolidado de las principales restricciones del proyecto.

\begin{table}[H]
\centering
\caption{Resumen de Restricciones del Proyecto}
\label{tab:resumen-restricciones}
\begin{tabular}{|p{3.5cm}|p{9cm}|}
\hline
\textbf{Categoría} & \textbf{Restricción Principal} \\
\hline
Computacional & Hardware limitado: 16-32GB RAM, GPU mínimo 8GB VRAM para modelos 7B params \\
\hline
Temporal & 6-12 meses para tesis completa, 4-6 meses desarrollo efectivo \\
\hline
Datos & Rate limiting 1-2 req/s, dataset 9 meses (mar-dic 2024) \\
\hline
Legal & Cumplimiento Ley 1581/2012 (CO), LFPDP (MX), Ley 25.326 (AR) \\
\hline
Taxonomías & ESCO v1.1.0 desactualizada (2016-2017), enfoque europeo \\
\hline
Evaluación & Gold Standard limitado a 300 ofertas (1.3\% corpus) \\
\hline
Publicación & Formato LaTeX PUJ, 80-120 páginas, código MIT \\
\hline
\end{tabular}
\end{table}

Estas restricciones han sido consideradas en el diseño arquitectónico del sistema y las decisiones de implementación, priorizando soluciones viables dentro de los límites del proyecto académico.
