\chapter{METODOLOGÍA}

El proyecto combinó CRISP-DM con un esquema modular iterativo para manejar, a la vez, la complejidad analítica de la extracción de habilidades y la construcción técnica de un sistema por componentes. CRISP-DM aportó la estructura del ciclo de vida, mientras que la modularidad permitió desarrollar, probar y mejorar cada módulo antes de integrarlo al pipeline completo.

El proceso inició con la clarificación del problema: la ausencia de herramientas automatizadas capaces de identificar habilidades técnicas emergentes en el mercado laboral latinoamericano. Mediante revisión documental, análisis de vacantes y estudio del estado del arte, se definieron objetivos, métricas y criterios de éxito que guiaron el diseño metodológico y el alcance del sistema.

Luego se realizó la caracterización de fuentes de datos, identificando variabilidad entre portales, mezcla frecuente de español e inglés y falta de estandarización. Esto justificó la creación de un conjunto de referencia anotado y la adopción de ESCO como taxonomía base para la normalización de habilidades.

La preparación de datos se desarrolló de forma iterativa, mediante módulos de scraping, limpieza, normalización y deduplicación sometidos a ciclos de inspección manual, pruebas controladas y verificación de coherencia. En paralelo, se construyó un gold standard de 300 vacantes para evaluar formalmente los modelos.

Siguiendo la misma lógica incremental del modelo con dos pipelines complementarios, el primero sinedo técnicas tradicionales basadas en reglas, regex y reconocimiento de entidades y el segundo se establecio como modelos de lenguaje con mapeo semántico hacia ESCO. Cada iteración se validó contra el gold standard, permitiendo mejorar precisión, cobertura y coherencia en la extracción.

La evaluación del sistema combinó métricas cuantitativas, revisión cualitativa y análisis de robustez operativa, lo que abrió nuevos ciclos de refinamiento y fortaleció la calidad global del pipeline.

Finalmente, todos los componentes se integraron en una herramienta operativa, con ejecución automática, CLI y documentación completa, validada en un entorno real para asegurar estabilidad, trazabilidad y operación recurrente sin supervisión.

\section{Diagrama de Flujo Metodológico}

El flujo metodológico del proyecto integró las seis fases de CRISP-DM (Business Understanding, Data Understanding, Data Preparation, Modeling, Evaluation, Deployment) con las siete etapas del pipeline de software mediante ciclos iterativos de refinamiento. Cada fase de CRISP-DM se tradujo en actividades concretas de desarrollo de software, estableciendo un proceso sistemático que equilibró diseño metodológico y evolución técnica. Los resultados de la fase de Evaluación retroalimentaron continuamente la fase de Modelado, generando versiones mejoradas de los pipelines de extracción mediante ajuste de parámetros, refinamiento de reglas lingüísticas y optimización de prompts para LLMs.

\begin{figure}[H]
\centering
\includegraphics[width=0.75\textwidth]{diagrams/DiagramaMetodologia.png}
\caption{Flujo Metodológico del Proyecto: Integración de CRISP-DM con Pipeline de 7 Etapas}
\label{fig:flujo-metodologico}
\end{figure}
