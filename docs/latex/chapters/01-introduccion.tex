\chapter{INTRODUCCIÓN}

El mercado laboral tecnológico en América Latina atraviesa una transformación profunda impulsada por la digitalización de la economía. La pandemia de COVID-19 aceleró este proceso, intensificando la demanda de habilidades técnicas IT especializadas (lenguajes de programación, frameworks, herramientas cloud/DevOps, metodologías ágiles) y exponiendo las brechas de capital humano en la región \cite{azuara2022}. En este escenario, identificar con precisión qué competencias técnicas están siendo requeridas por el mercado se ha vuelto estratégico para gobiernos que diseñan políticas de formación, instituciones educativas que ajustan sus programas, y profesionales que planifican su desarrollo de carrera en el sector tecnológico.

Sin embargo, medir esta demanda de manera sistemática presenta desafíos importantes. Los portales de empleo en la región publican vacantes en formatos heterogéneos, sin vocabularios estandarizados, y con alta volatilidad \cite{echeverria2022tecnica}. Las encuestas tradicionales, aunque valiosas, suelen ser retrospectivas y de baja periodicidad, limitando su utilidad para capturar tecnologías emergentes que evolucionan más rápido que los ciclos de actualización de instrumentos de medición \cite{rubio2025}. Los estudios previos en países como Colombia, México y Argentina han aportado evidencia empírica importante, pero se han basado principalmente en análisis de frecuencia de términos y clasificaciones manuales, enfoques que no logran capturar habilidades implícitas inferibles del contexto ni identificar tecnologías emergentes ausentes en taxonomías oficiales \cite{aguilera2018, campos2024}.

Este proyecto evaluó la viabilidad técnica de construir un observatorio automatizado de demanda laboral tecnológica que recolectó 30,660 ofertas de empleo de Colombia, México y Argentina mediante web scraping de siete portales. Se implementó Pipeline A basado en NER y expresiones regulares para procesamiento escalable del corpus completo (latencia 0.97s por oferta), y se desarrolló Pipeline B experimental con LLM Gemma 3 4B evaluado sobre un gold standard de 300 ofertas anotadas manualmente con 6,174 hard skills técnicas. La comparación rigurosa mediante evaluación dual Pre-ESCO y Post-ESCO demostró que modelos de lenguaje aproximan mejor el mapeo humano de competencias (F1=84.26\% vs 72.53\% de métodos tradicionales), capturando habilidades implícitas inferibles del contexto y tecnologías emergentes que métodos basados en patrones omiten. El sistema normalizó extracciones contra taxonomía ESCO v1.1.0 extendida con 276 habilidades técnicas modernas (152 O*NET + 124 curadas manualmente) mediante matching de dos capas (exacto + difuso), y aplicó clustering no supervisado con UMAP y HDBSCAN identificando entre 34 y 53 familias tecnológicas en configuraciones optimizadas, sin categorías predefinidas.

Las contribuciones principales de este proyecto son cuatro. Primero, la validación empírica de viabilidad técnica de observatorios automatizados basados en web scraping multi-portal y multi-país para caracterización de demanda laboral tecnológica a escala regional. Segundo, evidencia cuantitativa de superioridad de modelos de lenguaje sobre métodos deterministas para aproximar juicio humano en extracción de habilidades técnicas (mejora de +11.73pp en F1-Score), incluyendo capacidad de inferir competencias implícitas y capturar tecnologías emergentes, con caracterización explícita de trade-off en costo computacional (hasta 43$\times$ mayor latencia). Tercero, identificación de limitaciones críticas de taxonomías internacionales para vocabulario IT moderno, evidenciando mediante experimentación con matcher evolutivo qué proporción significativa de extracciones corresponde a tecnologías ausentes en estándares oficiales como ESCO v1.1.0. Cuarto, metodología de evaluación dual Pre-ESCO y Post-ESCO sobre gold standard anotado manualmente, permitiendo comparación sistemática de pipelines distinguiendo capacidad de extracción pura versus alineación con taxonomías controladas.
