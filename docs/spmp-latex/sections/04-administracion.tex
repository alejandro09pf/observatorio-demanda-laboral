\chapter{Administración del Proyecto}

\section{Métodos y Herramientas de Estimación}

El presente proyecto ha sido estimado utilizando una combinación de métodos empíricos y de juicio experto, apoyados en la experiencia previa de los integrantes del equipo, la naturaleza del trabajo requerido, y la complejidad técnica de cada fase. Dado que no se cuenta con herramientas formales de estimación como PERT o COCOMO, se optó por un enfoque ágil y práctico, ajustado a las condiciones reales del equipo.

\subsection{Método de Estimación del Proyecto}

El presente proyecto se apoya en una estimación realista de esfuerzos basada en tres principios clave que garantizan viabilidad operativa y académica del cronograma propuesto. El primer principio corresponde al análisis detallado de tareas específicas por fase metodológica, descomponiendo cada etapa del pipeline CRISP-DM en actividades concretas con complejidad técnica estimable. El segundo principio considera la disponibilidad horaria semanal efectiva del equipo, balanceando la dedicación al proyecto con otras responsabilidades curriculares del programa de Ingeniería de Sistemas. El tercer principio integra la experiencia práctica acumulada por los integrantes en proyectos previos similares de scraping, procesamiento de texto y análisis de datos, permitiendo estimaciones fundamentadas empíricamente. Dado que se trata de un trabajo de grado con duración prevista de entre 14 y 16 semanas y un equipo conformado por dos estudiantes con roles complementarios, se estimó un rango total de 440 a 500 horas de trabajo distribuidas entre ambos integrantes según especialización técnica y responsabilidades asignadas.

\subsubsection{Método de estimación utilizado}

Para esta estimación se empleó técnica de descomposición jerárquica de trabajo conocida como Work Breakdown Structure o WBS, donde cada fase del proyecto fue sistemáticamente dividida en tareas concretas y medibles, a las cuales se asignaron horas aproximadas según complejidad algorítmica, herramientas requeridas, curva de aprendizaje esperada y experiencia previa documentada del equipo en componentes similares. Esta estimación bottom-up fue posteriormente contrastada con la disponibilidad semanal esperada de cada integrante mediante proyección de calendario académico, ajustada conservadoramente por compromisos académicos paralelos incluyendo otras asignaturas, evaluaciones y trabajos grupales que compiten por tiempo efectivo de dedicación.

El enfoque adoptado es cualitativo en naturaleza, iterativo en ejecución y conservador en supuestos de productividad. En lugar de aplicar fórmulas matemáticas complejas de estimación como COCOMO que requieren calibración histórica de proyectos previos con métricas detalladas de líneas de código y factores de complejidad, se optó por estimación fundamentada en experticia directa del equipo técnico, validada mediante comparación con planes de proyectos académicos previos similares y distribuida proporcionalmente según carga de trabajo balanceada entre ambos integrantes considerando fortalezas individuales.

\subsubsection{Estimación del esfuerzo por integrante}

\begin{table}[H]
\centering
\begin{tabular}{|l|l|c|}
\hline
\textbf{Integrante} & \textbf{Rol principal} & \textbf{Horas estimadas} \\
\hline
Nicolás Camacho & Líder técnico, arquitectura & 170 \\
\hline
Alejandro Pinzón & Documentación y pruebas & 130 \\
\hline
Alejandro Pinzón & Desarrollo de módulos & 140 \\
\hline
\textbf{Total estimado} & & \textbf{440 horas} \\
\hline
\end{tabular}
\caption{Tiempo de Esfuerzo por Integrante}
\end{table}

\subsubsection{Estimación por fase metodológica}

\begin{table}[H]
\centering
\begin{tabular}{|l|c|}
\hline
\textbf{Fase} & \textbf{Horas estimadas} \\
\hline
Diseño inicial del sistema & 45 \\
\hline
Scraping y carga a base de datos & 70 \\
\hline
NER y normalización de habilidades & 60 \\
\hline
Enriquecimiento con LLMs & 60 \\
\hline
Embeddings y clustering & 50 \\
\hline
Visualización macro evaluativa & 40 \\
\hline
Documentación, validación y ajustes finales & 70 \\
\hline
\textbf{Total} & \textbf{395 horas} \\
\hline
\end{tabular}
\caption{Tiempo Estimado por Fase}
\end{table}

\textit{Nota:} El total por fases metodológicas no considera horas de reuniones, organización, imprevistos o revisión con el asesor, por lo que se reserva un margen adicional de 45 a 55 horas para estas actividades transversales de coordinación y gestión que completan las 440 a 500 horas globales estimadas del proyecto.

\subsubsection{Herramientas empleadas para la estimación}

El proceso de estimación se apoyó en tres herramientas complementarias que facilitaron planificación estructurada y validación colaborativa del cronograma propuesto. La herramienta principal consistió en Google Sheets como plataforma de tabulación y cálculo, donde se registraron todas las actividades descompuestas del WBS con distribución detallada de tiempos estimados por fase metodológica e integrante responsable, permitiendo visualización matricial de carga de trabajo y cálculo automático de totales agregados. Como segundo componente se estableció planificación semanal estimada fundamentada en disponibilidad real del equipo, proyectando dedicación efectiva de 8 a 12 horas por persona por semana considerando compromisos académicos paralelos, lo cual permitió traducir estimaciones totales en horas a cronograma calendario de 14 a 16 semanas de duración. Finalmente, se incorporó retroalimentación continua del director de proyecto Ing. Luis Gabriel Moreno Sandoval mediante revisión de estimaciones en reuniones de asesoría quincenal, validando viabilidad técnica de las cargas propuestas, identificando actividades subestimadas o sobreestimadas, y aprobando formalmente la distribución final del esfuerzo entre ambos integrantes.

\section{Inicio del proyecto}

\subsection{Entrenamiento del Personal}

Dado que el proyecto Observatorio de Demanda Laboral en Tecnología en Latinoamérica involucra herramientas avanzadas como modelos de lenguaje grandes, scraping web dinámico con JavaScript, procesamiento semántico especializado en español técnico, generación de embeddings multilingües y técnicas de clustering no supervisado basado en densidad, se ha establecido un plan de entrenamiento ligero pero técnicamente enfocado que permita nivelar conocimientos del equipo en aspectos técnicos esenciales del stack tecnológico sin comprometer significativamente el cronograma establecido de 14 a 16 semanas.

El entrenamiento será liderado por Nicolás Camacho, quien cuenta con mayor experiencia técnica previa en tecnologías clave del proyecto incluyendo Python para ciencia de datos, bibliotecas de NLP en español y frameworks de scraping web. Su rol de transferencia de conocimiento incluirá diseño de cápsulas formativas concisas y demostración práctica de implementaciones mediante cuatro mecanismos pedagógicos complementarios.

El primer mecanismo consiste en explicaciones prácticas durante reuniones técnicas semanales, donde Nicolás presentará en vivo el funcionamiento del código que esté desarrollando en cada fase, incluyendo justificación técnica de decisiones arquitectónicas, recomendaciones de buenas prácticas de ingeniería de software y anticipación de errores comunes detectados durante implementación. El segundo mecanismo emplea ejemplos rápidos y contextualizados mediante notebooks Jupyter o scripts Python breves ejecutados sobre datos sintéticos simples, ilustrando conceptos clave como configuración de spiders Scrapy, interacción con Selenium WebDriver, diseño de prompts estructurados para LLMs, generación de embeddings con SentenceTransformers, proyección dimensional con UMAP y parametrización de clustering HDBSCAN.

El tercer mecanismo facilita lecturas y recursos técnicos recomendados mediante curación de documentación oficial de bibliotecas utilizadas, artículos académicos clave de estado del arte y videos tutoriales cortos disponibles en español o inglés según el tema y nivel de dificultad técnica, compartidos en repositorio GitHub del proyecto con anotaciones contextualizadas. El cuarto mecanismo establece revisión cruzada continua de avances, donde los integrantes aplicarán conocimientos adquiridos directamente en sus tareas asignadas del WBS, recibiendo retroalimentación técnica detallada de Nicolás mediante comentarios en pull requests de GitHub y validación funcional de implementaciones durante sesiones de pair programming remoto.

La planificación detallada del entrenamiento interno por componente técnico se presenta a continuación:

\begin{table}[H]
\centering
\small
\begin{tabular}{|p{3.5cm}|p{2cm}|p{2.5cm}|p{3cm}|p{2.5cm}|}
\hline
\textbf{Habilidad o tema técnico} & \textbf{Integrantes a entrenar} & \textbf{Responsable del entrenamiento} & \textbf{Método principal} & \textbf{Tiempo estimado} \\
\hline
Scraping web con Scrapy, Selenium y Playwright & Alejandro & Nicolás & Ejemplos en vivo + código comentado & Semanas 2–3 \\
\hline
Procesamiento de lenguaje en español (spaCy, regex, taxonomías laborales) & Alejandro & Nicolás & Lecturas + demos & Semanas 3–5 \\
\hline
Uso de LLMs y diseño de prompts & Alejandro & Nicolás & Ejercicios guiados + discusión en grupo & Semanas 5–6 \\
\hline
Embeddings multilingües y reducción dimensional & Alejandro & Nicolás & Notebooks breves + aplicación directa & Semanas 6–7 \\
\hline
Clustering con HDBSCAN y UMAP & Alejandro & Nicolás & Ejemplo sintético + retroalimentación & Semana 8 \\
\hline
Visualización macro con Dash / Plotly & Alejandro & Nicolás & Revisión cruzada de visualizaciones & Semanas 9–10 \\
\hline
Documentación técnica y guía metodológica & Todo el equipo & Alejandro & Plantillas + validación por Nicolás & Permanente \\
\hline
\end{tabular}
\caption{Plan de Entrenamiento Interno}
\end{table}

Este entrenamiento se desarrollará de manera continua, práctica y orientada a resolver necesidades técnicas reales del proyecto mediante aprendizaje situado en contexto de desarrollo. No se contempla fase de formación teórica previa extensa desacoplada de implementación, sino que el aprendizaje técnico estará completamente integrado al flujo de trabajo del WBS, adaptado al tiempo disponible según cronograma y priorizando componentes más urgentes del pipeline según secuencia de dependencias entre fases.

Nicolás brindará adicionalmente apoyo técnico puntual cuando surjan dudas específicas de debugging, optimización de rendimiento o selección entre alternativas de implementación, fomentando cultura de colaboración técnica horizontal y mejora iterativa donde se valoran preguntas fundamentadas, exploración autónoma de documentación oficial y experimentación controlada con validación empírica de resultados. La documentación técnica del sistema, liderada por Alejandro Pinzón en su rol de responsable de documentación académica, incluirá además explicaciones simplificadas didácticas de procesos técnicos clave mediante diagramas de flujo y pseudocódigo, con el fin de reforzar aprendizaje colectivo del equipo, facilitar onboarding de futuros colaboradores y dejar registro permanente de decisiones técnicas críticas para futuras iteraciones o réplicas académicas del observatorio.

\subsection{Infraestructura}

Para el adecuado desarrollo del proyecto ``Observatorio de Demanda Laboral en Tecnología en Latinoamérica'', se requiere una infraestructura técnica que permita la implementación modular del sistema, desde la recolección de datos hasta el análisis semántico y la documentación de resultados. Esta infraestructura combina herramientas de software de código abierto, recursos computacionales locales y servicios colaborativos en la nube, con una planeación clara sobre su adquisición, configuración y mantenimiento.

Los detalles completos de herramientas de software, especificaciones de equipos, y actividades de obtención, despliegue y mantenimiento se encuentran documentados en las tablas 9, 10, 11, 12 y 13 del presente documento.

\section{Planes de Trabajo del Proyecto}

\subsection{Descomposición de Actividades}

La estructura de descomposición de actividades del proyecto (WBS - Work Breakdown Structure) se organiza por fases metodológicas, cada una con sus respectivas subactividades. La siguiente tabla representa la estructura general adoptada para la ejecución, según se detalla en la Tabla 14.

\subsection{Calendarización}

La calendarización define las fechas estimadas de inicio y finalización de cada fase principal, así como su secuencia de ejecución, distribuida en una duración total de 16 semanas, tal como se muestra en la Tabla 15 y la Figura 4 (Carta Gantt).

\begin{figure}[H]
\centering
\includegraphics[width=\textwidth]{figures/CartaGantt.png}
\caption{Carta Gantt del cronograma del proyecto (16 semanas)}
\label{fig:carta-gantt}
\end{figure}

\subsection{Asignación de Recursos}

Para cada fase principal del proyecto, se han identificado los recursos necesarios en términos de recursos humanos, software, hardware y documentación, con el fin de asegurar el cumplimiento eficiente de los objetivos. Los detalles completos se encuentran en la Tabla 16.

\subsection{Asignación de Presupuesto y Justificación}

Este proyecto no contempla flujo de dinero real ni remuneración alguna para los integrantes. Sin embargo, se presenta una estimación simbólica del costo técnico potencial, que refleja lo que implicaría económicamente replicar el esfuerzo si se ejecutara en un entorno profesional o institucional. Esta estimación es útil para evaluar necesidades técnicas, justificar decisiones y proyectar posibles inversiones en caso de escalamiento.

El presupuesto estimado se presenta en la Tabla 17, con un total aproximado de hasta \$8.600.000 COP en valor técnico para réplica profesional del proyecto.

\textit{Esta asignación tiene fines estimativos y no representa una solicitud ni gestión presupuestal formal. Ningún recurso económico será solicitado ni entregado a los integrantes.}
