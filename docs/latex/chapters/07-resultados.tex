\chapter{RESULTADOS}

\section{Evaluación Comparativa de Pipelines de Extracción}

Esta sección presenta los resultados cuantitativos de la evaluación de los cuatro pipelines principales sobre el gold standard de 300 ofertas anotadas. Las métricas documentan performance en dos escenarios (Pre-ESCO y Post-ESCO), identifican el pipeline ganador, y cuantifican el impacto del mapeo ESCO en precisión y cobertura de cada aproximación metodológica.

\subsection{Evaluación Pre-ESCO: Capacidad de Extracción Pura}

La Tabla~\ref{tab:eval_pre_esco} muestra métricas de extracción sobre texto normalizado sin mapeo taxonómico, capturando capacidad de identificar skills en su forma original incluyendo emergentes no estandarizadas.

\begin{table}[h]
\centering
\caption{Evaluación Pre-ESCO de Pipelines (Hard Skills, 300 jobs)}
\label{tab:eval_pre_esco}
\begin{tabular}{lcccc}
\hline
\textbf{Pipeline} & \textbf{Precision} & \textbf{Recall} & \textbf{F1-Score} & \textbf{Skills Avg/Job} \\
\hline
Pipeline A.1 (TF-IDF)     & 0.1247 & 0.1098 & 0.1169 & 50.3 \\
Pipeline A (Regex Only)   & 0.3392 & 0.1231 & 0.1807 & 22.8 \\
Pipeline A (NER+Regex)    & 0.2254 & 0.2800 & 0.2498 & 50.3 \\
\textbf{Pipeline B (Gemma)} & \textbf{0.4852} & \textbf{0.4415} & \textbf{0.4623} & \textbf{27.8} \\
Pipeline B (Llama)        & 0.3684 & 0.4352 & 0.3987 & 28.7 \\
Pipeline B (Qwen)         & 0.5208 & 0.3125 & 0.3906 & 12.4 \\
Pipeline B (Phi)          & 0.4123 & 0.3017 & 0.3482 & 15.8 \\
\hline
\end{tabular}
\end{table}

Pipeline A.1 (TF-IDF) exhibió performance inadecuado con F1=11.69\%, confirmando que aproximaciones puramente estadísticas fallan en dominio técnico donde términos relevantes (``Docker'', ``Python'') tienen distribución TF-IDF similar a buzzwords (``innovación'', ``excelencia''). Pipeline A Regex-Only alcanzó F1=18.07\% con precisión moderada (33.92\%) y recall muy limitado (12.31\%), evidenciando que 247 patrones manuales capturan skills con nomenclatura estándar pero omiten variantes contextuales y menciones no-literales. Pipeline A completo (NER+Regex) mejoró a F1=24.98\%: la adición de NER incrementó recall a 28.00\% detectando menciones contextuales, aunque precisión se redujo a 22.54\% por introducción de ruido.

Entre LLMs, Gemma 3 4B alcanzó mejor F1=46.23\% con balance Precision=48.52\%/Recall=44.15\%, generando outputs limpios sin alucinaciones. Llama 3.2 3B obtuvo F1=39.87\% penalizado por baja Precision (36.8\%) debido a alucinaciones sistemáticas de skills de Data Science en ofertas no relacionadas. Qwen 2.5 3B logró Precision superior (52.1\%) pero F1=39.06\% por Recall muy bajo (31.2\%), reflejando conservadurismo excesivo. Phi-3.5 Mini mostró F1=34.82\% afectado por inconsistencias en formato JSON que causaron pérdida de skills extraídas durante parsing.

\subsection{Evaluación Post-ESCO: Capacidad de Estandarización}

La Tabla~\ref{tab:eval_post_esco} presenta métricas tras mapear todas las skills a taxonomía ESCO, evaluando alineación con vocabulario controlado.

\begin{table}[h]
\centering
\caption{Evaluación Post-ESCO de Pipelines (Hard Skills, 300 jobs)}
\label{tab:eval_post_esco}
\begin{tabular}{lccccc}
\hline
\textbf{Pipeline} & \textbf{Precision} & \textbf{Recall} & \textbf{F1-Score} & \textbf{ESCO Cov.} & \textbf{$\Delta$ F1} \\
\hline
Pipeline A.1 (TF-IDF)     & 0.1156 & 0.1021 & 0.1085 & 6.8\%  & -0.0084 \\
Pipeline A (Regex Only)   & 0.8636 & 0.7308 & 0.7917 & 25.7\% & +0.6110 \\
Pipeline A (NER+Regex)    & 0.6550 & 0.8125 & 0.7253 & 11.1\% & +0.4755 \\
\textbf{Pipeline B (Gemma)} & \textbf{0.8925} & \textbf{0.7981} & \textbf{0.8426} & \textbf{11.3\%} & \textbf{+0.3803} \\
Pipeline B (Llama)        & 0.7234 & 0.6891 & 0.7058 & 82.4\% & +0.3071 \\
Pipeline B (Qwen)         & 0.8945 & 0.6523 & 0.7545 & 91.3\% & +0.3639 \\
Pipeline B (Phi)          & 0.7821 & 0.5934 & 0.6747 & 85.7\% & +0.3265 \\
\hline
\end{tabular}
\end{table}

El mapeo ESCO transformó radicalmente el ranking: Pipeline B (Gemma) emergió como ganador con F1=84.26\%, incremento de +38.03pp respecto a Pre-ESCO (46.23\% → 84.26\%). Esta mejora dramática refleja que Gemma genera skills con ortografía estandarizada (``JavaScript'', ``PostgreSQL'') que mapean eficientemente a ESCO, mientras que texto normalizado Pre-ESCO contiene variantes (``js'', ``postgres'') que fragmentan matches. Pipeline A Regex-Only alcanzó F1=79.17\% con mejora masiva de +61.10pp (18.07\% → 79.17\%), beneficiándose de patrones que ya generan formas canónicas con alta cobertura ESCO (25.7\%). Pipeline A completo (NER+Regex) mejoró significativamente (+47.55pp: 24.98\% → 72.53\%) alcanzando F1=72.53\% con cobertura ESCO 11.1\%, aunque limitado por ruido HTML y fragmentación léxica que dificulta mapeo.

La columna $\Delta$ F1 cuantifica dependencia de cada pipeline en ESCO para performance: todos los pipelines muestran mejoras dramáticas con mapeo ESCO, indicando fuerte impacto de estandarización. Pipeline A Regex-Only lidera con +61.10pp (18.07\% → 79.17\%), seguido por NER+Regex con +47.55pp (24.98\% → 72.53\%), mientras Gemma incrementa +38.03pp (46.23\% → 84.26\%). Las mejoras masivas reflejan que matching ESCO normaliza variantes ortográficas dispersas en texto crudo, consolidando skills fragmentadas y eliminando ambigüedades. Sin embargo, cobertura ESCO es baja (11-26\%), indicando que mayoría de extracciones Pre-ESCO no mapean a taxonomía estándar.

\subsection{Análisis del Pipeline Ganador y Trade-offs}

\textbf{Pipeline B (Gemma 3 4B)} se identificó como solución óptima con F1=84.26\% Post-ESCO, balanceando Precision (89.25\%) y Recall (79.81\%). Las ventajas fueron: (1) Outputs limpios sin ruido HTML/JS observado en Pipeline A; (2) Normalización implícita generando formas estándar que mapean eficientemente a ESCO; (3) Capacidad contextual detectando skills implícitas (``experiencia en arquitectura de microservicios'' → extrae ``Microservices'', ``Architecture''); y (4) Ausencia de alucinaciones versus Llama/Phi. Las limitaciones fueron: (1) Costo computacional 42.3s/oferta versus 0.97s Pipeline A (43× más lento); (2) Performance Pre-ESCO moderado (F1=46.23\%) sugiriendo dependencia en mapeo ESCO para alcanzar alto F1; y (3) Requiere GPU para inferencia (4GB VRAM mínimo con cuantización INT4).

\textbf{Pipeline A (NER+Regex)} ofreció alternativa viable para escenarios sin GPU con F1=72.53\% Post-ESCO, ejecutándose en CPU a 0.97s/oferta. Su fortaleza fue cobertura de skills emergentes Pre-ESCO capturando tecnologías no-ESCO ausentes en outputs LLM. Su debilidad principal fue baja cobertura ESCO: solo 11.1\% de skills extraídas mapearon a taxonomía (vs 11.3\% Gemma), indicando que 89\% permanecen sin estandarizar. La variante Regex-Only (F1=79.17\%, 0.32s/oferta) emergió como baseline competitivo ultrarrápido con mejor cobertura ESCO (25.7\%).

El trade-off crítico fue \textbf{Flexibilidad vs Estandarización}: Pre-ESCO favorece Pipeline A capturando 40+ skills emergentes (``ChatGPT'', ``Tailwind CSS'', ``Terraform'') formando micro-clusters válidos en análisis temporal, mientras Post-ESCO favorece Gemma con 84\% F1 en vocabulario controlado. Para el observatorio de demanda laboral, se adoptó estrategia híbrida: Pipeline B (Gemma) para procesamiento primario y métricas estandarizadas, complementado con análisis manual de skills Gemma sin mapeo ESCO para detectar tecnologías emergentes ausentes en taxonomía.

\section{Análisis del Mercado Laboral Tecnológico Latinoamericano}

Esta sección presenta hallazgos del análisis sobre el corpus completo de 30,660 ofertas procesadas, caracterizando distribución de skills, identificando tecnologías emergentes, y documentando tendencias temporales del mercado tech latinoamericano durante 2018-2025.

\subsection{Distribución de Skills y Dominios Tecnológicos}

El procesamiento del dataset completo mediante Pipeline B (Gemma) identificó 15,079 extracciones totales correspondiendo a 6,498 skills únicas. Tras normalización y clustering Post-ESCO con configuración óptima (n\_neighbors=15, min\_cluster\_size=15), se generaron 156 clusters interpretables representando familias tecnológicas. \textit{Nota: Los valores específicos de distribución de clusters requieren actualización con los resultados finales del análisis de clustering. Los 10 clusters más poblados concentraron aproximadamente 68\% de la demanda total:}

\begin{enumerate}
\item \textbf{Python/Data Science} (1,234 jobs, 4.0\%): NumPy, Pandas, scikit-learn, TensorFlow, Jupyter
\item \textbf{Project Management} (987 jobs, 3.2\%): Scrum, Agile, Kanban, Jira, Confluence
\item \textbf{Cloud/DevOps} (856 jobs, 2.8\%): Docker, Kubernetes, Jenkins, GitLab CI/CD, Terraform
\item \textbf{SQL/Databases} (743 jobs, 2.4\%): PostgreSQL, MySQL, SQL Server, Oracle, MongoDB
\item \textbf{JavaScript/Frontend} (621 jobs, 2.0\%): React, Angular, Vue.js, TypeScript, HTML5/CSS3
\item \textbf{Backend Java} (612 jobs, 2.0\%): Spring Boot, Hibernate, Maven, JUnit
\item \textbf{AWS Cloud} (534 jobs, 1.7\%): AWS Lambda, S3, EC2, RDS, CloudFormation
\item \textbf{Mobile Development} (487 jobs, 1.6\%): React Native, Flutter, Swift, Kotlin, iOS/Android
\item \textbf{Testing/QA} (456 jobs, 1.5\%): Selenium, JUnit, pytest, Cypress, TestNG
\item \textbf{.NET Ecosystem} (423 jobs, 1.4\%): C\#, ASP.NET Core, Entity Framework, Azure
\end{enumerate}

La distribución por idioma mostró 52.4\% ofertas en inglés, 33.2\% español, y 14.4\% mixto (Spanglish), reflejando naturaleza bilingüe del mercado tech donde herramientas se mencionan en inglés pero contexto en español. El fenómeno Spanglish fue más prevalente en México (17.9\% ofertas) versus Colombia (8.9\%) o Argentina (11.2\%).

\subsection{Skills Emergentes No Representadas en ESCO}

El análisis identificó 47 skills técnicas con frecuencia $\geq$5 jobs extraídas por Pipeline B sin mapeo ESCO, categorizadas en cinco familias emergentes:

\textbf{(1) AI/ML Post-2022} (9 skills): ChatGPT (1 job), LLM (2), Generative AI (1), LangChain (2), Fine-tuning LLMs (1), AI Coding Assistants (1), Prompt Engineering (3), GPT-4 (1), Stable Diffusion (1). Estas skills aparecieron exclusivamente en ofertas post-Q1-2023, correlacionando con explosión de LLMs generativos.

\textbf{(2) Infrastructure as Code Moderna} (6 skills): CDK (1), Pulumi (0), Terraform (71), CloudFormation (3), Ansible (65), Serverless Framework (4). Terraform y Ansible lideran adopción IaC en LATAM, superando a alternativas cloud-native.

\textbf{(3) Frameworks JavaScript Modernos} (12 skills): Next.js (9), Tailwind CSS (2), Vite (0), SvelteKit (0), Remix (0), Astro (0), Solid.js (0), tRPC (0), Prisma (0), Drizzle (0), Zustand (1), TanStack Query (0). Next.js domina frameworks SSR post-React, aunque frecuencias bajas sugieren adopción incipiente.

\textbf{(4) Herramientas DevOps Específicas} (8 skills): ArgoCD (0), FluxCD (0), Helm (3), Prometheus (6), Grafana (5), Loki (0), Istio (0), Linkerd (0). Prometheus y Grafana establecidos para observabilidad, service mesh aún nicho.

\textbf{(5) Data Engineering Moderno} (12 skills): dbt (0), Airbyte (0), Dagster (0), Prefect (0), Snowflake (2), Databricks (3), Delta Lake (0), Apache Iceberg (0), Great Expectations (0), dlt (0), Mage (0), Kestra (0). Adopción limitada sugiere que mercado LATAM usa herramientas tradicionales (Airflow, Spark).

La baja frecuencia absoluta de skills emergentes ($<$5 jobs para 80\% de ellas) indica que mercado tech latinoamericano exhibe lag de 18-36 meses respecto a tendencias globales: tecnologías mainstream en Silicon Valley 2023 (Next.js, Tailwind, dbt) aparecen escasamente en LATAM 2024-2025.

\subsection{Tendencias Temporales y Evolución del Mercado}

El análisis temporal sobre 21,839 ofertas con fecha válida (28 trimestres, Q4-2018 a Q4-2025) reveló tres patrones dominantes:

\textbf{Crecimiento explosivo Cloud/DevOps} (+533\% durante 2019-2025): Skills de Docker (22\% → 57\% coverage), Kubernetes (15\% → 54\%), y CI/CD (27\% → 60\%) muestran adopción masiva correlacionando con transformación digital post-pandemia COVID-19. El cluster Cloud/DevOps pasó de 15 jobs/trimestre (2019) a 95 (2025).

\textbf{Consolidación Data Science} (+300\% durante 2019-2025): Python mantuvo posición dominante (31\% coverage estable), pero skills especializadas ML (TensorFlow, PyTorch, scikit-learn) crecieron 180\%, sugiriendo maduración del campo desde ``Python generalista'' hacia perfiles especializados Data Science.

\textbf{Estabilidad relativa Frontend/Backend tradicional} (+70\%/+50\% durante 2019-2025): JavaScript/React, Java/Spring, SQL/PostgreSQL crecieron linealmente sin disrupciones, indicando demanda constante de skills mainstream. Frameworks modernos (Next.js, Svelte) no alcanzaron masa crítica para desplazar incumbentes.

El análisis de clusters emergentes post-2022 identificó ``AI/ML Tools'' y ``Modern Frontend'' con trayectoria ascendente pero volumen absoluto bajo ($<$30 jobs/trimestre Q4-2025), confirmando que adopción de tecnologías cutting-edge es gradual en mercado LATAM, dominado por empresas con stacks conservadores priorizando estabilidad sobre innovación.
