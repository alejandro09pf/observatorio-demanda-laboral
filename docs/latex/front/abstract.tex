\thispagestyle{fancy}

\begin{center}
    {\large\bfseries RESUMEN}
\end{center}

\vspace{1cm}

El mercado laboral tecnológico latinoamericano carece de sistemas automatizados para caracterizar la demanda de habilidades técnicas IT de manera sistemática y actualizada, enfrentando el desafío de capturar tecnologías emergentes que evolucionan más rápido que taxonomías oficiales. Este proyecto evaluó la viabilidad de construir un observatorio automatizado que recolectó 30,660 ofertas laborales de Colombia, México y Argentina mediante web scraping de siete portales, focalizándose en extracción de hard skills (lenguajes de programación, frameworks, herramientas cloud/DevOps, metodologías ágiles). Se implementó Pipeline A (NER + Regex) procesando el corpus completo con latencia de 0.97s por oferta, y se desarrolló Pipeline B experimental (LLM Gemma 3 4B) evaluado sobre gold standard de 300 ofertas anotadas manualmente con 6,174 hard skills. Los resultados demostraron superioridad de modelos de lenguaje para aproximar mapeo humano de competencias técnicas (F1=84.26\% vs 72.53\%), capturando habilidades implícitas inferibles del contexto y tecnologías emergentes ausentes en vocabularios controlados que métodos basados en patrones omiten. El sistema normalizó extracciones contra taxonomía ESCO v1.1.0 extendida con 276 habilidades técnicas modernas (152 O*NET + 124 curadas manualmente) mediante matcher conservador de dos capas (exacto + difuso threshold 0.92), alcanzando 12.6\% de cobertura. Experimentos con matcher enhanced aumentaron cobertura a ~25\% pero no se implementó en producción para mantener neutralidad entre pipelines, evidenciando que 70-87\% de habilidades detectadas son emergentes no presentes en taxonomías estándar (Next.js, Tailwind CSS, Terraform, Bun), validando necesidad de captura automática de vocabulario IT actual. El clustering no supervisado (UMAP reducción 768D a 2D + HDBSCAN density-based) identificó entre 34 y 53 familias tecnológicas en configuraciones optimizadas, sin categorías predefinidas. Los hallazgos validan tres conclusiones: primero, la viabilidad técnica de observatorios basados en scraping multi-portal; segundo, la superioridad de LLMs versus métodos deterministas para extracción semántica y detección de emergentes, con trade-off de costo computacional hasta 43$\times$ mayor; y por último, la obsolescencia crítica de taxonomías oficiales para vocabulario IT actual, evidenciando necesidad de actualización continua.
