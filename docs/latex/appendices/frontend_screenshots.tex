\section*{APÉNDICE G: CAPTURAS DE PANTALLA ADICIONALES DEL FRONTEND}
\addcontentsline{toc}{section}{Apéndice G: Capturas de Pantalla Adicionales del Frontend}

Este apéndice complementa el Capítulo 6 (Desarrollo), específicamente la Sección 6.3 que describe la implementación del frontend web del observatorio con arquitectura Next.js 14, React 18, TailwindCSS para diseño responsivo, shadcn/ui para componentes de interfaz, y Recharts para visualizaciones de datos. Mientras el capítulo principal presenta las tres interfaces fundamentales del sistema mediante las Figuras 6.3 (Dashboard con vista de skills más demandadas y distribución geográfica), 6.4 (Módulo de ofertas laborales con listado paginado y búsqueda de texto completo), y 6.5 (Vista detallada de habilidad individual con estadísticas y co-ocurrencias), este apéndice proporciona siete capturas de pantalla adicionales que documentan funcionalidades avanzadas no incluidas en el capítulo principal por restricciones de espacio: el sistema de filtrado multidimensional con cinco filtros simultáneos (país, método de extracción, estado del empleo, tipo de habilidad, estado de mapeo ESCO), las vistas de distribución de habilidades por país con barras de progreso indicando porcentaje de demanda regional, y el módulo completo de clustering de habilidades con selección de configuraciones entre 8 combinaciones de dataset y etapa ESCO.

Las capturas documentadas ilustran características críticas de usabilidad y analítica avanzada implementadas en el frontend: la Figura \ref{fig:dashboard-filtros-apendice} muestra el sistema de filtrado simultáneo que permite segmentación multidimensional de análisis; las Figuras \ref{fig:job-list-apendice} y \ref{fig:skill-detail-apendice} demuestran la navegación drill-down desde vistas agregadas hacia detalles individuales con paginación eficiente de grandes volúmenes de datos; las Figuras \ref{fig:skill-distribution-country-apendice} y \ref{fig:skill-availability-country-apendice} presentan dos perspectivas complementarias de análisis geográfico con métricas de frecuencia absoluta y relativa; y las Figuras \ref{fig:clustering-module-apendice} y \ref{fig:clustering-params-apendice} documentan la interfaz completa de experimentación con clustering HDBSCAN que expone métricas de calidad (silhouette score, Davies-Bouldin score) y permite comparación visual entre las 8 configuraciones experimentales (Manual Golden 300, Pipeline A Golden 300, Pipeline B Golden 300, Pipeline A Todos 30k × Pre-ESCO y Post-ESCO). Estas interfaces implementan los principios de diseño presentados en la Sección 6.3: componentes React modulares y reutilizables, Server-Side Rendering para optimización SEO y carga inicial rápida, diseño responsivo mobile-first mediante TailwindCSS, y visualizaciones interactivas con Recharts que permiten zoom, hover tooltips y exportación de gráficos.

\begin{figure}[H]
\centering
\includegraphics[width=0.95\textwidth]{figures/DashboarFiltrosFrontApendices.png}
\caption{Sistema de filtrado multidimensional del Dashboard mostrando los cinco filtros simultáneos disponibles: país, método de extracción, estado del empleo, tipo de habilidad y estado de mapeo ESCO.}
\label{fig:dashboard-filtros-apendice}
\end{figure}

\begin{figure}[H]
\centering
\includegraphics[width=0.95\textwidth]{figures/JobaddsFrontApendices.png}
\caption{Vista de listado de ofertas laborales con tabla paginada mostrando título, empresa, ubicación, portal de origen y fecha de publicación. El sistema permite búsqueda de texto completo y filtrado por país, portal y estado de procesamiento.}
\label{fig:job-list-apendice}
\end{figure}

\begin{figure}[H]
\centering
\includegraphics[width=0.95\textwidth]{figures/RequestedSkillFrontApendices.png}
\caption{Vista detallada de habilidad individual mostrando estadísticas generales, distribución geográfica, habilidades co-ocurrentes y lista paginada de empleos que requieren esa skill específica.}
\label{fig:skill-detail-apendice}
\end{figure}

\begin{figure}[H]
\centering
\includegraphics[width=0.95\textwidth]{figures/DistribucionDeHabilidadPorPaisApendices.png}
\caption{Distribución de habilidades por país mostrando barras de progreso que indican el porcentaje de demanda de una skill específica en cada país (Colombia, México, Argentina).}
\label{fig:skill-distribution-country-apendice}
\end{figure}

\begin{figure}[H]
\centering
\includegraphics[width=0.95\textwidth]{figures/DisponibilidadDeSkillPorPaisApendices.png}
\caption{Vista de disponibilidad de skills por país presentando estadísticas agregadas de habilidades demandadas en cada región con métricas de frecuencia absoluta y relativa.}
\label{fig:skill-availability-country-apendice}
\end{figure}

\begin{figure}[H]
\centering
\includegraphics[width=0.95\textwidth]{figures/ClusteringHabilidadesFrontApendices.png}
\caption{Módulo de clustering de habilidades mostrando selección de configuraciones (8 combinaciones de dataset y etapa ESCO), métricas de clustering (número de clusters, silhouette score, Davies-Bouldin score), y galería de visualizaciones interactivas.}
\label{fig:clustering-module-apendice}
\end{figure}

\begin{figure}[H]
\centering
\includegraphics[width=0.95\textwidth]{figures/ParametrosFrontApendices.png}
\caption{Interfaz de configuración de parámetros de clustering mostrando opciones de dataset (Manual Golden 300, Pipeline A Golden 300, Pipeline B Golden 300, Pipeline A Todos 30k) y etapa ESCO (pre y post) con métricas de evaluación.}
\label{fig:clustering-params-apendice}
\end{figure}
