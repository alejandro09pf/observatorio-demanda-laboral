\chapter{INTRODUCCIÓN}

El mercado laboral en América Latina atraviesa una transformación profunda impulsada por la digitalización de la economía. La pandemia de COVID-19 aceleró este proceso, intensificando la demanda de habilidades tecnológicas especializadas y exponiendo las brechas de capital humano en la región \cite{azuara2022}. En este escenario, identificar con precisión qué competencias están siendo requeridas por el mercado se ha vuelto estratégico para gobiernos que diseñan políticas de formación, instituciones educativas que ajustan sus programas, y profesionales que planifican su desarrollo de carrera.

Sin embargo, medir esta demanda de manera sistemática presenta desafíos importantes. Los portales de empleo en la región publican vacantes en formatos heterogéneos, sin vocabularios estandarizados, y con alta volatilidad \cite{echeverria2022}. Las encuestas tradicionales, aunque valiosas, suelen ser retrospectivas y de baja periodicidad, limitando su utilidad para capturar tendencias emergentes \cite{rubio2025}. Los estudios previos en países como Colombia, México y Argentina han aportado evidencia empírica importante, pero se han basado principalmente en análisis de frecuencia de términos y clasificaciones manuales, enfoques que no logran capturar la complejidad del lenguaje técnico ni identificar habilidades implícitas en las descripciones de las vacantes \cite{aguilera2018, martinez2024}.

Este proyecto desarrolló un Observatorio de Demanda Laboral para América Latina, un sistema automatizado que recolecta, procesa y analiza ofertas de empleo a escala regional. El observatorio integra múltiples fuentes de información en Colombia, México y Argentina, aplicando técnicas de inteligencia artificial para extraer y estructurar las habilidades demandadas. La solución combina métodos tradicionales de procesamiento de texto con modelos de lenguaje avanzados, permitiendo tanto la identificación de competencias mencionadas explícitamente como la inferencia de aquellas que se derivan del contexto del cargo. El sistema normaliza los resultados contra taxonomías internacionales reconocidas y aplica técnicas de agrupamiento para descubrir perfiles laborales emergentes, generando visualizaciones y reportes que facilitan la comprensión de la dinámica del mercado.

El valor principal de este proyecto reside en tres aspectos. Primero, su escala regional y enfoque multipaís, que permite comparaciones sistemáticas entre mercados. Segundo, su arquitectura dual que combina precisión en la extracción de términos conocidos con capacidad de descubrimiento de competencias nuevas. Tercero, su adaptación específica al contexto latinoamericano, atendiendo las particularidades del español técnico y la mezcla con anglicismos característica del sector tecnológico en la región.
